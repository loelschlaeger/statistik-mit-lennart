\documentclass[t,11pt]{beamer}
\usepackage{tikz}
\usepackage{pgfplots}
\usetikzlibrary{calc}
\usepackage[utf8]{inputenc}
\usepackage[ngerman]{babel}
\usepackage{amsmath,amsfonts,amssymb}
\usepackage{framed}
\usecolortheme{orchid}
\usepackage{etoolbox}
\useinnertheme[shadow=true]{rounded}

%%% PROGRESSBAR
\definecolor{pbblue}{HTML}{D8D8D8}% filling color for the progress bar
\definecolor{pbgray}{HTML}{F2F2F2}% background color for the progress bar
\useoutertheme{infolines}
\setbeamerfont{footline}{size=\normalsize}
\setbeamersize{text margin left=30pt,text margin right=30pt}
\makeatletter
\setbeamertemplate{footline}
{
	\leavevmode%
	\hbox{%
		\begin{beamercolorbox}[wd=.333333\paperwidth,ht=2.5ex,dp=1ex,center]{title in head/foot}%
			\usebeamerfont{title in head/foot}\insertshorttitle
		\end{beamercolorbox}%
		\begin{beamercolorbox}[wd=.333333\paperwidth,ht=2.5ex,dp=1ex,center]{date in head/foot}%
			%\usebeamerfont{date in head/foot}\insertshortdate{}\hspace*{2em}
			%\insertframenumber\hspace*{2ex} 
		\end{beamercolorbox}
		\begin{beamercolorbox}[wd=.333333\paperwidth,ht=3ex,dp=1ex,center]{author in head/foot}%
			\usebeamerfont{author in head/foot}\insertshortauthor~~%\beamer@ifempty{\insertshortinstitute}{}{(\insertshortinstitute)}
		\end{beamercolorbox}%
	}%
	\vskip0pt%
}
\makeatother
\makeatletter
\def\progressbar@progressbar{} % the progress bar
\newcount\progressbar@tmpcounta% auxiliary counter
\newcount\progressbar@tmpcountb% auxiliary counter
\newdimen\progressbar@pbht %progressbar height
\newdimen\progressbar@pbwd %progressbar width
\newdimen\progressbar@tmpdim % auxiliary dimension
\progressbar@pbwd=\linewidth
\progressbar@pbht=1.5ex
\def\progressbar@progressbar{%
    \progressbar@tmpcounta=\insertframenumber
    \progressbar@tmpcountb=\inserttotalframenumber
    \progressbar@tmpdim=\progressbar@pbwd
    \multiply\progressbar@tmpdim by \progressbar@tmpcounta
    \divide\progressbar@tmpdim by \progressbar@tmpcountb
  \begin{tikzpicture}[rounded corners=3pt,very thin]
    \shade[top color=pbgray!20,bottom color=pbgray!20,middle color=pbgray!50]
      (0pt, 0pt) rectangle ++ (\progressbar@pbwd, \progressbar@pbht);
      \shade[draw=pbblue,top color=pbblue!50,bottom color=pbblue!50,middle color=pbblue] %
        (0pt, 0pt) rectangle ++ (\progressbar@tmpdim, \progressbar@pbht);
    \draw[color=normal text.fg!50]  
      (0pt, 0pt) rectangle (\progressbar@pbwd, \progressbar@pbht) 
        node[pos=0.5,color=normal text.fg] {\textnormal{%
             \pgfmathparse{\insertframenumber*100/\inserttotalframenumber}%
             \pgfmathprintnumber[fixed,precision=0]{\pgfmathresult}\,\%%
        }%
    };
  \end{tikzpicture}%
}
\addtobeamertemplate{headline}{}
{%
  \begin{beamercolorbox}[wd=\paperwidth,ht=4ex,center,dp=1ex]{white}%
    \progressbar@progressbar%
  \end{beamercolorbox}%
}
\makeatother


%%% BLOCKS
% block = Aufgabe
\setbeamercolor{block title}{fg=black,bg=blue!30!white} 
\setbeamercolor{block body}{fg=black, bg=blue!3!white}

% alertblock = Definition
\setbeamercolor{block title alerted}{fg=black,bg=red!50!white} 
\setbeamercolor{block body alerted}{fg=black, bg=red!3!white}

% exampleblock = Wiederholung
\setbeamercolor{block title example}{fg=black,bg=green!30!white} 
\setbeamercolor{block body example}{fg=black, bg=green!3!white}



\begin{document}
	%\author{www.oilbat.de}
	%\title{Stochastik}
	\subtitle{}
	\logo{}
	\institute{}
	\date{}
	\subject{}
	\setbeamercovered{transparent}
	\setbeamertemplate{navigation symbols}{}

\addtocounter{framenumber}{-1}
\setbeamercovered{invisible}

\begin{frame}
\begin{alertblock}{Neyman-Pearson Test: Ausgangssituation}
	\begin{itemize}[<+->]
		\item Wir haben eine \textit{iid} Stichprobe $X_1,X_2,\dots,X_n$.
		\item Sie stammt entweder aus der Dichte $f_0$ oder $f_1$. Welche stimmt?
		\item Entscheidungsfunktion $d:\mathbb{R}^n\to\{0,1 \}$
		\begin{itemize}
			\item $d(X_1,X_2,\dots,X_n)=0\Rightarrow$ wir entscheiden uns für $f_0$
			\item $d(X_1,X_2,\dots,X_n)=1\Rightarrow$ wir entscheiden uns für $f_1$
		\end{itemize}
		\item Fehler 1. Art und Fehler 2. Art
		\begin{itemize}
			\item $\mathbf{P}(d=1\mid f_0$ stimmt$)=\alpha$
			\item $\mathbf{P}(d=0\mid f_1$ stimmt$)=\beta$
		\end{itemize}
	\end{itemize}
\end{alertblock}
\end{frame}

\begin{frame}
\begin{alertblock}{Neyman-Pearson Test: Ausgangssituation}
	\begin{itemize}%[<+->]
		\item Wir haben eine \textit{iid} Stichprobe $X_1,X_2,\dots,X_n$.
		\item Sie stammt entweder aus der Dichte $f_0$ oder $f_1$. Welche stimmt?
		\item Entscheidungsfunktion $d:\mathbb{R}^n\to\{0,1 \}$
		\begin{itemize}
			\item $d(X_1,X_2,\dots,X_n)=0\Rightarrow$ wir entscheiden uns für $f_0$
			\item $d(X_1,X_2,\dots,X_n)=1\Rightarrow$ wir entscheiden uns für $f_1$
		\end{itemize}
		\item Fehler 1. Art und Fehler 2. Art
		\begin{itemize}
			\item $\mathbf{P}(d=1\mid f_0$ stimmt$)=\alpha$
			\item $\mathbf{P}(d=0\mid f_1$ stimmt$)=\beta$
		\end{itemize}
	\end{itemize}
\end{alertblock}
\begin{exampleblock}{Unser Ziel}
	Finde einen Test, der 
	\begin{itemize}
		\item maximal mit Wahrscheinlichkeit $\alpha$ einen Fehler 1. Art macht,
		\item die Wahrscheinlichkeit $\beta$ für einen Fehler 2. Art minimiert.
	\end{itemize}
\end{exampleblock}
\end{frame}

\begin{frame}
	\begin{alertblock}{Die Idee von Neyman und Pearson}
		\begin{itemize}[<+->]
			\item Falls $f_1(X_1,X_2,\dots,X_n)$ deutlich größer ist als $f_0(X_1,X_2,\dots,X_n)$, so scheint $f_1$ die korrekte Dichte zu sein (Maximum-Likelihood Prinzip).
			\item Betrachte den Likelihood-Quotienten $$\frac{f_1(X_1,X_2,\dots,X_n)}{f_0(X_1,X_2,\dots,X_n)}.$$
			\item Falls dieser größer als ein Schwellwert $K$ ist, so entscheide dich für $f_1$, ansonsten entscheide dich für $f_0$.
			\item Wähle $K$ so, dass
			$$ \mathbf{P}\left(\frac{f_1(X_1,X_2,\dots,X_n)}{f_0(X_1,X_2,\dots,X_n)}>K~\bigg\vert~ f_0\text{ stimmt}\right)=\alpha .$$
		\end{itemize}
	\end{alertblock}
\end{frame}

\begin{frame}
	\begin{alertblock}{Das Neyman-Pearson Lemma}
		Gegeben
		\begin{itemize}
			\item $H_0:$ Wahre Dichte ist $f_0$, $H_1:$ Wahre Dichte ist $f_1$
			\item Signifikanzniveau $\alpha$
			\item Schwellwert $K$, sodass $ \mathbf{P}\left(\frac{f_1(X_1,X_2,\dots,X_n)}{f_0(X_1,X_2,\dots,X_n)}>K~\bigg\vert~ f_0\text{ stimmt}\right)=\alpha $
			\item Entscheidungsregel $d$, sodass
			\begin{itemize}
				\item $\frac{f_1(X_1,X_2,\dots,X_n)}{f_0(X_1,X_2,\dots,X_n)}<K \Rightarrow d=0$
				\item $\frac{f_1(X_1,X_2,\dots,X_n)}{f_0(X_1,X_2,\dots,X_n)}>K \Rightarrow d=1$
			\end{itemize}
		\end{itemize}
			Dann 
			\begin{itemize}
				\item ist $d$ ein Test, der unter allen Tests zum Signifikanzniveau $\alpha$ den Fehler 2. Art minimiert. (Neyman-Pearson Test)
			\end{itemize}
	\end{alertblock}
\end{frame}


\begin{frame}
	\begin{block}{Beispiel: IQ-Werte}
		\begin{itemize}[<+->]
			\item IQ-Werte von Erwachsenen seien normalverteilt mit $\mu_0=100$ und $\sigma^2 =15$
			\item Es wird behauptet, dass Studenten im Schnitt einen um 10\% höheren IQ besitzen als der durchschnittliche Erwachsene
			\item Teste also
			\begin{itemize}
				\item $H_0:$ $\mathcal{N}(\mu_0=100,\sigma^2=15)$ gegen
				\item $H_1:$ $\mathcal{N}(\mu_1=110,\sigma^2=15)$
			\end{itemize}
			\item wähle Signifikanzniveau, z.B. $\alpha=5\%$
			\item Gegeben seien die IQ-Werte $X_1,X_2,\dots,X_{10}$.
			\item Deren gemeinsame Dichte ist (mit $\sigma^2=15$ und $j=0,1$)
			$$f_j(X_1,X_2,\dots,X_{10})=\prod_{i=1}^{10}\frac{1}{\sqrt{2\pi\sigma^2}}\exp \left[-\frac{(X_i-\mu_j)^2}{2\sigma^2}\right]$$
		\end{itemize}
	\end{block}
\end{frame}

\begin{frame}
Mittels einiger Umformungen:
\begin{align*}
&f_j(X_1,X_2,\dots,X_{10})\\&=\prod_{i=1}^{10}\frac{1}{\sqrt{2\pi\sigma^2}}\exp \left[-\frac{(X_i-\mu_j)^2}{2\sigma^2}\right]\\
&= ...\\
&= \left( \frac{1}{\sqrt{2\pi\sigma^2}} \right)^{10}\exp \left[ -\frac{1}{2\sigma^2} \left( \sum_{i=1}^{10} X_i^2 - 20\mu_j\bar{X} + 10\mu_j^2  \right)  \right] 
\end{align*} 
\end{frame}

\begin{frame}
	Mittels einiger Umformungen:
	\begin{align*}
		&f_j(X_1,X_2,\dots,X_{10})\\&=\prod_{i=1}^{10}\frac{1}{\sqrt{2\pi\sigma^2}}\exp \left[-\frac{(X_i-\mu_j)^2}{2\sigma^2}\right]\\
		&= ...\\
		&= \left( \frac{1}{\sqrt{2\pi\sigma^2}} \right)^{10}\exp \left[ -\frac{1}{2\sigma^2} \left( \sum_{i=1}^{10} X_i^2 - 20\mu_j\bar{X} + 10\mu_j^2  \right)  \right] 
	\end{align*} 
	Damit erhalten wir den Likelihood-Quotienten: 
	\begin{align*}
		\frac{f_1(X_1,X_2,\dots,X_{10})}{f_0(X_1,X_2,\dots,X_{10})}&=\exp \left[ - \frac{1}{2\sigma^2} \left( -20(\mu_1-\mu_0)\bar{X}+10(\mu_1^2-\mu_0^2)  \right)  \right] \\
		&= \exp \left[ \frac{20}{3}\bar{X}-7000  \right]
	\end{align*}
	(Beachte: $\sigma^2=15,\mu_0=100,\mu_1=110$)
\end{frame}

\begin{frame}
Wir müssen jetzt den Schwellwert $K$ finden, sodass unter $H_0$
\begin{align*}
\mathbf{P}\left(\frac{f_1(X_1,X_2,\dots,X_{10})}{f_0(X_1,X_2,\dots,X_{10})}>K~\bigg\vert~ f_0\text{ stimmt}\right)=\alpha, 
\end{align*}
also
\begin{align*}
\mathbf{P}\left(\exp \left[ \frac{20}{3}\bar{X}-7000  \right]>K~\bigg\vert~ f_0\text{ stimmt}\right)=5\%.
\end{align*}
\end{frame}

\begin{frame}
	Wir müssen jetzt den Schwellwert $K$ finden, sodass unter $H_0$
	\begin{align*}
		\mathbf{P}\left(\frac{f_1(X_1,X_2,\dots,X_{10})}{f_0(X_1,X_2,\dots,X_{10})}>K~\bigg\vert~ f_0\text{ stimmt}\right)=\alpha, 
	\end{align*}
	also
	\begin{align*}
	\mathbf{P}\left(\exp \left[ \frac{20}{3}\bar{X}-7000  \right]>K~\bigg\vert~ f_0\text{ stimmt}\right)=5\%.
	\end{align*}
	Es gilt
	\begin{align*}
		\exp \left[ \frac{20}{3}\bar{X}-7000  \right]>K \Leftrightarrow \bar{X}>\frac{3}{20}(7000+\ln(K)).
	\end{align*}
\end{frame}

\begin{frame}
Wir müssen jetzt den Schwellwert $K$ finden, sodass unter $H_0$
\begin{align*}
\mathbf{P}\left(\frac{f_1(X_1,X_2,\dots,X_{10})}{f_0(X_1,X_2,\dots,X_{10})}>K~\bigg\vert~ f_0\text{ stimmt}\right)=\alpha, 
\end{align*}
also
\begin{align*}
\mathbf{P}\left(\exp \left[ \frac{20}{3}\bar{X}-7000  \right]>K~\bigg\vert~ f_0\text{ stimmt}\right)=5\%.
\end{align*}
Es gilt
\begin{align*}
\exp \left[ \frac{20}{3}\bar{X}-7000  \right]>K \Leftrightarrow \bar{X}>\frac{3}{20}(7000+\ln(K)).
\end{align*}
Unter $H_0$ gilt
\begin{align*}
\bar{X}\sim \mathcal{N}\left(\mu_0=100,\sigma^2=\frac{15}{10}\right)
\end{align*}
\end{frame}

\begin{frame}
Wir müssen also $K$ so wählen, dass
\begin{align*}
\mathbf{P}\left(\bar{X}>\frac{3}{20}(7000+\ln(K))\right)=5\%, 
\end{align*}
unter der Bedingung, dass $\bar{X}\sim \mathcal{N}(\mu_0=100,\sigma^2=\frac{15}{10})$. 

\end{frame}

\begin{frame}
Wir müssen also $K$ so wählen, dass
\begin{align*}
\mathbf{P}\left(\bar{X}>\frac{3}{20}(7000+\ln(K))\right)=5\%, 
\end{align*}
unter der Bedingung, dass $\bar{X}\sim \mathcal{N}(\mu_0=100,\sigma^2=\frac{15}{10})$. 

\vspace{0.5cm}

Mit anderen Worten, wir suchen das 95\%-Quantil der $\mathcal{N}(\mu_0=100,\sigma^2=\frac{15}{10})$-Verteilung. Dies ist

\begin{center}
	\texttt{qnorm(0.95,mean=100,sd=sqrt(1.5))}$=102$
\end{center}

\end{frame}

\begin{frame}
	Wir müssen also $K$ so wählen, dass
	\begin{align*}
		\mathbf{P}\left(\bar{X}>\frac{3}{20}(7000+\ln(K))\right)=5\%, 
	\end{align*}
	unter der Bedingung, dass $\bar{X}\sim \mathcal{N}(\mu_0=100,\sigma^2=\frac{15}{10})$. 
	
	\vspace{0.5cm}
	
	Mit anderen Worten, wir suchen das 95\%-Quantil der $\mathcal{N}(\mu_0=100,\sigma^2=\frac{15}{10})$-Verteilung. Dies ist
	
	\begin{center}
		\texttt{qnorm(0.95,mean=100,sd=sqrt(1.5))}$=102$
	\end{center}

	Damit haben wir den Neyman-Pearson Test gefunden:
	
	\vspace{0.5cm}
	
	Verwerfe $H_0:$ $\mathcal{N}(\mu_0=100,\sigma^2=15)$ (bzw. $d=1$), falls $\bar{X}>102$.
	
\end{frame}


\end{document}