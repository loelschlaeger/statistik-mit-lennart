\documentclass[t,11pt]{beamer}
\usepackage{tikz}
\usetikzlibrary{calc}
\usepackage[utf8]{inputenc}
\usepackage[ngerman]{babel}
\usepackage{amsmath}
\usepackage{amsfonts}
\usepackage{amssymb}
\usepackage{framed}
\colorlet{LightBlue}{blue}

% Theorem box
\pgfdeclarelayer{background}
\pgfsetlayers{background,main}
\tikzstyle{thmbox} = [inner sep=1em]
\tikzstyle{thmborder} = [draw=blue, fill=none,line width =1.pt, rounded corners=5pt]

\def\parchmentframe#1{
	\tikz{
		\node[thmbox] (A) {#1};
		\begin{pgfonlayer}{background}
			\fill[thmborder] 
			(A.south east) -- (A.south west) -- 
			(A.north west) -- (A.north east) -- cycle;
\end{pgfonlayer}}}

\def\parchmentframetop#1{
	\tikz{
		\node[thmbox] (A) {#1};
		\begin{pgfonlayer}{background}    
			\fill[thmborder]
			(A.south west) -- (A.north west) -- 
			(A.north east) -- (A.south east);
\end{pgfonlayer}}}

\def\parchmentframebottom#1{
	\tikz{
		\node[thmbox] (A) {#1};
		\begin{pgfonlayer}{background}    
			\fill[thmborder]
			(A.north west) -- (A.south west) -- 
			(A.south east) -- (A.north east);
\end{pgfonlayer}}}

\def\parchmentframemiddle#1{
	\tikz{
		\node[thmbox] (A) {#1};
		\begin{pgfonlayer}{background}    
			\fill[thmborder]
			(A.north west) -- (A.south west);
			\fill[thmborder]
			(A.south east) -- (A.north east);
\end{pgfonlayer}}}

\newenvironment<>{myTheorem}[1][]{%
	\def\FrameCommand{\parchmentframe}%
	\def\FirstFrameCommand{\parchmentframetop}%
	\def\LastFrameCommand{\parchmentframebottom}%
	\def\MidFrameCommand{\parchmentframemiddle}%
	\vskip\baselineskip
	\MakeFramed{\FrameRestore}
	\noindent\tikz\node[inner sep=1.2ex, draw=blue, fill=blue!10,
	anchor=west, overlay, line width = 1.pt, rounded corners=4pt] at (0em, 1em) 
	{\color{LightBlue}{Aufgabe\if\relax\detokenize{#1}\relax\else\space (#1)\fi}};\par\nobreak}%
{\endMakeFramed}
%%end theorem box

\definecolor{pbblue}{HTML}{D8D8D8}% filling color for the progress bar
\definecolor{pbgray}{HTML}{F2F2F2}% background color for the progress bar

\useoutertheme{infolines}
\setbeamerfont{footline}{size=\normalsize}
\setbeamersize{text margin left=30pt,text margin right=30pt}
\makeatletter
\setbeamertemplate{footline}
{
	\leavevmode%
	\hbox{%
		\begin{beamercolorbox}[wd=.333333\paperwidth,ht=2.5ex,dp=1ex,center]{title in head/foot}%
			\usebeamerfont{title in head/foot}\insertshorttitle
		\end{beamercolorbox}%
		\begin{beamercolorbox}[wd=.333333\paperwidth,ht=2.5ex,dp=1ex,center]{date in head/foot}%
			%\usebeamerfont{date in head/foot}\insertshortdate{}\hspace*{2em}
			%\insertframenumber\hspace*{2ex} 
		\end{beamercolorbox}
		\begin{beamercolorbox}[wd=.333333\paperwidth,ht=3ex,dp=1ex,center]{author in head/foot}%
			\usebeamerfont{author in head/foot}\insertshortauthor~~%\beamer@ifempty{\insertshortinstitute}{}{(\insertshortinstitute)}
		\end{beamercolorbox}%
	}%
	\vskip0pt%
}
\makeatother


\makeatletter
\def\progressbar@progressbar{} % the progress bar
\newcount\progressbar@tmpcounta% auxiliary counter
\newcount\progressbar@tmpcountb% auxiliary counter
\newdimen\progressbar@pbht %progressbar height
\newdimen\progressbar@pbwd %progressbar width
\newdimen\progressbar@tmpdim % auxiliary dimension

\progressbar@pbwd=\linewidth
\progressbar@pbht=1.5ex

% the progress bar
\def\progressbar@progressbar{%

    \progressbar@tmpcounta=\insertframenumber
    \progressbar@tmpcountb=\inserttotalframenumber
    \progressbar@tmpdim=\progressbar@pbwd
    \multiply\progressbar@tmpdim by \progressbar@tmpcounta
    \divide\progressbar@tmpdim by \progressbar@tmpcountb

  \begin{tikzpicture}[rounded corners=3pt,very thin]

    \shade[top color=pbgray!20,bottom color=pbgray!20,middle color=pbgray!50]
      (0pt, 0pt) rectangle ++ (\progressbar@pbwd, \progressbar@pbht);

      \shade[draw=pbblue,top color=pbblue!50,bottom color=pbblue!50,middle color=pbblue] %
        (0pt, 0pt) rectangle ++ (\progressbar@tmpdim, \progressbar@pbht);

    \draw[color=normal text.fg!50]  
      (0pt, 0pt) rectangle (\progressbar@pbwd, \progressbar@pbht) 
        node[pos=0.5,color=normal text.fg] {die Aufgabe ist zu \textnormal{%
             \pgfmathparse{\insertframenumber*100/\inserttotalframenumber}%
             \pgfmathprintnumber[fixed,precision=0]{\pgfmathresult}\,\% gelöst%
        }%
    };
  \end{tikzpicture}%
}

\addtobeamertemplate{headline}{}
{%
  \begin{beamercolorbox}[wd=\paperwidth,ht=4ex,center,dp=1ex]{white}%
    \progressbar@progressbar%
  \end{beamercolorbox}%
}
\makeatother

\begin{document}
	\author{www.oilbat.de}
	%\title{Stochastik}
	\subtitle{}
	\logo{}
	\institute{}
	\date{}
	\subject{}
	\setbeamercovered{transparent}
	\setbeamertemplate{navigation symbols}{}

\addtocounter{framenumber}{-1}
\setbeamercovered{invisible}

\begin{frame}
\begin{myTheorem}
Berechnen Sie ein geeignetes Zusammenhangsmaß der folgenden zwei Datenreihen.
\end{myTheorem}

\end{frame}

\begin{frame}
Zusammenhangsmaße:
\begin{itemize}[<+->]
	\item Für zwei nominale Variablen: (korrigierter) Kontingenzkoeffizient
	\item Für zwei ordinale Variablen: Spearmans Rangkorrelationskoeffizient
	\item Für zwei metrische Variablen: Bestimmtheitsmaß 
\end{itemize}
\end{frame}

\begin{frame}
Zusammenhangsmaße:
\begin{itemize}
	\item Für zwei nominale Variablen: (korrigierter) Kontingenzkoeffizient
	\item Für zwei ordinale Variablen: Spearmans Rangkorrelationskoeffizient
	\item Für zwei metrische Variablen: Bestimmtheitsmaß 
\end{itemize}

\end{frame}

\begin{frame}
Spearmans Rangkorrelationskoeffizient
\begin{center}
	\renewcommand{\arraystretch}{1.2}
	\begin{tabular}{cccccc}
		$x$ & $y$ & $rg(x)$ & $rg(y)$ & $rg(x)-rg(y)$ &  $(rg(x)-rg(y))^2$ \\
		\hline
		1 & 2 &  &  &  & \\
		3 & 4 &  &  &  &\\
		5 & 4 &  &  &  & \\
		4 & 3 &  &  &  & \\
		5 & 5 &  &  &  & \\
		5 & 4 &  &  &  &  \\
		1 & 2 &  &  &  & \\
		2 & 3 &  &  &  & \\
		2 & 2 &  &  & & \\
		3 & 4 &  &  &  & \\
		\hline
		$\sum$ &&&&& 
	\end{tabular}
\end{center}
\end{frame}

\begin{frame}
Spearmans Rangkorrelationskoeffizient
\begin{center}
	\renewcommand{\arraystretch}{1.2}
	\begin{tabular}{cccccc}
		$x$ & $y$ & $rg(x)$ & $rg(y)$ & $rg(x)-rg(y)$ &  $(rg(x)-rg(y))^2$ \\
		\hline
		\textbf{1} & 2 &  &  &  & \\
		3 & 4 &  &  &  &\\
		5 & 4 &  &  &  & \\
		4 & 3 &  &  &  & \\
		5 & 5 &  &  &  & \\
		5 & 4 &  &  &  &  \\
		\textbf{1} & 2 &  &  &  & \\
		2 & 3 &  &  &  & \\
		2 & 2 &  &  & & \\
		3 & 4 &  &  &  & \\
		\hline
		$\sum$ &&&&& 
	\end{tabular}
\end{center}
\end{frame}

\begin{frame}
Spearmans Rangkorrelationskoeffizient
\begin{center}
	\renewcommand{\arraystretch}{1.2}
	\begin{tabular}{cccccc}
		$x$ & $y$ & $rg(x)$ & $rg(y)$ & $rg(x)-rg(y)$ &  $(rg(x)-rg(y))^2$ \\
		\hline
		\textbf{1} & 2 & \textbf{1,5} &  &  & \\
		3 & 4 &  &  &  &\\
		5 & 4 &  &  &  & \\
		4 & 3 &  &  &  & \\
		5 & 5 &  &  &  & \\
		5 & 4 &  &  &  &  \\
		\textbf{1} & 2 & \textbf{1,5} &  &  & \\
		2 & 3 &  &  &  & \\
		2 & 2 &  &  & & \\
		3 & 4 &  &  &  & \\
		\hline
		$\sum$ &&&&& 
	\end{tabular}
\end{center}
\end{frame}

\begin{frame}
Spearmans Rangkorrelationskoeffizient
\begin{center}
	\renewcommand{\arraystretch}{1.2}
	\begin{tabular}{cccccc}
		$x$ & $y$ & $rg(x)$ & $rg(y)$ & $rg(x)-rg(y)$ &  $(rg(x)-rg(y))^2$ \\
		\hline
		1 & 2 & 1,5  &  &  & \\
		3 & 4 &  &  &  &\\
		5 & 4 &  &  &  & \\
		4 & 3 &  &  &  & \\
		5 & 5 &  &  &  & \\
		5 & 4 &  &  &  &  \\
		1 & 2 & 1,5 &  &  & \\
		\textbf{2} & 3 &  &  &  & \\
		\textbf{2} & 2 &  &  & & \\
		3 & 4 &  &  &  & \\
		\hline
		$\sum$ &&&&& 
	\end{tabular}
\end{center}
\end{frame}

\begin{frame}
Spearmans Rangkorrelationskoeffizient
\begin{center}
	\renewcommand{\arraystretch}{1.2}
	\begin{tabular}{cccccc}
		$x$ & $y$ & $rg(x)$ & $rg(y)$ & $rg(x)-rg(y)$ &  $(rg(x)-rg(y))^2$ \\
		\hline
		1 & 2 & 1,5  &  &  & \\
		3 & 4 &  &  &  &\\
		5 & 4 &  &  &  & \\
		4 & 3 &  &  &  & \\
		5 & 5 &  &  &  & \\
		5 & 4 &  &  &  &  \\
		1 & 2 & 1,5 &  &  & \\
		\textbf{2} & 3 & \textbf{3,5} &  &  & \\
		\textbf{2} & 2 & \textbf{3,5} &  & & \\
		3 & 4 &  &  &  & \\
		\hline
		$\sum$ &&&&& 
	\end{tabular}
\end{center}
\end{frame}

\begin{frame}
Spearmans Rangkorrelationskoeffizient
\begin{center}
	\renewcommand{\arraystretch}{1.2}
	\begin{tabular}{cccccc}
		$x$ & $y$ & $rg(x)$ & $rg(y)$ & $rg(x)-rg(y)$ &  $(rg(x)-rg(y))^2$ \\
		\hline
		1 & 2 & 1,5  &  &  & \\
		\textbf{3} & 4 &  &  &  &\\
		5 & 4 &  &  &  & \\
		4 & 3 &  &  &  & \\
		5 & 5 &  &  &  & \\
		5 & 4 &  &  &  &  \\
		1 & 2 & 1,5 &  &  & \\
		2 & 3 & 3,5 &  &  & \\
		2 & 2 & 3,5 &  & & \\
		\textbf{3} & 4 &  &  &  & \\
		\hline
		$\sum$ &&&&& 
	\end{tabular}
\end{center}
\end{frame}

\begin{frame}
Spearmans Rangkorrelationskoeffizient
\begin{center}
	\renewcommand{\arraystretch}{1.2}
	\begin{tabular}{cccccc}
		$x$ & $y$ & $rg(x)$ & $rg(y)$ & $rg(x)-rg(y)$ &  $(rg(x)-rg(y))^2$ \\
		\hline
		1 & 2 & 1,5  &  &  & \\
		\textbf{3} & 4 & \textbf{5,5} &  &  &\\
		5 & 4 &  &  &  & \\
		4 & 3 &  &  &  & \\
		5 & 5 &  &  &  & \\
		5 & 4 &  &  &  &  \\
		1 & 2 & 1,5 &  &  & \\
		2 & 3 & 3,5 &  &  & \\
		2 & 2 & 3,5 &  & & \\
		\textbf{3} & 4 & \textbf{5,5} &  &  & \\
		\hline
		$\sum$ &&&&& 
	\end{tabular}
\end{center}
\end{frame}

\begin{frame}
Spearmans Rangkorrelationskoeffizient
\begin{center}
	\renewcommand{\arraystretch}{1.2}
	\begin{tabular}{cccccc}
		$x$ & $y$ & $rg(x)$ & $rg(y)$ & $rg(x)-rg(y)$ &  $(rg(x)-rg(y))^2$ \\
		\hline
		1 & 2 & 1,5  &  &  & \\
		3 & 4 & 5,5 &  &  &\\
		5 & 4 &  &  &  & \\
		\textbf{4} & 3 &  &  &  & \\
		5 & 5 &  &  &  & \\
		5 & 4 &  &  &  &  \\
		1 & 2 & 1,5 &  &  & \\
		2 & 3 & 3,5 &  &  & \\
		2 & 2 & 3,5 &  & & \\
		3 & 4 & 5,5 &  &  & \\
		\hline
		$\sum$ &&&&& 
	\end{tabular}
\end{center}
\end{frame}

\begin{frame}
Spearmans Rangkorrelationskoeffizient
\begin{center}
	\renewcommand{\arraystretch}{1.2}
	\begin{tabular}{cccccc}
		$x$ & $y$ & $rg(x)$ & $rg(y)$ & $rg(x)-rg(y)$ &  $(rg(x)-rg(y))^2$ \\
		\hline
		1 & 2 & 1,5  &  &  & \\
		3 & 4 & 5,5 &  &  &\\
		5 & 4 &  &  &  & \\
		\textbf{4} & 3 & \textbf{7} &  &  & \\
		5 & 5 &  &  &  & \\
		5 & 4 &  &  &  &  \\
		1 & 2 & 1,5 &  &  & \\
		2 & 3 & 3,5 &  &  & \\
		2 & 2 & 3,5 &  & & \\
		3 & 4 & 5,5 &  &  & \\
		\hline
		$\sum$ &&&&& 
	\end{tabular}
\end{center}
\end{frame}

\begin{frame}
Spearmans Rangkorrelationskoeffizient
\begin{center}
	\renewcommand{\arraystretch}{1.2}
	\begin{tabular}{cccccc}
		$x$ & $y$ & $rg(x)$ & $rg(y)$ & $rg(x)-rg(y)$ &  $(rg(x)-rg(y))^2$ \\
		\hline
		1 & 2 & 1,5  &  &  & \\
		3 & 4 & 5,5 &  &  &\\
		\textbf{5} & 4 &  &  &  & \\
		4 & 3 & 7 &  &  & \\
		\textbf{5} & 5 &  &  &  & \\
		\textbf{5} & 4 &  &  &  &  \\
		1 & 2 & 1,5 &  &  & \\
		2 & 3 & 3,5 &  &  & \\
		2 & 2 & 3,5 &  & & \\
		3 & 4 & 5,5 &  &  & \\
		\hline
		$\sum$ &&&&& 
	\end{tabular}
\end{center}
\end{frame}

\begin{frame}
Spearmans Rangkorrelationskoeffizient
\begin{center}
	\renewcommand{\arraystretch}{1.2}
	\begin{tabular}{cccccc}
		$x$ & $y$ & $rg(x)$ & $rg(y)$ & $rg(x)-rg(y)$ &  $(rg(x)-rg(y))^2$ \\
		\hline
		1 & 2 & 1,5  &  &  & \\
		3 & 4 & 5,5 &  &  &\\
		\textbf{5} & 4 & \textbf{9} &  &  & \\
		4 & 3 & 7 &  &  & \\
		\textbf{5} & 5 & \textbf{9} &  &  & \\
		\textbf{5} & 4 & \textbf{9} &  &  &  \\
		1 & 2 & 1,5 &  &  & \\
		2 & 3 & 3,5 &  &  & \\
		2 & 2 & 3,5 &  & & \\
		3 & 4 & 5,5 &  &  & \\
		\hline
		$\sum$ &&&&& 
	\end{tabular}
\end{center}
\end{frame}

\begin{frame}
Spearmans Rangkorrelationskoeffizient
\begin{center}
	\renewcommand{\arraystretch}{1.2}
	\begin{tabular}{cccccc}
		$x$ & $y$ & $rg(x)$ & $rg(y)$ & $rg(x)-rg(y)$ &  $(rg(x)-rg(y))^2$ \\
		\hline
		1 & 2 & 1,5 & \textbf{2} &  & \\
		3 & 4 & 5,5 & \textbf{7,5} &  & \\
		5 & 4 & 9 & \textbf{7,5} &  & \\
		4 & 3 & 7 & \textbf{4,5} &  & \\
		5 & 5 & 9 & \textbf{10} &  & \\
		5 & 4 & 9 & \textbf{7,5} &  &  \\
		1 & 2 & 1,5 & \textbf{2} &  & \\
		2 & 3 & 3,5 & \textbf{4,5} &  & \\
		2 & 2 & 3,5 & \textbf{2} &  & \\
		3 & 4 & 5,5 & \textbf{7,5} &  & \\
		\hline
		$\sum$ &&&&& 
	\end{tabular}
\end{center}
\end{frame}

\begin{frame}
Spearmans Rangkorrelationskoeffizient
\begin{center}
	\renewcommand{\arraystretch}{1.2}
	\begin{tabular}{cccccc}
		$x$ & $y$ & $rg(x)$ & $rg(y)$ & $rg(x)-rg(y)$ &  $(rg(x)-rg(y))^2$ \\
		\hline
		1 & 2 & 1,5 & 2 & \textbf{-0,5} & \\
		3 & 4 & 5,5 & 7,5 & \textbf{-2} & \\
		5 & 4 & 9 & 7,5 & \textbf{1,5} & \\
		4 & 3 & 7 & 4,5 & \textbf{2,5} & \\
		5 & 5 & 9 & 10 & \textbf{-1} & \\
		5 & 4 & 9 & 7,5 & \textbf{-1,5} &  \\
		1 & 2 & 1,5 & 2 & \textbf{-0,5} & \\
		2 & 3 & 3,5 & 4,5 & \textbf{-1} & \\
		2 & 2 & 3,5 & 2 & \textbf{1,5} & \\
		3 & 4 & 5,5 & 7,5 & \textbf{-2} & \\
		\hline
		$\sum$ &&&&& 
	\end{tabular}
\end{center}
\end{frame}

\begin{frame}
Spearmans Rangkorrelationskoeffizient
\begin{center}
	\renewcommand{\arraystretch}{1.2}
	\begin{tabular}{cccccc}
		$x$ & $y$ & $rg(x)$ & $rg(y)$ & $rg(x)-rg(y)$ &  $(rg(x)-rg(y))^2$ \\
		\hline
		1 & 2 & 1,5 & 2 & -0,5 & \textbf{0,25}\\
		3 & 4 & 5,5 & 7,5 & -2 & \textbf{4}\\
		5 & 4 & 9 & 7,5 & 1,5 & \textbf{2,25}\\
		4 & 3 & 7 & 4,5 & 2,5 & \textbf{6,25}\\
		5 & 5 & 9 & 10 & -1 & \textbf{1}\\
		5 & 4 & 9 & 7,5 & -1,5 & \textbf{2,25} \\
		1 & 2 & 1,5 & 2 & -0,5 & \textbf{0,25}\\
		2 & 3 & 3,5 & 4,5 & -1 & \textbf{1}\\
		2 & 2 & 3,5 & 2 & 1,5 & \textbf{2,25}\\
		3 & 4 & 5,5 & 7,5 & -2 & \textbf{4}\\
		\hline
		$\sum$ &&&&&
	\end{tabular}
\end{center}
\end{frame}

\begin{frame}
Spearmans Rangkorrelationskoeffizient
\begin{center}
	\renewcommand{\arraystretch}{1.2}
	\begin{tabular}{cccccc}
		$x$ & $y$ & $rg(x)$ & $rg(y)$ & $rg(x)-rg(y)$ &  $(rg(x)-rg(y))^2$ \\
		\hline
		1 & 2 & 1,5 & 2 & -0,5 & 0,25\\
		3 & 4 & 5,5 & 7,5 & -2 & 4\\
		5 & 4 & 9 & 7,5 & 1,5 & 2,25\\
		4 & 3 & 7 & 4,5 & 2,5 & 6,25\\
		5 & 5 & 9 & 10 & -1 & 1\\
		5 & 4 & 9 & 7,5 & -1,5 & 2,25 \\
		1 & 2 & 1,5 & 2 & -0,5 & 0,25\\
		2 & 3 & 3,5 & 4,5 & -1 & 1\\
		2 & 2 & 3,5 & 2 & 1,5 & 2,25\\
		3 & 4 & 5,5 & 7,5 & -2 & 4\\
		\hline
		$\sum$ &&&&& \textbf{23,5}
	\end{tabular}
\end{center}
\end{frame}

\begin{frame}
Spearmans Rangkorrelationskoeffizient
\begin{align*}
	r&=1-\frac{6 \cdot \sum (rg(x)-rg(y))^2}{(n-1)\cdot n \cdot (n+1)}\\
	&= 1-\frac{6 \cdot 23,5}{9\cdot 10 \cdot 11}\\
	&= 0,8576
\end{align*}
\end{frame}

\begin{frame}
Spearmans Rangkorrelationskoeffizient
\begin{align*}
r&=1-\frac{6 \cdot \sum (rg(x)-rg(y))^2}{(n-1)\cdot n \cdot (n+1)}\\
&= 1-\frac{6 \cdot 23,5}{9\cdot 10 \cdot 11}\\
&= 0,8576
\end{align*}
Interpretation
\begin{itemize}[<+->]
	\item $r \in [-1;+1]$
	\item $r\approx -1$: negativer Zusammenhang
	\item $r\approx 0$: kein Zusammenhang
	\item $r\approx +1$: positiver Zusammenhang
\end{itemize}
\end{frame}

\end{document}