\documentclass[t,11pt]{beamer}
\usepackage{tikz}
\usepackage{pgfplots}
\usetikzlibrary{calc}
\usepackage[utf8]{inputenc}
\usepackage[ngerman]{babel}
\usepackage{amsmath,amsfonts,amssymb}
\usepackage{framed}
\usecolortheme{orchid}
\usepackage{etoolbox}
\useinnertheme[shadow=true]{rounded}

%%% PROGRESSBAR
\definecolor{pbblue}{HTML}{D8D8D8}% filling color for the progress bar
\definecolor{pbgray}{HTML}{F2F2F2}% background color for the progress bar
\useoutertheme{infolines}
\setbeamerfont{footline}{size=\normalsize}
\setbeamersize{text margin left=30pt,text margin right=30pt}
\makeatletter
\setbeamertemplate{footline}
{
	\leavevmode%
	\hbox{%
		\begin{beamercolorbox}[wd=.333333\paperwidth,ht=2.5ex,dp=1ex,center]{title in head/foot}%
			\usebeamerfont{title in head/foot}\insertshorttitle
		\end{beamercolorbox}%
		\begin{beamercolorbox}[wd=.333333\paperwidth,ht=2.5ex,dp=1ex,center]{date in head/foot}%
			%\usebeamerfont{date in head/foot}\insertshortdate{}\hspace*{2em}
			%\insertframenumber\hspace*{2ex} 
		\end{beamercolorbox}
		\begin{beamercolorbox}[wd=.333333\paperwidth,ht=3ex,dp=1ex,center]{author in head/foot}%
			\usebeamerfont{author in head/foot}\insertshortauthor~~%\beamer@ifempty{\insertshortinstitute}{}{(\insertshortinstitute)}
		\end{beamercolorbox}%
	}%
	\vskip0pt%
}
\makeatother
\makeatletter
\def\progressbar@progressbar{} % the progress bar
\newcount\progressbar@tmpcounta% auxiliary counter
\newcount\progressbar@tmpcountb% auxiliary counter
\newdimen\progressbar@pbht %progressbar height
\newdimen\progressbar@pbwd %progressbar width
\newdimen\progressbar@tmpdim % auxiliary dimension
\progressbar@pbwd=\linewidth
\progressbar@pbht=1.5ex
\def\progressbar@progressbar{%
    \progressbar@tmpcounta=\insertframenumber
    \progressbar@tmpcountb=\inserttotalframenumber
    \progressbar@tmpdim=\progressbar@pbwd
    \multiply\progressbar@tmpdim by \progressbar@tmpcounta
    \divide\progressbar@tmpdim by \progressbar@tmpcountb
  \begin{tikzpicture}[rounded corners=3pt,very thin]
    \shade[top color=pbgray!20,bottom color=pbgray!20,middle color=pbgray!50]
      (0pt, 0pt) rectangle ++ (\progressbar@pbwd, \progressbar@pbht);
      \shade[draw=pbblue,top color=pbblue!50,bottom color=pbblue!50,middle color=pbblue] %
        (0pt, 0pt) rectangle ++ (\progressbar@tmpdim, \progressbar@pbht);
    \draw[color=normal text.fg!50]  
      (0pt, 0pt) rectangle (\progressbar@pbwd, \progressbar@pbht) 
        node[pos=0.5,color=normal text.fg] {\textnormal{%
             \pgfmathparse{\insertframenumber*100/\inserttotalframenumber}%
             \pgfmathprintnumber[fixed,precision=0]{\pgfmathresult}\,\%%
        }%
    };
  \end{tikzpicture}%
}
\addtobeamertemplate{headline}{}
{%
  \begin{beamercolorbox}[wd=\paperwidth,ht=4ex,center,dp=1ex]{white}%
    \progressbar@progressbar%
  \end{beamercolorbox}%
}
\makeatother


%%% BLOCKS
% block = Aufgabe
\setbeamercolor{block title}{fg=black,bg=blue!30!white} 
\setbeamercolor{block body}{fg=black, bg=blue!3!white}

% alertblock = Definition
\setbeamercolor{block title alerted}{fg=black,bg=red!50!white} 
\setbeamercolor{block body alerted}{fg=black, bg=red!3!white}

% exampleblock = Wiederholung
\setbeamercolor{block title example}{fg=black,bg=green!30!white} 
\setbeamercolor{block body example}{fg=black, bg=green!3!white}



\begin{document}
	\author{www.oilbat.de}
	%\title{Stochastik}
	\subtitle{}
	\logo{}
	\institute{}
	\date{}
	\subject{}
	\setbeamercovered{transparent}
	\setbeamertemplate{navigation symbols}{}

\addtocounter{framenumber}{-1}
\setbeamercovered{invisible}

\begin{frame}
	\begin{block}{Aufgabe}
		Berechne den Korrelationskoeffizienten nach Spearman für die folgende, zweidimensionale Stichprobe.
	\end{block}
	\vspace{0.5cm}
	\begin{tabular}{|cc|cccccccc|}
		\hline
		\multicolumn{2}{|c|}{Person} & 1 & 2 & 3 & 4 & 5 & 6 & 7 & 8 \\
		\hline 
		Größe [cm] & x & 174 & 183 & 162  & 170 & 182 & 176 & 173 & 198  \\ 
		Gewicht [kg] & y &  73 & 93 & 74 & 58 & 90 & 88 & 72 & 91\\ 
		\hline 
	\end{tabular}

\end{frame}

\begin{frame}
\begin{block}{Aufgabe}
	Berechne den Korrelationskoeffizienten nach Spearman für die folgende, zweidimensionale Stichprobe.
\end{block}
\vspace{0.5cm}
\begin{tabular}{|cc|cccccccc|}
	\hline
	\multicolumn{2}{|c|}{Person} & 1 & 2 & 3 & 4 & 5 & 6 & 7 & 8 \\
	\hline 
	Größe [cm] & x & 174 & 183 & 162  & 170 & 182 & 176 & 173 & 198  \\ 
	\multicolumn{2}{|c|}{\textbf{Rang}} & 4 & 7 & 1 & 2 & 6 & 5 & 3 & 8 \\
	\hline
	Gewicht [kg] & y &  73 & 93 & 74 & 58 & 90 & 88 & 72 & 91\\ 
	\multicolumn{2}{|c|}{\textbf{Rang}} & 3 & 8 & 4 & 1 & 6 & 5 & 2 & 7 \\
	\hline 
\end{tabular}

\end{frame}

\begin{frame}
\begin{alertblock}{Definition: Spearmans Korrelationskoeffizient}
	Für gegebene Daten $(x_i,y_i)$, $i=1,\dots,n$, ist der Korrelationskoeffizient nach Spearman $r_{SP}$ definiert durch
	\begin{align*}
	r_{SP}=\frac{\sum_{i=1}^{n}\left[rg(x_i)-\bar{rg}_X\right]\left[rg(y_i)-\bar{rg}_Y\right]}{\sqrt{\sum_{i=1}^{n}\left[rg(x_i)-\bar{rg}_X\right]^2}\sqrt{\sum_{i=1}^{n}\left[rg(y_i)-\bar{rg}_Y\right]^2}},
	\end{align*}
	wobei
	\begin{align*}
	\bar{rg}_X=\frac{1}{n}\sum_{i=1}^{n}rg(x_i)=\frac{1}{n}\sum_{i=1}^{n}i=\frac{1}{n}\cdot\frac{n(n+1)}{2}=\frac{n+1}{2}=\bar{rg}_Y.
	\end{align*}
\end{alertblock}
\end{frame}


\begin{frame}
	\begin{alertblock}{Definition: Spearmans Korrelationskoeffizient}
		Für gegebene Daten $(x_i,y_i)$, $i=1,\dots,n$, ist der Korrelationskoeffizient nach Spearman $r_{SP}$ definiert durch
		\begin{align*}
			r_{SP}=\frac{\sum_{i=1}^{n}\left[rg(x_i)-\bar{rg}_X\right]\left[rg(y_i)-\bar{rg}_Y\right]}{\sqrt{\sum_{i=1}^{n}\left[rg(x_i)-\bar{rg}_X\right]^2}\sqrt{\sum_{i=1}^{n}\left[rg(y_i)-\bar{rg}_Y\right]^2}},
		\end{align*}
		wobei
		\begin{align*}
		\bar{rg}_X=\frac{1}{n}\sum_{i=1}^{n}rg(x_i)=\frac{1}{n}\sum_{i=1}^{n}i=\frac{1}{n}\cdot\frac{n(n+1)}{2}=\frac{n+1}{2}=\bar{rg}_Y.
		\end{align*}
	\end{alertblock}
\vspace{0.5cm}
	In unserem Beispiel: $\bar{rg}_X=\bar{rg}_Y=\frac{n+1}{2}=\frac{9}{2}=4,5$
\end{frame}

\begin{frame}
	\begin{align*}
		r_{SP} &= \frac{\sum_{i=1}^{n}\left[rg(x_i)-\bar{rg}_X\right]\left[rg(y_i)-\bar{rg}_Y\right]}{\sqrt{\sum_{i=1}^{n}\left[rg(x_i)-\bar{rg}_X\right]^2}\sqrt{\sum_{i=1}^{n}\left[rg(y_i)-\bar{rg}_Y\right]^2}} \\
		&= \frac{\sum_{i=1}^{8}\left[rg(x_i)-4,5\right]\left[rg(y_i)-4,5\right]}{\sqrt{\sum_{i=1}^{8}\left[rg(x_i)-4,5\right]^2}\sqrt{\sum_{i=8}^{n}\left[rg(y_i)-4,5\right]^2}}\\
		&= \frac{(4-4,5)\cdot (3-4,5)+\dots+(8-36)\cdot (7-4,5)}{\sqrt{(4-4,5)^2+\dots+(8-4,5)^2}\sqrt{(3-4,5)^2+\dots+(7-4,5)^2}} \\
		&= \frac{5}{6}
	\end{align*}
\end{frame}

\begin{frame}
\begin{tabular}{|cc|cccccccc|}
	\hline
	\multicolumn{2}{|c|}{Person} & 1 & 2 & 3 & 4 & 5 & 6 & 7 & 8 \\
	\hline 
	Größe [cm] & x & 174 & 183 & 162  & 170 & 182 & 176 & 173 & 198  \\ 
	Gewicht [kg] & y &  73 & 93 & 74 & 58 & 90 & 88 & 72 & 91\\ 
	\hline 
\end{tabular}
$$\text{Korrelationskoeffizient nach Spearman: }r_{SP}=\frac{5}{6}$$
\end{frame}

\begin{frame}
\begin{tabular}{|cc|cccccccc|}
	\hline
	\multicolumn{2}{|c|}{Person} & 1 & 2 & 3 & 4 & 5 & 6 & 7 & 8 \\
	\hline 
	Größe [cm] & x & 174 & 183 & 162  & 170 & 182 & 176 & 173 & 198  \\ 
	Gewicht [kg] & y &  73 & 93 & 74 & 58 & 90 & 88 & 72 & 91\\ 
	\hline 
\end{tabular}
$$\text{Korrelationskoeffizient nach Spearman: }r_{SP}=\frac{5}{6}$$
\begin{alertblock}{Interpretation}
	\begin{itemize}
		\item $r_{SP} \in [-1,1]$
		\item $r_{SP}>0$ -- positiver Zusammenhang
		\item $r_{SP}=0$ -- kein Zusammenhang
		\item $r_{SP}<0$ -- negativer Zusammenhang
	\end{itemize}
\end{alertblock}
\end{frame}

\begin{frame}
	\begin{exampleblock}{Einfachere Berechnung von $r_{SP}$}
		Für Daten $(x_i,y_i)$, $i=1,\dots,n$ mit $x_i\neq x_j$ und $y_i\neq y_j$ für alle $i,j$ gilt
		\begin{align*}
			r_{SP}&=1-\frac{6 \cdot \sum_{i=1}^{n} (rg(x_i)-rg(y_i))^2}{(n-1)\cdot n \cdot (n+1)}.
		\end{align*}
	\end{exampleblock}
\vspace{0.5cm}
	\textbf{Wichtig:} Gilt nur falls die Daten paarweise verschieden sind (d.h. wenn es keine \textit{Bindungen} gibt)!
\end{frame}

\begin{frame}
\begin{exampleblock}{Einfachere Berechnung von $r_{SP}$}
\end{exampleblock}
	\begin{center}
		\renewcommand{\arraystretch}{1.2}
		\begin{tabular}{cccccc}
			$x$ & $y$ & $rg(x)$ & $rg(y)$ & $rg(x)-rg(y)$ &  $(rg(x)-rg(y))^2$ \\
			\hline
			174 & 73 & 4 & 3 &  & \\
			183 & 93 & 7 & 8 &  &\\
			162 & 74 & 1 & 4 &  & \\
			170 & 58 & 2 & 1 &  & \\
			182 & 90 & 6 & 6 &  & \\
			176 & 88 & 5 & 5 &  &  \\
			173 & 72 & 3 & 2 &  & \\
			198 & 91 & 8 & 7 &  & \\
			\hline
			$\sum$ &&&&& 
		\end{tabular}
	\end{center}
\end{frame}

\begin{frame}
\begin{exampleblock}{Einfachere Berechnung von $r_{SP}$}
\end{exampleblock}
\begin{center}
	\renewcommand{\arraystretch}{1.2}
	\begin{tabular}{cccccc}
		$x$ & $y$ & $rg(x)$ & $rg(y)$ & $rg(x)-rg(y)$ &  $(rg(x)-rg(y))^2$ \\
		\hline
		174 & 73 & 4 & 3 & 1 &  \\
		183 & 93 & 7 & 8 & -1 & \\
		162 & 74 & 1 & 4 & -3 &  \\
		170 & 58 & 2 & 1 & 1 & \\
		182 & 90 & 6 & 6 & 0 & \\
		176 & 88 & 5 & 5 & 0 &  \\
		173 & 72 & 3 & 2 & 1 & \\
		198 & 91 & 8 & 7 & 1 & \\
		\hline
		$\sum$ &&&&& 
	\end{tabular}
\end{center}
\end{frame}

\begin{frame}
\begin{exampleblock}{Einfachere Berechnung von $r_{SP}$}
\end{exampleblock}
\begin{center}
	\renewcommand{\arraystretch}{1.2}
	\begin{tabular}{cccccc}
		$x$ & $y$ & $rg(x)$ & $rg(y)$ & $rg(x)-rg(y)$ &  $(rg(x)-rg(y))^2$ \\
		\hline
		174 & 73 & 4 & 3 & 1 & 1 \\
		183 & 93 & 7 & 8 & -1 & 1\\
		162 & 74 & 1 & 4 & -3 & 9 \\
		170 & 58 & 2 & 1 & 1 & 1\\
		182 & 90 & 6 & 6 & 0 & 0\\
		176 & 88 & 5 & 5 & 0 & 0 \\
		173 & 72 & 3 & 2 & 1 & 1\\
		198 & 91 & 8 & 7 & 1 & 1\\
		\hline
		$\sum$ &&&&& 
	\end{tabular}
\end{center}
\end{frame}

\begin{frame}
\begin{exampleblock}{Einfachere Berechnung von $r_{SP}$}
\end{exampleblock}
\begin{center}
	\renewcommand{\arraystretch}{1.2}
	\begin{tabular}{cccccc}
		$x$ & $y$ & $rg(x)$ & $rg(y)$ & $rg(x)-rg(y)$ &  $(rg(x)-rg(y))^2$ \\
		\hline
		174 & 73 & 4 & 3 & 1 & 1 \\
		183 & 93 & 7 & 8 & -1 & 1\\
		162 & 74 & 1 & 4 & -3 & 9 \\
		170 & 58 & 2 & 1 & 1 & 1\\
		182 & 90 & 6 & 6 & 0 & 0\\
		176 & 88 & 5 & 5 & 0 & 0 \\
		173 & 72 & 3 & 2 & 1 & 1\\
		198 & 91 & 8 & 7 & 1 & 1\\
		\hline
		$\sum$ &&&&& 14
	\end{tabular}
\end{center}
\end{frame}

\begin{frame}
\begin{exampleblock}{Einfachere Berechnung von $r_{SP}$}
	Für Daten $(x_i,y_i)$, $i=1,\dots,n$ mit $x_i\neq x_j$ und $y_i\neq y_j$ für alle $i,j$ gilt
	\begin{align*}
	r_{SP}&=1-\frac{6 \cdot \sum_{i=1}^{n} (rg(x_i)-rg(y_i))^2}{(n-1)\cdot n \cdot (n+1)}.
	\end{align*}
\end{exampleblock}
\begin{align*}
r_{SP}&=1-\frac{6 \cdot \sum_{i=1}^{8} (rg(x_i)-rg(y_i))^2}{(8-1)\cdot 8 \cdot (8+1)}\\
&= 1-\frac{6 \cdot 14}{7\cdot 8 \cdot 9}\\
&= \frac{5}{6}
\end{align*}
\end{frame}



\end{document}