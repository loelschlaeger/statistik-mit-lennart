\documentclass[t,11pt]{beamer}
\usepackage{tikz}
\usepackage{pgfplots}
\usetikzlibrary{calc}
\usepackage[utf8]{inputenc}
\usepackage[ngerman]{babel}
\usepackage{amsmath,amsfonts,amssymb}
\usepackage{framed}
\usecolortheme{orchid}
\usepackage{etoolbox}
\useinnertheme[shadow=true]{rounded}

%%% PROGRESSBAR
\definecolor{pbblue}{HTML}{D8D8D8}% filling color for the progress bar
\definecolor{pbgray}{HTML}{F2F2F2}% background color for the progress bar
\useoutertheme{infolines}
\setbeamerfont{footline}{size=\normalsize}
\setbeamersize{text margin left=30pt,text margin right=30pt}
\makeatletter
\setbeamertemplate{footline}
{
	\leavevmode%
	\hbox{%
		\begin{beamercolorbox}[wd=.333333\paperwidth,ht=2.5ex,dp=1ex,center]{title in head/foot}%
			\usebeamerfont{title in head/foot}\insertshorttitle
		\end{beamercolorbox}%
		\begin{beamercolorbox}[wd=.333333\paperwidth,ht=2.5ex,dp=1ex,center]{date in head/foot}%
			%\usebeamerfont{date in head/foot}\insertshortdate{}\hspace*{2em}
			%\insertframenumber\hspace*{2ex} 
		\end{beamercolorbox}
		\begin{beamercolorbox}[wd=.333333\paperwidth,ht=3ex,dp=1ex,center]{author in head/foot}%
			\usebeamerfont{author in head/foot}\insertshortauthor~~%\beamer@ifempty{\insertshortinstitute}{}{(\insertshortinstitute)}
		\end{beamercolorbox}%
	}%
	\vskip0pt%
}
\makeatother
\makeatletter
\def\progressbar@progressbar{} % the progress bar
\newcount\progressbar@tmpcounta% auxiliary counter
\newcount\progressbar@tmpcountb% auxiliary counter
\newdimen\progressbar@pbht %progressbar height
\newdimen\progressbar@pbwd %progressbar width
\newdimen\progressbar@tmpdim % auxiliary dimension
\progressbar@pbwd=\linewidth
\progressbar@pbht=1.5ex
\def\progressbar@progressbar{%
    \progressbar@tmpcounta=\insertframenumber
    \progressbar@tmpcountb=\inserttotalframenumber
    \progressbar@tmpdim=\progressbar@pbwd
    \multiply\progressbar@tmpdim by \progressbar@tmpcounta
    \divide\progressbar@tmpdim by \progressbar@tmpcountb
  \begin{tikzpicture}[rounded corners=3pt,very thin]
    \shade[top color=pbgray!20,bottom color=pbgray!20,middle color=pbgray!50]
      (0pt, 0pt) rectangle ++ (\progressbar@pbwd, \progressbar@pbht);
      \shade[draw=pbblue,top color=pbblue!50,bottom color=pbblue!50,middle color=pbblue] %
        (0pt, 0pt) rectangle ++ (\progressbar@tmpdim, \progressbar@pbht);
    \draw[color=normal text.fg!50]  
      (0pt, 0pt) rectangle (\progressbar@pbwd, \progressbar@pbht) 
        node[pos=0.5,color=normal text.fg] {\textnormal{%
             \pgfmathparse{\insertframenumber*100/\inserttotalframenumber}%
             \pgfmathprintnumber[fixed,precision=0]{\pgfmathresult}\,\%%
        }%
    };
  \end{tikzpicture}%
}
\addtobeamertemplate{headline}{}
{%
  \begin{beamercolorbox}[wd=\paperwidth,ht=4ex,center,dp=1ex]{white}%
    \progressbar@progressbar%
  \end{beamercolorbox}%
}
\makeatother


%%% BLOCKS
% block = Aufgabe
\setbeamercolor{block title}{fg=black,bg=blue!30!white} 
\setbeamercolor{block body}{fg=black, bg=blue!3!white}

% alertblock = Definition
\setbeamercolor{block title alerted}{fg=black,bg=red!50!white} 
\setbeamercolor{block body alerted}{fg=black, bg=red!3!white}

% exampleblock = Wiederholung
\setbeamercolor{block title example}{fg=black,bg=green!30!white} 
\setbeamercolor{block body example}{fg=black, bg=green!3!white}



\begin{document}
	%\author{www.oilbat.de}
	%\title{Stochastik}
	\subtitle{}
	\logo{}
	\institute{}
	\date{}
	\subject{}
	\setbeamercovered{transparent}
	\setbeamertemplate{navigation symbols}{}

\addtocounter{framenumber}{-1}
\setbeamercovered{invisible}

\begin{frame}
\begin{block}{Aufgabe}
	Wir wollen annehmen, dass sich die Akkuladedauer für Handys gut durch eine Normalverteilung mit einer Standardabweichung von 20 beschreiben lässt. Aus $\mathcal{N}(\mu, \sigma^2 = 20^2)$ wurde eine Stichprobe mit den Realisationen $(x_1,\dots,x_{100})$ gezogen, wobei $\bar{x} = 121,06$ Minuten. Ermitteln Sie die Grenzen eines 95\%-KIs für $\mu$.
\end{block}
\end{frame}

\begin{frame}
\begin{block}{Aufgabe}
	Wir wollen annehmen, dass sich die Akkuladedauer für Handys gut durch eine Normalverteilung mit einer Standardabweichung von 20 beschreiben lässt. Aus $\mathcal{N}(\mu, \sigma^2 = 20^2)$ wurde eine Stichprobe mit den Realisationen $(x_1,\dots,x_{100})$ gezogen, wobei $\bar{x} = 121,06$ Minuten. Ermitteln Sie die Grenzen eines 95\%-KIs für $\mu$.
\end{block}
\begin{alertblock}{Definition: $1-\alpha$-KI für $\mu$ falls $\sigma$ bekannt ist}
	\begin{align*}
	\left[ \bar{X} - z_{1-\alpha/2} \frac{\sigma}{\sqrt{n}};~\bar{X} + z_{1-\alpha/2} \frac{\sigma}{\sqrt{n}}  \right]
	\end{align*}
	\vspace{-0.5cm}
	\begin{itemize}
		\item $\alpha$ ist das Konfidenzniveau
		\item $z_{1-\alpha/2}$ ist das $1-\alpha/2$-Quantil der Standardnormalverteilung
		\item $\sigma$ ist die \textbf{bekannte} Standardabweichung
		\item $n$ ist der Stichprobenumfang
	\end{itemize}
\end{alertblock}
\end{frame}

\begin{frame}
	\begin{align*}
		\frac{\bar{X}-\mu}{\sigma/\sqrt{n}}\sim \mathcal{N}(0,1)
	\end{align*}
\end{frame}

\begin{frame}
\begin{align*}
\frac{\bar{X}-\mu}{\sigma/\sqrt{n}}\sim \mathcal{N}(0,1)
\end{align*}
\begin{align*}
1-\alpha &= P(-z_{1-\alpha/2}\leq \frac{\bar{X}-\mu}{\sigma/\sqrt{n}} \leq z_{1-\alpha/2}) \\
&= P(-z_{1-\alpha/2} \frac{\sigma}{\sqrt{n}}\leq \bar{X}-\mu \leq z_{1-\alpha/2} \frac{\sigma}{\sqrt{n}}) \\
&= P(\bar{X} -z_{1-\alpha/2} \frac{\sigma}{\sqrt{n}}\leq \mu \leq \bar{X} + z_{1-\alpha/2} \frac{\sigma}{\sqrt{n}})
\end{align*}
\end{frame}

\begin{frame}
\begin{align*}
\frac{\bar{X}-\mu}{\sigma/\sqrt{n}}\sim \mathcal{N}(0,1)
\end{align*}
\begin{align*}
1-\alpha &= P(-z_{1-\alpha/2}\leq \frac{\bar{X}-\mu}{\sigma/\sqrt{n}} \leq z_{1-\alpha/2}) \\
&= P(-z_{1-\alpha/2} \frac{\sigma}{\sqrt{n}}\leq \bar{X}-\mu \leq z_{1-\alpha/2} \frac{\sigma}{\sqrt{n}}) \\
&= P(\bar{X} -z_{1-\alpha/2} \frac{\sigma}{\sqrt{n}}\leq \mu \leq \bar{X} + z_{1-\alpha/2} \frac{\sigma}{\sqrt{n}})
\end{align*}
\begin{align*}
\left[ \bar{X} - z_{1-\alpha/2} \frac{\sigma}{\sqrt{n}};~\bar{X} + z_{1-\alpha/2} \frac{\sigma}{\sqrt{n}}  \right]
\end{align*}
\end{frame}

\begin{frame}
	\begin{block}{Aufgabe}
		Wir wollen annehmen, dass sich die Akkuladedauer für Handys gut durch eine Normalverteilung mit einer Standardabweichung von 20 beschreiben lässt. Aus $\mathcal{N}(\mu, \sigma^2 = 20^2)$ wurde eine Stichprobe mit den Realisationen $(x_1,\dots,x_{100})$ gezogen, wobei $\bar{x} = 121,06$ Minuten. Ermitteln Sie die Grenzen eines 95\%-KIs für $\mu$.
	\end{block}
	\begin{itemize}
		\item Arithmetisches Stichprobenmittel: $\bar{x} = 121,06$
		\item Konfidenzniveau: $\alpha=5\%$
		\item Damit ist das Quantil $z_{1-\alpha/2}=z_{0,975}=1,96$
		\item Die \textbf{bekannte} Standardabweichung ist $\sigma=20$
		\item Der Stichprobenumfang ist $n=100$
	\end{itemize}
\end{frame}

\begin{frame}
\begin{itemize}
	\item Arithmetisches Stichprobenmittel: $\bar{x} = 121,06$
	\item Konfidenzniveau: $\alpha=5\%$
	\item Damit ist das Quantil $z_{1-\alpha/2}=z_{0,975}=1,96$
	\item Die \textbf{bekannte} Standardabweichung ist $\sigma=20$
	\item Der Stichprobenumfang ist $n=100$
\end{itemize}
\begin{alertblock}{Definition: $1-\alpha$-KI für $\mu$ falls $\sigma$ bekannt ist}
	\begin{align*}
	\left[ \bar{X} - z_{1-\alpha/2} \frac{\sigma}{\sqrt{n}};~\bar{X} + z_{1-\alpha/2} \frac{\sigma}{\sqrt{n}}  \right]
	\end{align*}
\end{alertblock}
\end{frame}

\begin{frame}
\begin{itemize}
	\item Arithmetisches Stichprobenmittel: $\bar{x} = 121,06$
	\item Konfidenzniveau: $\alpha=5\%$
	\item Damit ist das Quantil $z_{1-\alpha/2}=z_{0,975}=1,96$
	\item Die \textbf{bekannte} Standardabweichung ist $\sigma=20$
	\item Der Stichprobenumfang ist $n=100$
\end{itemize}
\begin{alertblock}{Definition: $1-\alpha$-KI für $\mu$ falls $\sigma$ bekannt ist}
	\begin{align*}
	\left[ \bar{X} - z_{1-\alpha/2} \frac{\sigma}{\sqrt{n}};~\bar{X} + z_{1-\alpha/2} \frac{\sigma}{\sqrt{n}}  \right]
	\end{align*}
\end{alertblock}
\begin{align*}
\left[ 121,06 - 1,96\cdot \frac{20}{\sqrt{100}};~ 121,06 + 1,96\cdot \frac{20}{\sqrt{100}} \right] = \left[ 117,14;~124,98 \right]
\end{align*}
\end{frame}

\end{document}