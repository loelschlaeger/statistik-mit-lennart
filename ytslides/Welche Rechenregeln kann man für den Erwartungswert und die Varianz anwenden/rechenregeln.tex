\documentclass[t,11pt]{beamer}
\usepackage{tikz}
\usepackage{pgfplots}
\usetikzlibrary{calc}
\usepackage[utf8]{inputenc}
\usepackage[ngerman]{babel}
\usepackage{amsmath,amsfonts,amssymb}
\usepackage{framed}
\usecolortheme{orchid}
\usepackage{etoolbox}
\useinnertheme[shadow=true]{rounded}

\usepackage{pgfplots}

\pgfmathdeclarefunction{gauss}{2}{
	\pgfmathparse{1/(#2*sqrt(2*pi))*exp(-((x-#1)^2)/(2*#2^2))}
}

%%% PROGRESSBAR
\definecolor{pbblue}{HTML}{D8D8D8}% filling color for the progress bar
\definecolor{pbgray}{HTML}{F2F2F2}% background color for the progress bar
\useoutertheme{infolines}
\setbeamerfont{footline}{size=\normalsize}
\setbeamersize{text margin left=30pt,text margin right=30pt}
\makeatletter
\setbeamertemplate{footline}
{
	\leavevmode%
	\hbox{%
		\begin{beamercolorbox}[wd=.333333\paperwidth,ht=2.5ex,dp=1ex,center]{title in head/foot}%
			\usebeamerfont{title in head/foot}\insertshorttitle
		\end{beamercolorbox}%
		\begin{beamercolorbox}[wd=.333333\paperwidth,ht=2.5ex,dp=1ex,center]{date in head/foot}%
			%\usebeamerfont{date in head/foot}\insertshortdate{}\hspace*{2em}
			%\insertframenumber\hspace*{2ex} 
		\end{beamercolorbox}
		\begin{beamercolorbox}[wd=.333333\paperwidth,ht=3ex,dp=1ex,center]{author in head/foot}%
			\usebeamerfont{author in head/foot}\insertshortauthor~~%\beamer@ifempty{\insertshortinstitute}{}{(\insertshortinstitute)}
		\end{beamercolorbox}%
	}%
	\vskip0pt%
}
\makeatother
\makeatletter
\def\progressbar@progressbar{} % the progress bar
\newcount\progressbar@tmpcounta% auxiliary counter
\newcount\progressbar@tmpcountb% auxiliary counter
\newdimen\progressbar@pbht %progressbar height
\newdimen\progressbar@pbwd %progressbar width
\newdimen\progressbar@tmpdim % auxiliary dimension
\progressbar@pbwd=\linewidth
\progressbar@pbht=1.5ex
\def\progressbar@progressbar{%
    \progressbar@tmpcounta=\insertframenumber
    \progressbar@tmpcountb=\inserttotalframenumber
    \progressbar@tmpdim=\progressbar@pbwd
    \multiply\progressbar@tmpdim by \progressbar@tmpcounta
    \divide\progressbar@tmpdim by \progressbar@tmpcountb
  \begin{tikzpicture}[rounded corners=3pt,very thin]
    \shade[top color=pbgray!20,bottom color=pbgray!20,middle color=pbgray!50]
      (0pt, 0pt) rectangle ++ (\progressbar@pbwd, \progressbar@pbht);
      \shade[draw=pbblue,top color=pbblue!50,bottom color=pbblue!50,middle color=pbblue] %
        (0pt, 0pt) rectangle ++ (\progressbar@tmpdim, \progressbar@pbht);
    \draw[color=normal text.fg!50]  
      (0pt, 0pt) rectangle (\progressbar@pbwd, \progressbar@pbht) 
        node[pos=0.5,color=normal text.fg] {\textnormal{%
             \pgfmathparse{\insertframenumber*100/\inserttotalframenumber}%
             \pgfmathprintnumber[fixed,precision=0]{\pgfmathresult}\,\%%
        }%
    };
  \end{tikzpicture}%
}
\addtobeamertemplate{headline}{}
{%
  \begin{beamercolorbox}[wd=\paperwidth,ht=4ex,center,dp=1ex]{white}%
    \progressbar@progressbar%
  \end{beamercolorbox}%
}
\makeatother


%%% BLOCKS
% block = Aufgabe
\setbeamercolor{block title}{fg=black,bg=blue!30!white} 
\setbeamercolor{block body}{fg=black, bg=blue!3!white}

% alertblock = Definition
\setbeamercolor{block title alerted}{fg=black,bg=red!50!white} 
\setbeamercolor{block body alerted}{fg=black, bg=red!3!white}

% exampleblock = Wiederholung
\setbeamercolor{block title example}{fg=black,bg=green!30!white} 
\setbeamercolor{block body example}{fg=black, bg=green!3!white}



\begin{document}
	%\author{www.oilbat.de}
	%\title{Stochastik}
	\subtitle{}
	\logo{}
	\institute{}
	\date{}
	\subject{}
	\setbeamercovered{transparent}
	\setbeamertemplate{navigation symbols}{}

\addtocounter{framenumber}{-1}
\setbeamercovered{invisible}

\begin{frame}
	\begin{block}{Aufgabe}
		\begin{itemize}[<+->]
			\item Sei $X\sim \mathcal{U}_{[0,1]}$. Was ist dann $\mathbb{E}(2X)$? 
			\item Sei $Y\sim Poi(4)$. Was ist dann $\mathbb{V}(Y-0,5)$? 
			\item Sei $Z\sim \mathcal{N}(\mu,\sigma^2)$. Was ist dann die Verteilung von $\frac{Z-\mu}{\sigma} $?
		\end{itemize}
	\end{block}
\end{frame}

\begin{frame}
\begin{block}{Aufgabe}
	\begin{itemize}%[<+->]
		\item Sei $X\sim \mathcal{U}_{[0,1]}$. Was ist dann $\mathbb{E}(2X)$? 
		\item Sei $Y\sim Poi(4)$. Was ist dann $\mathbb{V}(Y-0,5)$? 
		\item Sei $Z\sim \mathcal{N}(\mu,\sigma^2)$. Was ist dann die Verteilung von $\frac{Z-\mu}{\sigma} $?
	\end{itemize}
\end{block}
\begin{alertblock}{Rechenregeln für Erwartungswert und Varianz}
	Sei $X$ eine Zufallsvariable und $a,b\in\mathbb{R}$. Dann gilt:
	\begin{align*}
	\mathbb{E}(aX+b) &= a\mathbb{E}(X) + b \\
	\mathbb{V}(aX+b) &= a^2\mathbb{V}(X) \\
	\end{align*}
\end{alertblock}
\end{frame}

\begin{frame}
\textbf{Rechenregeln:}
\begin{align*}
\mathbb{E}(aX+b) = a\mathbb{E}(X) + b, \ \
\mathbb{V}(aX+b) = a^2\mathbb{V}(X) 
\end{align*}
\end{frame}

\begin{frame}
\textbf{Rechenregeln:}
\begin{align*}
\mathbb{E}(aX+b) = a\mathbb{E}(X) + b, \ \
\mathbb{V}(aX+b) = a^2\mathbb{V}(X) 
\end{align*}
\textbf{Beispiel:}
\begin{align*}
\mathbb{E}({\color{blue}{}X})=0 \ \Rightarrow \
\mathbb{E}({\color{red}{}X+6})=? 
\end{align*}
\end{frame}

\begin{frame}
\textbf{Rechenregeln:}
\begin{align*}
\mathbb{E}(aX+b) = a\mathbb{E}(X) + b, \ \
\mathbb{V}(aX+b) = a^2\mathbb{V}(X) 
\end{align*}
\textbf{Beispiel:}
\begin{align*}
\mathbb{E}({\color{blue}{}X})=0 \ \Rightarrow \
\mathbb{E}({\color{red}{}X+6})=? 
\end{align*}
\begin{tikzpicture}[remember picture,overlay,scale=0.8]
\tikzset{shift={(current page.center)},xshift=-4cm,yshift=-5cm}
\begin{axis}[every axis plot post/.append style={
	mark=none,domain=-5:10,samples=40,smooth},
% All plots: from -2:2, 50 samples, smooth, no marks
axis x line*=bottom, % no box around the plot, only x and y axis
axis y line*=left, % the * suppresses the arrow tips
enlargelimits=upper] % extend the axes a bit to the right and top
\addplot[color=blue] {gauss(0,1)};
%\addplot[color=red] {gauss(6,1)};
\end{axis}
\end{tikzpicture}
\end{frame}

\begin{frame}
\textbf{Rechenregeln:}
\begin{align*}
\mathbb{E}(aX+b) = a\mathbb{E}(X) + b, \ \
\mathbb{V}(aX+b) = a^2\mathbb{V}(X) 
\end{align*}
\textbf{Beispiel:}
\begin{align*}
\mathbb{E}({\color{blue}{}X})=0 \ \Rightarrow \
\mathbb{E}({\color{red}{}X+6})=? 
\end{align*}
\begin{tikzpicture}[remember picture,overlay,scale=0.8]
\tikzset{shift={(current page.center)},xshift=-4cm,yshift=-5cm}
\begin{axis}[every axis plot post/.append style={
	mark=none,domain=-5:10,samples=40,smooth},
% All plots: from -2:2, 50 samples, smooth, no marks
axis x line*=bottom, % no box around the plot, only x and y axis
axis y line*=left, % the * suppresses the arrow tips
enlargelimits=upper] % extend the axes a bit to the right and top
\addplot[color=blue] {gauss(0,1)};
\addplot[color=red] {gauss(6,1)};
\end{axis}
\end{tikzpicture}
\end{frame}

\begin{frame}
\textbf{Rechenregeln:}
\begin{align*}
\mathbb{E}(aX+b) = a\mathbb{E}(X) + b, \ \
\mathbb{V}(aX+b) = a^2\mathbb{V}(X) 
\end{align*}
\textbf{Beispiel:}
\begin{align*}
\mathbb{E}({\color{blue}{}X})=0 \ \Rightarrow \
\mathbb{E}({\color{red}{}X+6})=6 
\end{align*}
\begin{tikzpicture}[remember picture,overlay,scale=0.8]
\tikzset{shift={(current page.center)},xshift=-4cm,yshift=-5cm}
\begin{axis}[every axis plot post/.append style={
	mark=none,domain=-5:10,samples=40,smooth},
% All plots: from -2:2, 50 samples, smooth, no marks
axis x line*=bottom, % no box around the plot, only x and y axis
axis y line*=left, % the * suppresses the arrow tips
enlargelimits=upper] % extend the axes a bit to the right and top
\addplot[color=blue] {gauss(0,1)};
\addplot[color=red] {gauss(6,1)};
\end{axis}
\end{tikzpicture}
\end{frame}

\begin{frame}
\textbf{Rechenregeln:}
\begin{align*}
\mathbb{E}(aX+b) = a\mathbb{E}(X) + b, \ \
\mathbb{V}(aX+b) = a^2\mathbb{V}(X) 
\end{align*}
\textbf{Beispiel:}
\begin{align*}
\mathbb{V}({\color{blue}{}X})=1 \ \Rightarrow \
\mathbb{V}({\color{red}{}X+6})=? 
\end{align*}
\begin{tikzpicture}[remember picture,overlay,scale=0.8]
\tikzset{shift={(current page.center)},xshift=-4cm,yshift=-5cm}
\begin{axis}[every axis plot post/.append style={
	mark=none,domain=-5:10,samples=40,smooth},
% All plots: from -2:2, 50 samples, smooth, no marks
axis x line*=bottom, % no box around the plot, only x and y axis
axis y line*=left, % the * suppresses the arrow tips
enlargelimits=upper] % extend the axes a bit to the right and top
\addplot[color=blue] {gauss(0,1)};
\addplot[color=red] {gauss(6,1)};
\end{axis}
\end{tikzpicture}
\end{frame}

\begin{frame}
\textbf{Rechenregeln:}
\begin{align*}
\mathbb{E}(aX+b) = a\mathbb{E}(X) + b, \ \
\mathbb{V}(aX+b) = a^2\mathbb{V}(X) 
\end{align*}
\textbf{Beispiel:}
\begin{align*}
\mathbb{V}({\color{blue}{}X})=1 \ \Rightarrow \
\mathbb{V}({\color{red}{}X+6})=1 
\end{align*}
\begin{tikzpicture}[remember picture,overlay,scale=0.8]
\tikzset{shift={(current page.center)},xshift=-4cm,yshift=-5cm}
\begin{axis}[every axis plot post/.append style={
	mark=none,domain=-5:10,samples=40,smooth},
% All plots: from -2:2, 50 samples, smooth, no marks
axis x line*=bottom, % no box around the plot, only x and y axis
axis y line*=left, % the * suppresses the arrow tips
enlargelimits=upper] % extend the axes a bit to the right and top
\addplot[color=blue] {gauss(0,1)};
\addplot[color=red] {gauss(6,1)};
\end{axis}
\end{tikzpicture}
\end{frame}

\begin{frame}
\textbf{Rechenregeln:}
\begin{align*}
\mathbb{E}(aX+b) = a\mathbb{E}(X) + b, \ \
\mathbb{V}(aX+b) = a^2\mathbb{V}(X) 
\end{align*}
\textbf{Beispiel:}
\begin{align*}
{\color{blue}X}\sim\mathcal{N}(0,1) \ \Rightarrow \
{\color{red}X+6}\sim\mathcal{N}(6,1)
\end{align*}
\begin{tikzpicture}[remember picture,overlay,scale=0.8]
\tikzset{shift={(current page.center)},xshift=-4cm,yshift=-5cm}
\begin{axis}[every axis plot post/.append style={
	mark=none,domain=-5:10,samples=40,smooth},
% All plots: from -2:2, 50 samples, smooth, no marks
axis x line*=bottom, % no box around the plot, only x and y axis
axis y line*=left, % the * suppresses the arrow tips
enlargelimits=upper] % extend the axes a bit to the right and top
\addplot[color=blue] {gauss(0,1)};
\addplot[color=red] {gauss(6,1)};
\end{axis}
\end{tikzpicture}
\end{frame}

\begin{frame}
\textbf{Rechenregeln:}
\begin{align*}
\mathbb{E}(aX+b) = a\mathbb{E}(X) + b, \ \
\mathbb{V}(aX+b) = a^2\mathbb{V}(X) 
\end{align*}
\textbf{Beispiel:}
\begin{align*}
\mathbb{E}({\color{blue}{}X})=2 \ \Rightarrow \
\mathbb{E}({\color{red}{}0,5\cdot X})=? 
\end{align*}
\end{frame}

\begin{frame}
\textbf{Rechenregeln:}
\begin{align*}
\mathbb{E}(aX+b) = a\mathbb{E}(X) + b, \ \
\mathbb{V}(aX+b) = a^2\mathbb{V}(X) 
\end{align*}
\textbf{Beispiel:}
\begin{align*}
\mathbb{E}({\color{blue}{}X})=2 \ \Rightarrow \
\mathbb{E}({\color{red}{}0,5\cdot X})=? 
\end{align*}
\begin{tikzpicture}[remember picture,overlay,scale=0.8]
\tikzset{shift={(current page.center)},xshift=-4cm,yshift=-5cm}
\begin{axis}[every axis plot post/.append style={
	mark=none,domain=-5:10,samples=40,smooth},
% All plots: from -2:2, 50 samples, smooth, no marks
axis x line*=bottom, % no box around the plot, only x and y axis
axis y line*=left, % the * suppresses the arrow tips
enlargelimits=upper] % extend the axes a bit to the right and top
\addplot[color=white] {gauss(1,0.5)};
\addplot[color=blue] {gauss(2,2)};
\end{axis}
\end{tikzpicture}
\end{frame}

\begin{frame}
\textbf{Rechenregeln:}
\begin{align*}
\mathbb{E}(aX+b) = a\mathbb{E}(X) + b, \ \
\mathbb{V}(aX+b) = a^2\mathbb{V}(X) 
\end{align*}
\textbf{Beispiel:}
\begin{align*}
\mathbb{E}({\color{blue}{}X})=2 \ \Rightarrow \
\mathbb{E}({\color{red}{}0,5\cdot X})=? 
\end{align*}
\begin{tikzpicture}[remember picture,overlay,scale=0.8]
\tikzset{shift={(current page.center)},xshift=-4cm,yshift=-5cm}
\begin{axis}[every axis plot post/.append style={
	mark=none,domain=-5:10,samples=40,smooth},
% All plots: from -2:2, 50 samples, smooth, no marks
axis x line*=bottom, % no box around the plot, only x and y axis
axis y line*=left, % the * suppresses the arrow tips
enlargelimits=upper] % extend the axes a bit to the right and top
\addplot[color=blue] {gauss(2,2)};
\addplot[color=red] {gauss(1,0.5)};
\end{axis}
\end{tikzpicture}
\end{frame}

\begin{frame}
\textbf{Rechenregeln:}
\begin{align*}
\mathbb{E}(aX+b) = a\mathbb{E}(X) + b, \ \
\mathbb{V}(aX+b) = a^2\mathbb{V}(X) 
\end{align*}
\textbf{Beispiel:}
\begin{align*}
\mathbb{E}({\color{blue}{}X})=2 \ \Rightarrow \
\mathbb{E}({\color{red}{}0,5\cdot X})=1 
\end{align*}
\begin{tikzpicture}[remember picture,overlay,scale=0.8]
\tikzset{shift={(current page.center)},xshift=-4cm,yshift=-5cm}
\begin{axis}[every axis plot post/.append style={
	mark=none,domain=-5:10,samples=40,smooth},
% All plots: from -2:2, 50 samples, smooth, no marks
axis x line*=bottom, % no box around the plot, only x and y axis
axis y line*=left, % the * suppresses the arrow tips
enlargelimits=upper] % extend the axes a bit to the right and top
\addplot[color=blue] {gauss(2,2)};
\addplot[color=red] {gauss(1,0.5)};
\end{axis}
\end{tikzpicture}
\end{frame}

\begin{frame}
\textbf{Rechenregeln:}
\begin{align*}
\mathbb{E}(aX+b) = a\mathbb{E}(X) + b, \ \
\mathbb{V}(aX+b) = a^2\mathbb{V}(X) 
\end{align*}
\textbf{Beispiel:}
\begin{align*}
\mathbb{V}({\color{blue}{}X})=2 \ \Rightarrow \
\mathbb{V}({\color{red}{}0,5\cdot X})=? 
\end{align*}
\begin{tikzpicture}[remember picture,overlay,scale=0.8]
\tikzset{shift={(current page.center)},xshift=-4cm,yshift=-5cm}
\begin{axis}[every axis plot post/.append style={
	mark=none,domain=-5:10,samples=40,smooth},
% All plots: from -2:2, 50 samples, smooth, no marks
axis x line*=bottom, % no box around the plot, only x and y axis
axis y line*=left, % the * suppresses the arrow tips
enlargelimits=upper] % extend the axes a bit to the right and top
\addplot[color=blue] {gauss(2,2)};
\addplot[color=red] {gauss(1,0.5)};
\end{axis}
\end{tikzpicture}
\end{frame}

\begin{frame}
\textbf{Rechenregeln:}
\begin{align*}
\mathbb{E}(aX+b) = a\mathbb{E}(X) + b, \ \
\mathbb{V}(aX+b) = a^2\mathbb{V}(X) 
\end{align*}
\textbf{Beispiel:}
\begin{align*}
\mathbb{V}({\color{blue}{}X})=2 \ \Rightarrow \
\mathbb{V}({\color{red}{}0,5\cdot X})=0,5^2\cdot \mathbb{V}({\color{blue}{}X})=0,5^2\cdot 2 = 0,5
\end{align*}
\begin{tikzpicture}[remember picture,overlay,scale=0.8]
\tikzset{shift={(current page.center)},xshift=-4cm,yshift=-5cm}
\begin{axis}[every axis plot post/.append style={
	mark=none,domain=-5:10,samples=40,smooth},
% All plots: from -2:2, 50 samples, smooth, no marks
axis x line*=bottom, % no box around the plot, only x and y axis
axis y line*=left, % the * suppresses the arrow tips
enlargelimits=upper] % extend the axes a bit to the right and top
\addplot[color=blue] {gauss(2,2)};
\addplot[color=red] {gauss(1,0.5)};
\end{axis}
\end{tikzpicture}
\end{frame}

\begin{frame}
\textbf{Rechenregeln:}
\begin{align*}
\mathbb{E}(aX+b) = a\mathbb{E}(X) + b, \ \
\mathbb{V}(aX+b) = a^2\mathbb{V}(X) 
\end{align*}
\textbf{Beispiel:}
\begin{align*}
{\color{blue}X}\sim\mathcal{N}(2,2) \ \Rightarrow \
{\color{red}0,5\cdot X}\sim\mathcal{N}(1,0.5)
\end{align*}
\begin{tikzpicture}[remember picture,overlay,scale=0.8]
\tikzset{shift={(current page.center)},xshift=-4cm,yshift=-5cm}
\begin{axis}[every axis plot post/.append style={
	mark=none,domain=-5:10,samples=40,smooth},
% All plots: from -2:2, 50 samples, smooth, no marks
axis x line*=bottom, % no box around the plot, only x and y axis
axis y line*=left, % the * suppresses the arrow tips
enlargelimits=upper] % extend the axes a bit to the right and top
\addplot[color=blue] {gauss(2,2)};
\addplot[color=red] {gauss(1,0.5)};
\end{axis}
\end{tikzpicture}
\end{frame}

\begin{frame}
	\textbf{Rechenregeln:}
	\begin{align*}
	\mathbb{E}(aX+b) = a\mathbb{E}(X) + b, \ \
	\mathbb{V}(aX+b) = a^2\mathbb{V}(X) 
	\end{align*}
	\begin{block}{Aufgabe}
		\begin{itemize}[<+->]
			\item Sei $X\sim \mathcal{U}_{[0,1]}$. Was ist dann $\mathbb{E}(2X)$?
			\item[] $X\sim \mathcal{U}_{[0,1]} \Rightarrow \mathbb{E}(X)=0,5 \Rightarrow \mathbb{E}(2X) = 1$
			\item Sei $Y\sim Poi(4)$. Was ist dann $\mathbb{V}(Y-0,5)$? 
			\item[] $Y\sim Poi(4) \Rightarrow \mathbb{V}(Y)=4 \Rightarrow \mathbb{V}(Y-0,5)=4$
			\item Sei $Z\sim \mathcal{N}(\mu,\sigma^2)$. Was ist dann die Verteilung von $\frac{Z-\mu}{\sigma} $?
			\item[] $Z\sim \mathcal{N}(\mu,\sigma^2) \Rightarrow \mathbb{E}\left(\frac{Z-\mu}{\sigma}\right) =0, \ \mathbb{V}\left(\frac{Z-\mu}{\sigma}\right) =1$
			\item[] $\Rightarrow \frac{Z-\mu}{\sigma} \sim \mathcal{N}(0,1)$ (\textbf{Standardisierung, z-Transformation})
		\end{itemize}
	\end{block}
\end{frame}



\end{document}