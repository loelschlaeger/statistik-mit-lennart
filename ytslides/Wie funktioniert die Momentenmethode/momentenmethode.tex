\documentclass[t,11pt,aspectratio=169]{beamer}
\usepackage{tikz}
\usepackage{pgfplots}
\usetikzlibrary{calc}
\usepackage[utf8]{inputenc}
\usepackage[ngerman]{babel}
\usepackage{amsmath,amsfonts,amssymb}
\usepackage{framed}
\usecolortheme{orchid}
\usepackage{etoolbox}
\useinnertheme[shadow=true]{rounded}

%%% PROGRESSBAR
\definecolor{pbblue}{HTML}{D8D8D8}% filling color for the progress bar
\definecolor{pbgray}{HTML}{F2F2F2}% background color for the progress bar
\useoutertheme{infolines}
\setbeamerfont{footline}{size=\normalsize}
\setbeamersize{text margin left=30pt,text margin right=30pt}
\makeatletter
\setbeamertemplate{footline}
{
	\leavevmode%
	\hbox{%
		\begin{beamercolorbox}[wd=.333333\paperwidth,ht=2.5ex,dp=1ex,center]{title in head/foot}%
			\usebeamerfont{title in head/foot}\insertshorttitle
		\end{beamercolorbox}%
		\begin{beamercolorbox}[wd=.333333\paperwidth,ht=2.5ex,dp=1ex,center]{date in head/foot}%
			%\usebeamerfont{date in head/foot}\insertshortdate{}\hspace*{2em}
			%\insertframenumber\hspace*{2ex} 
		\end{beamercolorbox}
		\begin{beamercolorbox}[wd=.333333\paperwidth,ht=3ex,dp=1ex,center]{author in head/foot}%
			\usebeamerfont{author in head/foot}\insertshortauthor~~%\beamer@ifempty{\insertshortinstitute}{}{(\insertshortinstitute)}
		\end{beamercolorbox}%
	}%
	\vskip0pt%
}
\makeatother
\makeatletter
\def\progressbar@progressbar{} % the progress bar
\newcount\progressbar@tmpcounta% auxiliary counter
\newcount\progressbar@tmpcountb% auxiliary counter
\newdimen\progressbar@pbht %progressbar height
\newdimen\progressbar@pbwd %progressbar width
\newdimen\progressbar@tmpdim % auxiliary dimension
\progressbar@pbwd=\linewidth
\progressbar@pbht=1.5ex
\def\progressbar@progressbar{%
    \progressbar@tmpcounta=\insertpagenumber
    \progressbar@tmpcountb=\insertdocumentendpage
    \progressbar@tmpdim=\progressbar@pbwd
    \multiply\progressbar@tmpdim by \progressbar@tmpcounta
    \divide\progressbar@tmpdim by \progressbar@tmpcountb
  \begin{tikzpicture}[rounded corners=3pt,very thin]
    \shade[top color=pbgray!20,bottom color=pbgray!20,middle color=pbgray!50]
      (0pt, 0pt) rectangle ++ (\progressbar@pbwd, \progressbar@pbht);
      \shade[draw=pbblue,top color=pbblue!50,bottom color=pbblue!50,middle color=pbblue] %
        (0pt, 0pt) rectangle ++ (\progressbar@tmpdim, \progressbar@pbht);
    \draw[color=normal text.fg!50]  
      (0pt, 0pt) rectangle (\progressbar@pbwd, \progressbar@pbht) 
        node[pos=0.5,color=normal text.fg] {\textnormal{%
             \pgfmathparse{\insertpagenumber*100/\insertdocumentendpage}%
             \pgfmathprintnumber[fixed,precision=0]{\pgfmathresult}\,\%%
        }%
    };
  \end{tikzpicture}%
}
\addtobeamertemplate{headline}{}
{%
  \begin{beamercolorbox}[wd=\paperwidth,ht=4ex,center,dp=1ex]{white}%
    \progressbar@progressbar%
  \end{beamercolorbox}%
}
\makeatother

\setbeamertemplate{frametitle}[default][center]

%%% BLOCKS
% block = Aufgabe
\setbeamercolor{block title}{fg=black,bg=blue!30!white} 
\setbeamercolor{block body}{fg=black, bg=blue!3!white}

% alertblock = Definition
\setbeamercolor{block title alerted}{fg=black,bg=red!50!white} 
\setbeamercolor{block body alerted}{fg=black, bg=red!3!white}

% exampleblock = Wiederholung
\setbeamercolor{block title example}{fg=black,bg=green!30!white} 
\setbeamercolor{block body example}{fg=black, bg=green!3!white}

\setbeamercovered{transparent}
\setbeamertemplate{navigation symbols}{}

\addtocounter{page}{-1}
\addtocounter{framenumber}{-1}
\setbeamercovered{invisible}



\begin{document}

\begin{frame}
\begin{block}{Beispiel 1 (Gleichverteilung)}
	Wir haben die folgenden 10 unabhängigen Realisationen einer Gleichverteilung auf dem Intervall $[0,\theta]$ gegeben:
	$$0.6 \ \ 5.5 \ \ 0.9 \ \ 4.5 \ \ 6.6 \ \ 2.0 \ \ 3.3 \ \ 5.0 \ \ 3.5 \ \ 3.1$$
	Schätze mit Hilfe der Momentenmethode den Wert von $\theta$.
\end{block}
\end{frame}

\begin{frame}
\begin{exampleblock}{Fahrplan für die Momentenmethode}
\begin{enumerate}[<+->]
	\item Gegeben seien $X_1,\dots,X_n$ unabhängig und identisch verteilte Zufallsvariablen.
	\item Wir kennen die Verteilung, aber ein oder mehrere Parameter sind unbekannt. Diese wollen wir schätzen.
	\item Wir bestimmen die ersten $k$ theoretischen Momente $\mathbb{E}(X^k)$ der Verteilung und ermitteln einen Zusammenhang zu den gesuchten Parametern.
	\item Wir ersetzen die theoretischen Momente $\mathbb{E}(X^k)$ durch die empirischen Momente $\frac{1}{n}\sum_{i=1}^{n}X_i^k$ und lösen nach den gesuchten Parametern auf.
	\item Wir erhalten einen Ausdruck, der die gesuchten Parameter durch die empirischen Momente erklärt. Diesen nennen wir Momentenschätzer.
\end{enumerate}
\end{exampleblock}
\end{frame}

\begin{frame}
\begin{enumerate}[<+->]
\item $X_1,\dots,X_n$ \textit{iid} gleichverteilt auf dem Intervall $[0,\theta]$
\item wir wollen $\theta$ schätzen
\item erstes theoretisches Moment ($k=1$): 
\begin{align*}
\mathbb{E}(X) &= \frac{\theta}{2} \\
\Longleftrightarrow \ \ \theta &= 2\cdot \mathbb{E}(X)
\end{align*}
\item erstes empirisches Moment ($k=1$):
\begin{align*}
\frac{1}{n}\sum_{i=1}^{n}X_i
\end{align*}
\item theoretisches durch empirisches Moment ersetzen:
\begin{align*}
\hat{\theta}_{MM} &= 2\cdot\frac{1}{n}\sum_{i=1}^{n}X_i=2\cdot \bar{X}
\end{align*}

\end{enumerate}
\end{frame}

\begin{frame}
\begin{block}{Beispiel 1 (Gleichverteilung)}
	Wir haben die folgenden 10 unabhängigen Realisationen einer Gleichverteilung auf dem Intervall $[0,\theta]$ gegeben:
	$$0.6 \ \ 5.5 \ \ 0.9 \ \ 4.5 \ \ 6.6 \ \ 2.0 \ \ 3.3 \ \ 5.0 \ \ 3.5 \ \ 3.1$$
	Schätze mit Hilfe der Momentenmethode den Wert von $\theta$.
\end{block}
\pause 
\begin{align*}
\hat{\theta}_{MM} &= 2\cdot\frac{1}{n}\sum_{i=1}^{n}X_i = 2\cdot 3.5 = 7
\end{align*}
\end{frame}

\begin{frame}
\begin{block}{Beispiel 2 (Poisson Verteilung)}
	Gegeben seien $X_1,\dots,X_n\overset{iid}{\sim}Poi(\lambda)$. Bestimme den Momentenschätzer $\hat{\lambda}_{MM}$ für den Parameter $\lambda$.
\end{block}
\pause
Wir wissen bereits:
\begin{align*}
\hat{\lambda}_{ML}=\frac{1}{n} \sum_{i=1}^{n} X_i=\bar{X}
\end{align*}
\pause
Aber was ist mit $\hat{\lambda}_{MM}$?
\end{frame}

\begin{frame}
\begin{enumerate}[<+->]
\item $X_1,\dots,X_n\overset{iid}{\sim}Poi(\lambda)$
\item wir wollen $\lambda$ schätzen
\item erstes theoretisches Moment ($k=1$):
\begin{itemize}
\item $\mathbb{E}(X) = \lambda	$
\end{itemize}
\item erstes empirisches Moment ($k=1$):
\begin{itemize}
\item $\frac{1}{n}\sum_{i=1}^{n}X_i$
\end{itemize}
\item theoretisches durch empirisches Moment ersetzen:
\begin{align*}
\hat{\lambda}_{MM} &= \frac{1}{n}\sum_{i=1}^{n}X_i
\end{align*}
\end{enumerate}
\end{frame}

\begin{frame}
\begin{block}{Beispiel 2 (Poisson Verteilung)}
	Gegeben seien $X_1,\dots,X_n\overset{iid}{\sim}Poi(\lambda)$. Bestimme den Momentenschätzer $\hat{\lambda}_{MM}$ für den Parameter $\lambda$.
\end{block}
\begin{align*}
\hat{\lambda}_{ML}&=\frac{1}{n} \sum_{i=1}^{n} X_i=\bar{X} \\
\hat{\lambda}_{MM}&=\frac{1}{n} \sum_{i=1}^{n} X_i=\bar{X} \\
\end{align*}
\end{frame}

\begin{frame}
\begin{block}{Beispiel 3 (Normalverteilung)}
	Seien $X_1,\dots,X_n\overset{iid}{\sim}\mathcal{N}(\mu,\sigma^2)$. Bestimme die Momentenschätzer $\hat{\mu}_{MM}$ und $\hat{\sigma^2}_{MM}$.
\end{block}
\end{frame}

\begin{frame}
\begin{enumerate}[<+->]
\item $X_1,\dots,X_n\overset{iid}{\sim}\mathcal{N}(\mu,\sigma^2)$
\item wir wollen $\mu$ und $\sigma^2$ schätzen
\item erstes und zweites theoretisches Moment ($k=1$ und $k=2$):
\begin{itemize}
\item $\mathbb{E}(X) = \mu	$
\item $\mathbb{V}(X)=\mathbb{E}(X^2)-\mathbb{E}(X)^2$
\item[] $ \Longleftrightarrow \mathbb{E}(X^2) = \mathbb{V}(X) + \mathbb{E}(X)^2 = \sigma^2 + \mu^2$
\end{itemize}
\item erstes und zweites empirisches Moment ($k=1$ und $k=2$):
\begin{itemize}
\item $k=1$: $\frac{1}{n}\sum_{i=1}^{n}X_i$, $k=2$: $\frac{1}{n}\sum_{i=1}^{n}X_i^2$
\end{itemize}
\item theoretisches durch empirisches Moment ersetzen:
\begin{align*}
\hat{\mu}_{MM} &= \frac{1}{n}\sum_{i=1}^{n}X_i\\
\hat{\sigma^2}_{MM} &= \frac{1}{n}\sum_{i=1}^{n}X_i^2 - \left(\frac{1}{n}\sum_{i=1}^{n}X_i\right)^2
\end{align*}
\end{enumerate}
\end{frame}

\begin{frame}
\begin{block}{Beispiel 3 (Normalverteilung)}
	Seien $X_1,\dots,X_n\overset{iid}{\sim}\mathcal{N}(\mu,\sigma^2)$. Bestimme die Momentenschätzer $\hat{\mu}_{MM}$ und $\hat{\sigma^2}_{MM}$.
\end{block}
\begin{align*}
\hat{\mu}_{MM} &= \frac{1}{n}\sum_{i=1}^{n}X_i\\
\hat{\sigma^2}_{MM} &= \frac{1}{n}\sum_{i=1}^{n}X_i^2 - \left(\frac{1}{n}\sum_{i=1}^{n}X_i\right)^2
\end{align*}
\end{frame}

\end{document}