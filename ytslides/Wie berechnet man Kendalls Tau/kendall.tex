\documentclass[t,11pt]{beamer}
\usepackage{tikz}
\usetikzlibrary{calc}
\usepackage[utf8]{inputenc}
\usepackage[ngerman]{babel}
\usepackage{amsmath}
\usepackage{amsfonts}
\usepackage{amssymb}
\usepackage{framed}
\colorlet{LightBlue}{blue}

% Theorem box
\pgfdeclarelayer{background}
\pgfsetlayers{background,main}
\tikzstyle{thmbox} = [inner sep=1em]
\tikzstyle{thmborder} = [draw=blue, fill=none,line width =1.pt, rounded corners=5pt]

\def\parchmentframe#1{
	\tikz{
		\node[thmbox] (A) {#1};
		\begin{pgfonlayer}{background}
			\fill[thmborder] 
			(A.south east) -- (A.south west) -- 
			(A.north west) -- (A.north east) -- cycle;
\end{pgfonlayer}}}

\def\parchmentframetop#1{
	\tikz{
		\node[thmbox] (A) {#1};
		\begin{pgfonlayer}{background}    
			\fill[thmborder]
			(A.south west) -- (A.north west) -- 
			(A.north east) -- (A.south east);
\end{pgfonlayer}}}

\def\parchmentframebottom#1{
	\tikz{
		\node[thmbox] (A) {#1};
		\begin{pgfonlayer}{background}    
			\fill[thmborder]
			(A.north west) -- (A.south west) -- 
			(A.south east) -- (A.north east);
\end{pgfonlayer}}}

\def\parchmentframemiddle#1{
	\tikz{
		\node[thmbox] (A) {#1};
		\begin{pgfonlayer}{background}    
			\fill[thmborder]
			(A.north west) -- (A.south west);
			\fill[thmborder]
			(A.south east) -- (A.north east);
\end{pgfonlayer}}}

\newenvironment<>{myTheorem}[1][]{%
	\def\FrameCommand{\parchmentframe}%
	\def\FirstFrameCommand{\parchmentframetop}%
	\def\LastFrameCommand{\parchmentframebottom}%
	\def\MidFrameCommand{\parchmentframemiddle}%
	\vskip\baselineskip
	\MakeFramed{\FrameRestore}
	\noindent\tikz\node[inner sep=1.2ex, draw=blue, fill=blue!10,
	anchor=west, overlay, line width = 1.pt, rounded corners=4pt] at (0em, 1em) 
	{\color{LightBlue}{Aufgabe\if\relax\detokenize{#1}\relax\else\space (#1)\fi}};\par\nobreak}%
{\endMakeFramed}
%%end theorem box

\definecolor{pbblue}{HTML}{D8D8D8}% filling color for the progress bar
\definecolor{pbgray}{HTML}{F2F2F2}% background color for the progress bar

\useoutertheme{infolines}
\setbeamerfont{footline}{size=\normalsize}
\setbeamersize{text margin left=30pt,text margin right=30pt}
\makeatletter
\setbeamertemplate{footline}
{
	\leavevmode%
	\hbox{%
		\begin{beamercolorbox}[wd=.333333\paperwidth,ht=2.5ex,dp=1ex,center]{title in head/foot}%
			\usebeamerfont{title in head/foot}\insertshorttitle
		\end{beamercolorbox}%
		\begin{beamercolorbox}[wd=.333333\paperwidth,ht=2.5ex,dp=1ex,center]{date in head/foot}%
			%\usebeamerfont{date in head/foot}\insertshortdate{}\hspace*{2em}
			%\insertframenumber\hspace*{2ex} 
		\end{beamercolorbox}
		\begin{beamercolorbox}[wd=.333333\paperwidth,ht=3ex,dp=1ex,center]{author in head/foot}%
			\usebeamerfont{author in head/foot}\insertshortauthor~~%\beamer@ifempty{\insertshortinstitute}{}{(\insertshortinstitute)}
		\end{beamercolorbox}%
	}%
	\vskip0pt%
}
\makeatother


\makeatletter
\def\progressbar@progressbar{} % the progress bar
\newcount\progressbar@tmpcounta% auxiliary counter
\newcount\progressbar@tmpcountb% auxiliary counter
\newdimen\progressbar@pbht %progressbar height
\newdimen\progressbar@pbwd %progressbar width
\newdimen\progressbar@tmpdim % auxiliary dimension

\progressbar@pbwd=\linewidth
\progressbar@pbht=1.5ex

% the progress bar
\def\progressbar@progressbar{%

    \progressbar@tmpcounta=\insertframenumber
    \progressbar@tmpcountb=\inserttotalframenumber
    \progressbar@tmpdim=\progressbar@pbwd
    \multiply\progressbar@tmpdim by \progressbar@tmpcounta
    \divide\progressbar@tmpdim by \progressbar@tmpcountb

  \begin{tikzpicture}[rounded corners=3pt,very thin]

    \shade[top color=pbgray!20,bottom color=pbgray!20,middle color=pbgray!50]
      (0pt, 0pt) rectangle ++ (\progressbar@pbwd, \progressbar@pbht);

      \shade[draw=pbblue,top color=pbblue!50,bottom color=pbblue!50,middle color=pbblue] %
        (0pt, 0pt) rectangle ++ (\progressbar@tmpdim, \progressbar@pbht);

    \draw[color=normal text.fg!50]  
      (0pt, 0pt) rectangle (\progressbar@pbwd, \progressbar@pbht) 
        node[pos=0.5,color=normal text.fg] {die Aufgabe ist zu \textnormal{%
             \pgfmathparse{\insertframenumber*100/\inserttotalframenumber}%
             \pgfmathprintnumber[fixed,precision=0]{\pgfmathresult}\,\% gelöst%
        }%
    };
  \end{tikzpicture}%
}

\addtobeamertemplate{headline}{}
{%
  \begin{beamercolorbox}[wd=\paperwidth,ht=4ex,center,dp=1ex]{white}%
    \progressbar@progressbar%
  \end{beamercolorbox}%
}
\makeatother

\begin{document}
	\author{www.oilbat.de}
	%\title{Stochastik}
	\subtitle{}
	\logo{}
	\institute{}
	\date{}
	\subject{}
	\setbeamercovered{transparent}
	\setbeamertemplate{navigation symbols}{}

\addtocounter{framenumber}{-1}
\setbeamercovered{invisible}

\begin{frame}
\begin{myTheorem}
Zwei deiner Freunde, Betty und Tim, möchten Kendalls Tau für ihre Präferenzen bei den folgenden vier Snacks berechnen: Chips, Schokolade, Lakritze und Eis. Betty hat die folgende Präferenzordnung: 1. Eis, 2. Schokolade, 3. Lakritze und 4. Chips. Tims Präferenzordnung ist: 1. Chips, 2. Eis, 3. Schokolade und 4. Lakritze. Kannst du ihnen helfen und Kendalls Tau berechnen? 
\end{myTheorem}

\end{frame}

\begin{frame}
Was ist Kendalls Tau eigentlich?
\end{frame}

\begin{frame}
Was ist Kendalls Tau eigentlich?
\begin{itemize}
	\item ein Zusammenhangsmaß
\end{itemize}
\end{frame}

\begin{frame}
Was ist Kendalls Tau eigentlich?
\begin{itemize}
	\item ein Zusammenhangsmaß
	\item für zwei mindestens ordinalskalierte Merkmale
\end{itemize}
\end{frame}

\begin{frame}
Was ist Kendalls Tau eigentlich?
\begin{itemize}
	\item ein Zusammenhangsmaß
	\item für zwei mindestens ordinalskalierte Merkmale
	\item Notation: $\tau$
\end{itemize}
\end{frame}

\begin{frame}
Was ist Kendalls Tau eigentlich?
\begin{itemize}
	\item ein Zusammenhangsmaß
	\item für zwei mindestens ordinalskalierte Merkmale
	\item Notation: $\tau$
	\item Prinzip: Geht von der nach dem 1. Merkmal sortierten Rangfolge aus. Misst dann, wie oft die Rangfolge der Beobachtungen vom 2. Merkmal diese Rangfolge \glqq durchbricht\grqq. Diese Anzahl wird durch die Anzahl der prinzipiell möglichen Rangfolgen dividiert.
\end{itemize}
\end{frame}

\begin{frame}
Was ist Kendalls Tau eigentlich?
\begin{itemize}
	\item ein Zusammenhangsmaß
	\item für zwei mindestens ordinalskalierte Merkmale
	\item Notation: $\tau$
	\item Prinzip: Geht von der nach dem 1. Merkmal sortierten Rangfolge aus. Misst dann, wie oft die Rangfolge der Beobachtungen vom 2. Merkmal diese Rangfolge \glqq durchbricht\grqq. Diese Anzahl wird durch die Anzahl der prinzipiell möglichen Rangfolgen dividiert.
	\item $\tau \in [-1;1]$
	\item $\tau\approx -1$: negativer Zusammenhang
	\item $\tau \approx 0$: kein Zusammenhang
	\item $\tau \approx +1$: positiver Zusammenhang
\end{itemize}
\end{frame}

\begin{frame}
\begin{tabular}{l|llll}
	& 1. Präferenz & 2. Präferenz & 3. Präferenz & 4. Präferenz \\
	\hline
	Betty & Eis & Schokolade & Lakritze & Chips \\
	Tim & Chips & Eis & Schokolade & Lakritze
\end{tabular}
\end{frame}

\begin{frame}
\begin{tabular}{l|llll}
	& 1. Präferenz & 2. Präferenz & 3. Präferenz & 4. Präferenz \\
	\hline
	Betty & Eis & Schokolade & Lakritze & Chips \\
	Tim & Chips & Eis & Schokolade & Lakritze
\end{tabular}
\vspace{0.3cm}

\begin{tabular}{l|llll}
	& 1. Präferenz & 2. Präferenz & 3. Präferenz & 4. Präferenz \\
	\hline
	Betty & Eis & Schokolade & Lakritze & Chips \\
	Tim & Chips & Eis & Schokolade & Lakritze
\end{tabular}
\end{frame}


\begin{frame}
\begin{tabular}{l|llll}
	& 1. Präferenz & 2. Präferenz & 3. Präferenz & 4. Präferenz \\
	\hline
	Betty & Eis & Schokolade & Lakritze & Chips \\
	Tim & Chips & Eis & Schokolade & Lakritze
\end{tabular}
\vspace{0.3cm}

\begin{tabular}{l|llll}
	& 1. Präferenz & 2. Präferenz & 3. Präferenz & 4. Präferenz \\
	\hline
	Betty & 1 & Schokolade & Lakritze & Chips \\
	Tim & Chips & 1 & Schokolade & Lakritze
\end{tabular}
\end{frame}

\begin{frame}
\begin{tabular}{l|llll}
	& 1. Präferenz & 2. Präferenz & 3. Präferenz & 4. Präferenz \\
	\hline
	Betty & Eis & Schokolade & Lakritze & Chips \\
	Tim & Chips & Eis & Schokolade & Lakritze
\end{tabular}
\vspace{0.3cm}

\begin{tabular}{l|llll}
	& 1. Präferenz & 2. Präferenz & 3. Präferenz & 4. Präferenz \\
	\hline
	Betty & 1 & 2 & Lakritze & Chips \\
	Tim & Chips & 1 & 2 & Lakritze
\end{tabular}
\end{frame}

\begin{frame}
\begin{tabular}{l|llll}
	& 1. Präferenz & 2. Präferenz & 3. Präferenz & 4. Präferenz \\
	\hline
	Betty & Eis & Schokolade & Lakritze & Chips \\
	Tim & Chips & Eis & Schokolade & Lakritze
\end{tabular}
\vspace{0.3cm}

\begin{tabular}{l|llll}
	& 1. Präferenz & 2. Präferenz & 3. Präferenz & 4. Präferenz \\
	\hline
	Betty & 1 & 2 & 3 & Chips \\
	Tim & Chips & 1 & 2 & 3
\end{tabular}
\end{frame}

\begin{frame}
\begin{tabular}{l|llll}
	& 1. Präferenz & 2. Präferenz & 3. Präferenz & 4. Präferenz \\
	\hline
	Betty & Eis & Schokolade & Lakritze & Chips \\
	Tim & Chips & Eis & Schokolade & Lakritze
\end{tabular}
\vspace{0.3cm}

\begin{tabular}{l|llll}
	& 1. Präferenz & 2. Präferenz & 3. Präferenz & 4. Präferenz \\
	\hline
	Betty & 1 & 2 & 3 & 4 \\
	Tim & 4 & 1 & 2 & 3
\end{tabular}
\end{frame}

\begin{frame}
\begin{tabular}{l|llll}
	& 1. Präferenz & 2. Präferenz & 3. Präferenz & 4. Präferenz \\
	\hline
	Betty & 1 & 2 & 3 & 4 \\
	Tim & 4 & 1 & 2 & 3
\end{tabular}
\end{frame}

\begin{frame}
\begin{tabular}{l|llll}
	& 1. Präferenz & 2. Präferenz & 3. Präferenz & 4. Präferenz \\
	\hline
	Betty & 1 & 2 & 3 & 4 \\
	Tim & 4 & 1 & 2 & 3 \\
	\hline
	Paare & (1,4) & (2,1) & (3,2) & (4,3)
\end{tabular}
\end{frame}

\begin{frame}
\begin{tabular}{l|llll}
	& 1. Präferenz & 2. Präferenz & 3. Präferenz & 4. Präferenz \\
	\hline
	Betty & 1 & 2 & 3 & 4 \\
	Tim & 4 & 1 & 2 & 3 \\
	\hline
	Paare & (1,4) & (2,1) & (3,2) & (4,3)
\end{tabular}
\vspace{0.3cm}

Vergleiche:
\vspace{0.3cm}

\begin{tabular}{|c|ccc|}
	\hline
	(1,4) & (2,1) & (3,2) & (4,3) \\
	\hline
	(2,1) & (3,2) & (4,3) &		 \\
	\hline
	(3,2) & (4,3) & 	  &  \\
	\hline
\end{tabular}

\end{frame}

\begin{frame}
\begin{tabular}{l|llll}
	& 1. Präferenz & 2. Präferenz & 3. Präferenz & 4. Präferenz \\
	\hline
	Betty & 1 & 2 & 3 & 4 \\
	Tim & 4 & 1 & 2 & 3 \\
	\hline
	Paare & (1,4) & (2,1) & (3,2) & (4,3)
\end{tabular}
\vspace{0.3cm}

Vergleiche:
\vspace{0.3cm}

\begin{tabular}{|c|ccc|}
	\hline
	(1,4) & (2,1) & (3,2) & (4,3) \\
	\hline
	(2,1) & (3,2) & (4,3) &		 \\
	\hline
	(3,2) & (4,3) & 	  &  \\
	\hline
\end{tabular}
\vspace{0.3cm}

Zähler:
\vspace{0.3cm}

\begin{tabular}{c|c}
	konkordant & diskonkordant \\
	\hline 
	& 
\end{tabular}

\end{frame}

\begin{frame}
\begin{tabular}{l|llll}
	& 1. Präferenz & 2. Präferenz & 3. Präferenz & 4. Präferenz \\
	\hline
	Betty & 1 & 2 & 3 & 4 \\
	Tim & 4 & 1 & 2 & 3 \\
	\hline
	Paare & (1,4) & (2,1) & (3,2) & (4,3)
\end{tabular}
\vspace{0.3cm}

Vergleiche:
\vspace{0.3cm}

\begin{tabular}{|c|ccc|}
	\hline
	\textbf{(1,4)} & \textbf{(2,1)} & \textbf{(3,2)} & \textbf{(4,3)} \\
	\hline
	(2,1) & (3,2) & (4,3) &		 \\
	\hline
	(3,2) & (4,3) & 	  &  \\
	\hline
\end{tabular}
\vspace{0.3cm}

Zähler:
\vspace{0.3cm}

\begin{tabular}{c|c}
	konkordant & diskonkordant \\
	\hline 
	& 
\end{tabular}

\end{frame}

\begin{frame}
\begin{tabular}{l|llll}
	& 1. Präferenz & 2. Präferenz & 3. Präferenz & 4. Präferenz \\
	\hline
	Betty & 1 & 2 & 3 & 4 \\
	Tim & 4 & 1 & 2 & 3 \\
	\hline
	Paare & (1,4) & (2,1) & (3,2) & (4,3)
\end{tabular}
\vspace{0.3cm}

Vergleiche:
\vspace{0.3cm}

\begin{tabular}{|c|ccc|}
	\hline
	\textbf{(1,4)} & \textbf{(2,1)} & \textbf{(3,2)} & \textbf{(4,3)} \\
	\hline
	(2,1) & (3,2) & (4,3) &		 \\
	\hline
	(3,2) & (4,3) & 	  &  \\
	\hline
\end{tabular}
\vspace{0.3cm}

Zähler:
\vspace{0.3cm}

\begin{tabular}{c|c}
	konkordant & diskonkordant \\
	\hline 
	& \textbf{3}
\end{tabular}

\end{frame}

\begin{frame}
\begin{tabular}{l|llll}
	& 1. Präferenz & 2. Präferenz & 3. Präferenz & 4. Präferenz \\
	\hline
	Betty & 1 & 2 & 3 & 4 \\
	Tim & 4 & 1 & 2 & 3 \\
	\hline
	Paare & (1,4) & (2,1) & (3,2) & (4,3)
\end{tabular}
\vspace{0.3cm}

Vergleiche:
\vspace{0.3cm}

\begin{tabular}{|c|ccc|}
	\hline
	(1,4) & (2,1) & (3,2) & (4,3) \\
	\hline
	\textbf{(2,1)} & \textbf{(3,2)} & \textbf{(4,3)} &		 \\
	\hline
	(3,2) & (4,3) & 	  &  \\
	\hline
\end{tabular}
\vspace{0.3cm}

Zähler:
\vspace{0.3cm}

\begin{tabular}{c|c}
	konkordant & diskonkordant \\
	\hline 
	& 3
\end{tabular}

\end{frame}

\begin{frame}
\begin{tabular}{l|llll}
	& 1. Präferenz & 2. Präferenz & 3. Präferenz & 4. Präferenz \\
	\hline
	Betty & 1 & 2 & 3 & 4 \\
	Tim & 4 & 1 & 2 & 3 \\
	\hline
	Paare & (1,4) & (2,1) & (3,2) & (4,3)
\end{tabular}
\vspace{0.3cm}

Vergleiche:
\vspace{0.3cm}

\begin{tabular}{|c|ccc|}
	\hline
	(1,4) & (2,1) & (3,2) & (4,3) \\
	\hline
	\textbf{(2,1)} & \textbf{(3,2)} & \textbf{(4,3)} &		 \\
	\hline
	(3,2) & (4,3) & 	  &  \\
	\hline
\end{tabular}
\vspace{0.3cm}

Zähler:
\vspace{0.3cm}

\begin{tabular}{c|c}
	konkordant & diskonkordant \\
	\hline 
	\textbf{2}& 3
\end{tabular}

\end{frame}

\begin{frame}
\begin{tabular}{l|llll}
	& 1. Präferenz & 2. Präferenz & 3. Präferenz & 4. Präferenz \\
	\hline
	Betty & 1 & 2 & 3 & 4 \\
	Tim & 4 & 1 & 2 & 3 \\
	\hline
	Paare & (1,4) & (2,1) & (3,2) & (4,3)
\end{tabular}
\vspace{0.3cm}

Vergleiche:
\vspace{0.3cm}

\begin{tabular}{|c|ccc|}
	\hline
	(1,4) & (2,1) & (3,2) & (4,3) \\
	\hline
	(2,1) & (3,2) & (4,3) &		 \\
	\hline
	\textbf{(3,2)} & \textbf{(4,3)} & 	  &  \\
	\hline
\end{tabular}
\vspace{0.3cm}

Zähler:
\vspace{0.3cm}

\begin{tabular}{c|c}
	konkordant & diskonkordant \\
	\hline 
	2 & 3
\end{tabular}

\end{frame}

\begin{frame}
\begin{tabular}{l|llll}
	& 1. Präferenz & 2. Präferenz & 3. Präferenz & 4. Präferenz \\
	\hline
	Betty & 1 & 2 & 3 & 4 \\
	Tim & 4 & 1 & 2 & 3 \\
	\hline
	Paare & (1,4) & (2,1) & (3,2) & (4,3)
\end{tabular}
\vspace{0.3cm}

Vergleiche:
\vspace{0.3cm}

\begin{tabular}{|c|ccc|}
	\hline
	(1,4) & (2,1) & (3,2) & (4,3) \\
	\hline
	(2,1) & (3,2) & (4,3) &		 \\
	\hline
	\textbf{(3,2)} & \textbf{(4,3)} & 	  &  \\
	\hline
\end{tabular}
\vspace{0.3cm}

Zähler:
\vspace{0.3cm}

\begin{tabular}{c|c}
	konkordant & diskonkordant \\
	\hline 
	\textbf{3} & 3
\end{tabular}

\end{frame}

\begin{frame}
	\begin{tabular}{c|c}
		konkordant & diskonkordant \\
		\hline 
		3 & 3
	\end{tabular}
\vspace{0.3cm}

$S=$\# konkordant $-$ \# diskonkordant $= 3-3 = 0$
\end{frame}

\begin{frame}
\begin{tabular}{c|c}
	konkordant & diskonkordant \\
	\hline 
	3 & 3
\end{tabular}
\vspace{0.3cm}

$S=$\# konkordant $-$ \# diskonkordant $= 3-3 = 0$
\vspace{0.3cm}

Formel für Kendalls Tau: $$\tau = \frac{S}{\frac{n(n-1)}{2}}$$
\end{frame}

\begin{frame}
\begin{tabular}{c|c}
	konkordant & diskonkordant \\
	\hline 
	3 & 3
\end{tabular}
\vspace{0.3cm}

$S=$\# konkordant $-$ \# diskonkordant $= 3-3 = 0$
\vspace{0.3cm}

Formel für Kendalls Tau: $$\tau = \frac{S}{\frac{n(n-1)}{2}}$$

\vspace{0.3cm}

Bei uns: $$\tau = \frac{0}{\frac{4\cdot 3}{2}}=0 \Rightarrow \text{kein Zusammenhang}$$
\end{frame}

\begin{frame}
\begin{myTheorem}
	Wie müssten die Präferenzen von Betty und Tim aussehen, damit Kendalls Tau einen Wert von +1 bzw. -1 annimmt? Überprüfe deine Überlegung! 
\end{myTheorem}
\end{frame}


\end{document}