\documentclass[t,11pt]{beamer}
\usepackage{tikz}
\usepackage{pgfplots}
\usetikzlibrary{calc}
\usepackage[utf8]{inputenc}
\usepackage[ngerman]{babel}
\usepackage{amsmath,amsfonts,amssymb}
\usepackage{framed}
\usecolortheme{orchid}
\usepackage{etoolbox}
\useinnertheme[shadow=true]{rounded}

%%% PROGRESSBAR
\definecolor{pbblue}{HTML}{D8D8D8}% filling color for the progress bar
\definecolor{pbgray}{HTML}{F2F2F2}% background color for the progress bar
\useoutertheme{infolines}
\setbeamerfont{footline}{size=\normalsize}
\setbeamersize{text margin left=30pt,text margin right=30pt}
\makeatletter
\setbeamertemplate{footline}
{
	\leavevmode%
	\hbox{%
		\begin{beamercolorbox}[wd=.333333\paperwidth,ht=2.5ex,dp=1ex,center]{title in head/foot}%
			\usebeamerfont{title in head/foot}\insertshorttitle
		\end{beamercolorbox}%
		\begin{beamercolorbox}[wd=.333333\paperwidth,ht=2.5ex,dp=1ex,center]{date in head/foot}%
			%\usebeamerfont{date in head/foot}\insertshortdate{}\hspace*{2em}
			%\insertframenumber\hspace*{2ex} 
		\end{beamercolorbox}
		\begin{beamercolorbox}[wd=.333333\paperwidth,ht=3ex,dp=1ex,center]{author in head/foot}%
			\usebeamerfont{author in head/foot}\insertshortauthor~~%\beamer@ifempty{\insertshortinstitute}{}{(\insertshortinstitute)}
		\end{beamercolorbox}%
	}%
	\vskip0pt%
}
\makeatother
\makeatletter
\def\progressbar@progressbar{} % the progress bar
\newcount\progressbar@tmpcounta% auxiliary counter
\newcount\progressbar@tmpcountb% auxiliary counter
\newdimen\progressbar@pbht %progressbar height
\newdimen\progressbar@pbwd %progressbar width
\newdimen\progressbar@tmpdim % auxiliary dimension
\progressbar@pbwd=\linewidth
\progressbar@pbht=1.5ex
\def\progressbar@progressbar{%
    \progressbar@tmpcounta=\insertframenumber
    \progressbar@tmpcountb=\inserttotalframenumber
    \progressbar@tmpdim=\progressbar@pbwd
    \multiply\progressbar@tmpdim by \progressbar@tmpcounta
    \divide\progressbar@tmpdim by \progressbar@tmpcountb
  \begin{tikzpicture}[rounded corners=3pt,very thin]
    \shade[top color=pbgray!20,bottom color=pbgray!20,middle color=pbgray!50]
      (0pt, 0pt) rectangle ++ (\progressbar@pbwd, \progressbar@pbht);
      \shade[draw=pbblue,top color=pbblue!50,bottom color=pbblue!50,middle color=pbblue] %
        (0pt, 0pt) rectangle ++ (\progressbar@tmpdim, \progressbar@pbht);
    \draw[color=normal text.fg!50]  
      (0pt, 0pt) rectangle (\progressbar@pbwd, \progressbar@pbht) 
        node[pos=0.5,color=normal text.fg] {\textnormal{%
             \pgfmathparse{\insertframenumber*100/\inserttotalframenumber}%
             \pgfmathprintnumber[fixed,precision=0]{\pgfmathresult}\,\%%
        }%
    };
  \end{tikzpicture}%
}
\addtobeamertemplate{headline}{}
{%
  \begin{beamercolorbox}[wd=\paperwidth,ht=4ex,center,dp=1ex]{white}%
    \progressbar@progressbar%
  \end{beamercolorbox}%
}
\makeatother


%%% BLOCKS
% block = Aufgabe
\setbeamercolor{block title}{fg=black,bg=blue!30!white} 
\setbeamercolor{block body}{fg=black, bg=blue!3!white}

% alertblock = Definition
\setbeamercolor{block title alerted}{fg=black,bg=red!50!white} 
\setbeamercolor{block body alerted}{fg=black, bg=red!3!white}

% exampleblock = Wiederholung
\setbeamercolor{block title example}{fg=black,bg=green!30!white} 
\setbeamercolor{block body example}{fg=black, bg=green!3!white}



\begin{document}
	%\author{www.oilbat.de}
	%\title{Stochastik}
	\subtitle{}
	\logo{}
	\institute{}
	\date{}
	\subject{}
	\setbeamercovered{transparent}
	\setbeamertemplate{navigation symbols}{}

\addtocounter{framenumber}{-1}
\setbeamercovered{invisible}

\begin{frame}
	\begin{block}{Aufgabe}
		Die Dichte einer stetigen Zufallsvariable $X$ ist gegeben durch
		\begin{align*}
			f(x)=\begin{cases}
			2ax & 0<x<1\\
			3a-ax & 1\leq x <3 \\
			0 & \text{sonst}
			\end{cases},~\text{wobei}~a>0.
		\end{align*}
		\begin{enumerate}
			\item Wie groß ist $a$?
			\item Wie sieht $f(x)$ aus?
			\item Wie groß sind die Wahrscheinlichkeiten
			\begin{itemize}
				\item $P(X=1)$,
				\item $P(0,5<X<2)$,
				\item $P(X<2)$?
			\end{itemize}
		\end{enumerate}
	\end{block}
\end{frame}

\begin{frame}
	\begin{alertblock}{Definition: Dichte}
		Die Funktion $f$ ist eine Dichte, falls 
		\begin{enumerate}
			\item $f(x)\geq 0$ für alle $x$
			\item und $\int_{-\infty}^{+\infty} f(x)dx = 1$.
		\end{enumerate}
	\end{alertblock}
\end{frame}

\begin{frame}
\begin{alertblock}{Definition: Dichte}
	Die Funktion $f$ ist eine Dichte, falls 
	\begin{enumerate}
		\item $f(x)\geq 0$ für alle $x$
		\item und $\int_{-\infty}^{+\infty} f(x)dx = 1$.
	\end{enumerate}
\end{alertblock}
\begin{align*}
1 = \int_{-\infty}^{+\infty} f(x)dx &= \int_{0}^{1} 2ax~dx + \int_{1}^{3} 3a-ax~dx \\
&= 2a\left[ \frac{1}{2}x^2 \right]^1_0 + \left[ 3ax - a\frac{1}{2}x^2 \right]^3_1 \\
&= a + 2a = 3a \\
\Rightarrow a = \frac{1}{3}
\end{align*}
\end{frame}

\begin{frame}
	\begin{block}{Dichte}
		\vspace{-0.5cm}
		\begin{align*}
		f(x)=\begin{cases}
		\frac{2}{3}x & 0<x<1\\
		1-\frac{1}{3}x & 1\leq x <3 \\
		0 & \text{sonst}
		\end{cases}
		\end{align*}
	\end{block}
\end{frame}

\begin{frame}
\begin{block}{Dichte}
	\vspace{-0.5cm}
	\begin{align*}
	f(x)=\begin{cases}
	\frac{2}{3}x & 0<x<1\\
	1-\frac{1}{3}x & 1\leq x <3 \\
	0 & \text{sonst}
	\end{cases}
	\end{align*}
\end{block}
\begin{align*}
&P(X=1) = \int_1^1 f(x) dx = 0 \\
&P(0,5<X<2) = \int_{0,5}^2 f(x)dx = \int_{0,5}^1 \frac{2}{3}x~dx + \int_{1}^2 1-\frac{1}{3}x~dx = \frac{3}{4} \\
&P(X<2) = 1-P(X\geq 2) = 1- \int_{2}^{+\infty}f(x)dx \\&= 1-  \int_{2}^{3}1-\frac{1}{3}x~dx = \frac{5}{6}
\end{align*}
\end{frame}



\end{document}