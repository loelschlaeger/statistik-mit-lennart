\documentclass[t,11pt]{beamer}
\usepackage{tikz}
\usetikzlibrary{calc}
\usepackage[utf8]{inputenc}
\usepackage[ngerman]{babel}
\usepackage{amsmath}
\usepackage{amsfonts}
\usepackage{amssymb}
\usepackage{framed}
\colorlet{LightBlue}{blue}

% Theorem box
\pgfdeclarelayer{background}
\pgfsetlayers{background,main}
\tikzstyle{thmbox} = [inner sep=1em]
\tikzstyle{thmborder} = [draw=blue, fill=none,line width =1.pt, rounded corners=5pt]

\def\parchmentframe#1{
	\tikz{
		\node[thmbox] (A) {#1};
		\begin{pgfonlayer}{background}
			\fill[thmborder] 
			(A.south east) -- (A.south west) -- 
			(A.north west) -- (A.north east) -- cycle;
\end{pgfonlayer}}}

\def\parchmentframetop#1{
	\tikz{
		\node[thmbox] (A) {#1};
		\begin{pgfonlayer}{background}    
			\fill[thmborder]
			(A.south west) -- (A.north west) -- 
			(A.north east) -- (A.south east);
\end{pgfonlayer}}}

\def\parchmentframebottom#1{
	\tikz{
		\node[thmbox] (A) {#1};
		\begin{pgfonlayer}{background}    
			\fill[thmborder]
			(A.north west) -- (A.south west) -- 
			(A.south east) -- (A.north east);
\end{pgfonlayer}}}

\def\parchmentframemiddle#1{
	\tikz{
		\node[thmbox] (A) {#1};
		\begin{pgfonlayer}{background}    
			\fill[thmborder]
			(A.north west) -- (A.south west);
			\fill[thmborder]
			(A.south east) -- (A.north east);
\end{pgfonlayer}}}

\newenvironment<>{myTheorem}[1][]{%
	\def\FrameCommand{\parchmentframe}%
	\def\FirstFrameCommand{\parchmentframetop}%
	\def\LastFrameCommand{\parchmentframebottom}%
	\def\MidFrameCommand{\parchmentframemiddle}%
	\vskip\baselineskip
	\MakeFramed{\FrameRestore}
	\noindent\tikz\node[inner sep=1.2ex, draw=blue, fill=blue!10,
	anchor=west, overlay, line width = 1.pt, rounded corners=4pt] at (0em, 1em) 
	{\color{LightBlue}{Aufgabe\if\relax\detokenize{#1}\relax\else\space (#1)\fi}};\par\nobreak}%
{\endMakeFramed}
%%end theorem box

\definecolor{pbblue}{HTML}{D8D8D8}% filling color for the progress bar
\definecolor{pbgray}{HTML}{F2F2F2}% background color for the progress bar

\useoutertheme{infolines}
\setbeamerfont{footline}{size=\normalsize}
\setbeamersize{text margin left=30pt,text margin right=30pt}
\makeatletter
\setbeamertemplate{footline}
{
	\leavevmode%
	\hbox{%
		\begin{beamercolorbox}[wd=.333333\paperwidth,ht=2.5ex,dp=1ex,center]{title in head/foot}%
			\usebeamerfont{title in head/foot}\insertshorttitle
		\end{beamercolorbox}%
		\begin{beamercolorbox}[wd=.333333\paperwidth,ht=2.5ex,dp=1ex,center]{date in head/foot}%
			%\usebeamerfont{date in head/foot}\insertshortdate{}\hspace*{2em}
			%\insertframenumber\hspace*{2ex} 
		\end{beamercolorbox}
		\begin{beamercolorbox}[wd=.333333\paperwidth,ht=3ex,dp=1ex,center]{author in head/foot}%
			\usebeamerfont{author in head/foot}\insertshortauthor~~%\beamer@ifempty{\insertshortinstitute}{}{(\insertshortinstitute)}
		\end{beamercolorbox}%
	}%
	\vskip0pt%
}
\makeatother


\makeatletter
\def\progressbar@progressbar{} % the progress bar
\newcount\progressbar@tmpcounta% auxiliary counter
\newcount\progressbar@tmpcountb% auxiliary counter
\newdimen\progressbar@pbht %progressbar height
\newdimen\progressbar@pbwd %progressbar width
\newdimen\progressbar@tmpdim % auxiliary dimension

\progressbar@pbwd=\linewidth
\progressbar@pbht=1.5ex

% the progress bar
\def\progressbar@progressbar{%

    \progressbar@tmpcounta=\insertframenumber
    \progressbar@tmpcountb=\inserttotalframenumber
    \progressbar@tmpdim=\progressbar@pbwd
    \multiply\progressbar@tmpdim by \progressbar@tmpcounta
    \divide\progressbar@tmpdim by \progressbar@tmpcountb

  \begin{tikzpicture}[rounded corners=3pt,very thin]

    \shade[top color=pbgray!20,bottom color=pbgray!20,middle color=pbgray!50]
      (0pt, 0pt) rectangle ++ (\progressbar@pbwd, \progressbar@pbht);

      \shade[draw=pbblue,top color=pbblue!50,bottom color=pbblue!50,middle color=pbblue] %
        (0pt, 0pt) rectangle ++ (\progressbar@tmpdim, \progressbar@pbht);

    \draw[color=normal text.fg!50]  
      (0pt, 0pt) rectangle (\progressbar@pbwd, \progressbar@pbht) 
        node[pos=0.5,color=normal text.fg] {die Aufgabe ist zu \textnormal{%
             \pgfmathparse{\insertframenumber*100/\inserttotalframenumber}%
             \pgfmathprintnumber[fixed,precision=0]{\pgfmathresult}\,\% gelöst%
        }%
    };
  \end{tikzpicture}%
}

\addtobeamertemplate{headline}{}
{%
  \begin{beamercolorbox}[wd=\paperwidth,ht=4ex,center,dp=1ex]{white}%
    \progressbar@progressbar%
  \end{beamercolorbox}%
}
\makeatother

\begin{document}
	\author{www.oilbat.de}
	%\title{Stochastik}
	\subtitle{}
	\logo{}
	\institute{}
	\date{}
	\subject{}
	\setbeamercovered{transparent}
	\setbeamertemplate{navigation symbols}{}

\addtocounter{framenumber}{-1}
\setbeamercovered{invisible}



\begin{frame}
\begin{myTheorem}
	Seien $X_1,\dots,X_n$ unabhängige, $\mathcal{N}(\mu,\sigma^2)$-verteilte Zufallsvariablen. Bestimme den Maximum Likelihood-Schätzer für $\mu$ und $\sigma$.
\end{myTheorem}
\end{frame}

\begin{frame}
\begin{myTheorem}
	Seien $X_1,\dots,X_n$ unabhängige, $\mathcal{N}(\mu,\sigma^2)$-verteilte Zufallsvariablen. Bestimme den Maximum Likelihood-Schätzer für $\mu$ und $\sigma$.
\end{myTheorem}
\begin{enumerate}
	\item Aufstellen der ML-Funktion: $L(\theta)=\prod_{i=1}^{n}f_{X_i}(\theta)$
	\item Logarithmierung: $\ln(L(\theta))=\sum_{i=1}^{n}\ln(f_{X_i}(\theta))$
	\item Bestimmung der Ableitung: $\frac{d}{d\theta}\ln(L(\theta))$
	\item Ableitung gleich Null setzen und nach dem gesuchten Parameter auflösen
\end{enumerate}
\end{frame}

\begin{frame}
\begin{myTheorem}
	Seien $X_1,\dots,X_n$ unabhängige, $\mathcal{N}(\mu,\sigma^2)$-verteilte Zufallsvariablen. Bestimme den Maximum Likelihood-Schätzer für $\mu$ und $\sigma$.
\end{myTheorem}
\begin{enumerate}
	\item Aufstellen der ML-Funktion: $L(\mu,\sigma)=\prod_{i=1}^{n}f_{X_i}(\mu,\sigma)$
\end{enumerate}


\end{frame}

\begin{frame}
\begin{myTheorem}
	Seien $X_1,\dots,X_n$ unabhängige, $\mathcal{N}(\mu,\sigma^2)$-verteilte Zufallsvariablen. Bestimme den Maximum Likelihood-Schätzer für $\mu$ und $\sigma$.
\end{myTheorem}
\begin{enumerate}
	\item Aufstellen der ML-Funktion: $L(\mu,\sigma)=\prod_{i=1}^{n}f_{X_i}(\mu,\sigma)$
\end{enumerate}

\begin{align*}
L(\mu,\sigma) &= \frac{1}{\sqrt{2\pi\sigma^2}}e^{-\frac{(X_1-\mu)^2}{2\sigma^2}}\cdot \dots \cdot \frac{1}{\sqrt{2\pi\sigma^2}}e^{-\frac{(X_n-\mu)^2}{2\sigma^2}} \\
\end{align*}

\end{frame}

\begin{frame}
\begin{myTheorem}
	Seien $X_1,\dots,X_n$ unabhängige, $\mathcal{N}(\mu,\sigma^2)$-verteilte Zufallsvariablen. Bestimme den Maximum Likelihood-Schätzer für $\mu$ und $\sigma$.
\end{myTheorem}
\begin{enumerate}
	\item Aufstellen der ML-Funktion: $L(\mu,\sigma)=\prod_{i=1}^{n}f_{X_i}(\mu,\sigma)$
\end{enumerate}

\begin{align*}
L(\mu,\sigma) &= \frac{1}{\sqrt{2\pi\sigma^2}}e^{-\frac{(X_1-\mu)^2}{2\sigma^2}}\cdot \dots \cdot \frac{1}{\sqrt{2\pi\sigma^2}}e^{-\frac{(X_n-\mu)^2}{2\sigma^2}} \\
&= \prod_{i=1}^{n} \frac{1}{\sqrt{2\pi\sigma^2}}e^{-\frac{(X_i-\mu)^2}{2\sigma^2}}
\end{align*}

\end{frame}


\begin{frame}
\begin{enumerate}
	\item $L(\mu,\sigma)=\prod_{i=1}^{n}f_{X_i}(\mu,\sigma)=\prod_{i=1}^{n} \frac{1}{\sqrt{2\pi\sigma^2}}e^{-\frac{(X_i-\mu)^2}{2\sigma^2}}$
	\item Logarithmierung: $\ln(L(\mu,\sigma))=\sum_{i=1}^{n}\ln(f_{X_i}(\mu,\sigma))$
\end{enumerate}


\end{frame}

\begin{frame}
\begin{enumerate}
	\item $L(\mu,\sigma)=\prod_{i=1}^{n}f_{X_i}(\mu,\sigma)=\prod_{i=1}^{n} \frac{1}{\sqrt{2\pi\sigma^2}}e^{-\frac{(X_i-\mu)^2}{2\sigma^2}}$
	\item Logarithmierung: $\ln(L(\mu,\sigma))=\sum_{i=1}^{n}\ln(f_{X_i}(\mu,\sigma))$
\end{enumerate}

\begin{align*}
\ln(L(\mu,\sigma)) &=  \sum_{i=1}^{n} \ln \left( \frac{1}{\sqrt{2\pi\sigma^2}}e^{-\frac{(X_i-\mu)^2}{2\sigma^2}} \right) 
\end{align*}

\end{frame}

\begin{frame}
\begin{enumerate}
	\item $L(\mu,\sigma)=\prod_{i=1}^{n}f_{X_i}(\mu,\sigma)=\prod_{i=1}^{n} \frac{1}{\sqrt{2\pi\sigma^2}}e^{-\frac{(X_i-\mu)^2}{2\sigma^2}}$
	\item Logarithmierung: $\ln(L(\mu,\sigma))=\sum_{i=1}^{n}\ln(f_{X_i}(\mu,\sigma))$
\end{enumerate}

\begin{align*}
\ln(L(\mu,\sigma)) &=  \sum_{i=1}^{n} \ln \left( \frac{1}{\sqrt{2\pi\sigma^2}}e^{-\frac{(X_i-\mu)^2}{2\sigma^2}} \right) \\ &= \sum_{i=1}^{n}\left[ \ln \left( \frac{1}{\sqrt{2\pi\sigma^2}}\right) -\frac{(X_i-\mu)^2}{2\sigma^2}\right] 
\end{align*}

\end{frame}

\begin{frame}
\begin{enumerate}
	\item $L(\mu,\sigma)=\prod_{i=1}^{n}f_{X_i}(\mu,\sigma)=\prod_{i=1}^{n} \frac{1}{\sqrt{2\pi\sigma^2}}e^{-\frac{(X_i-\mu)^2}{2\sigma^2}}$
	\item Logarithmierung: $\ln(L(\mu,\sigma))=\sum_{i=1}^{n}\ln(f_{X_i}(\mu,\sigma))$
\end{enumerate}

\begin{align*}
\ln(L(\mu,\sigma)) &=  \sum_{i=1}^{n} \ln \left( \frac{1}{\sqrt{2\pi\sigma^2}}e^{-\frac{(X_i-\mu)^2}{2\sigma^2}} \right) \\ &= \sum_{i=1}^{n}\left[ \ln \left( \frac{1}{\sqrt{2\pi\sigma^2}}\right) -\frac{(X_i-\mu)^2}{2\sigma^2}\right] \\ &=  \sum_{i=1}^{n}\left[ \ln 1 -\ln \sqrt{2\pi\sigma^2} -\frac{(X_i-\mu)^2}{2\sigma^2}\right] 
\end{align*}

\end{frame}

\begin{frame}
\begin{enumerate}
	\item $L(\mu,\sigma)=\prod_{i=1}^{n}f_{X_i}(\mu,\sigma)=\prod_{i=1}^{n} \frac{1}{\sqrt{2\pi\sigma^2}}e^{-\frac{(X_i-\mu)^2}{2\sigma^2}}$
	\item Logarithmierung: $\ln(L(\mu,\sigma))=\sum_{i=1}^{n}\ln(f_{X_i}(\mu,\sigma))$
\end{enumerate}

\begin{align*}
\ln(L(\mu,\sigma)) &=  \sum_{i=1}^{n} \ln \left( \frac{1}{\sqrt{2\pi\sigma^2}}e^{-\frac{(X_i-\mu)^2}{2\sigma^2}} \right) \\ &= \sum_{i=1}^{n}\left[ \ln \left( \frac{1}{\sqrt{2\pi\sigma^2}}\right) -\frac{(X_i-\mu)^2}{2\sigma^2}\right] \\ &=  \sum_{i=1}^{n}\left[ \ln 1 -\ln \sqrt{2\pi\sigma^2} -\frac{(X_i-\mu)^2}{2\sigma^2}\right] \\ &=  \sum_{i=1}^{n}\left[-\ln \sqrt{2\pi} - \ln \sigma -\frac{(X_i-\mu)^2}{2\sigma^2}\right]
\end{align*}

\end{frame}

\begin{frame}
\begin{enumerate}
	\item $L(\mu,\sigma)=\prod_{i=1}^{n}f_{X_i}(\mu,\sigma)=\prod_{i=1}^{n} \frac{1}{\sqrt{2\pi\sigma^2}}e^{-\frac{(X_i-\mu)^2}{2\sigma^2}}$
	\item $\ln(L(\mu,\sigma))=\sum_{i=1}^{n}\left[-\ln \sqrt{2\pi} - \ln \sigma -\frac{(X_i-\mu)^2}{2\sigma^2}\right]$
	\item Bestimmung der Ableitung
\end{enumerate}

	
\end{frame}

\begin{frame}
\begin{enumerate}
	\item $L(\mu,\sigma)=\prod_{i=1}^{n}f_{X_i}(\mu,\sigma)=\prod_{i=1}^{n} \frac{1}{\sqrt{2\pi\sigma^2}}e^{-\frac{(X_i-\mu)^2}{2\sigma^2}}$
	\item $\ln(L(\mu,\sigma))=\sum_{i=1}^{n}\left[-\ln \sqrt{2\pi} - \ln \sigma -\frac{(X_i-\mu)^2}{2\sigma^2}\right]$
	\item Bestimmung der Ableitung
\end{enumerate}

\begin{align*}
&\frac{\partial \ln(L(\mu,\sigma)) }{\partial \mu}= \sum_{i=1}^{n} -(-1) \frac{2(X_i-\mu)}{2\sigma^2} = \sum_{i=1}^{n} \frac{(X_i-\mu)}{\sigma^2} \\
\end{align*}

\end{frame}

\begin{frame}
\begin{enumerate}
	\item $L(\mu,\sigma)=\prod_{i=1}^{n}f_{X_i}(\mu,\sigma)=\prod_{i=1}^{n} \frac{1}{\sqrt{2\pi\sigma^2}}e^{-\frac{(X_i-\mu)^2}{2\sigma^2}}$
	\item $\ln(L(\mu,\sigma))=\sum_{i=1}^{n}\left[-\ln \sqrt{2\pi} - \ln \sigma -\frac{(X_i-\mu)^2}{2\sigma^2}\right]$
	\item Bestimmung der Ableitung
\end{enumerate}

\begin{align*}
&\frac{\partial \ln(L(\mu,\sigma)) }{\partial \mu}= \sum_{i=1}^{n} -(-1) \frac{2(X_i-\mu)}{2\sigma^2} = \sum_{i=1}^{n} \frac{(X_i-\mu)}{\sigma^2} \\
&\frac{\partial \ln(L(\mu,\sigma)) }{\partial \sigma} = \sum_{i=1}^{n} \left[ -\frac{1}{\sigma} -(-2)\frac{(X_i-\mu)^2}{2\sigma^3}  \right]  = \sum_{i=1}^{n} \left[ -\frac{1}{\sigma} +\frac{(X_i-\mu)^2}{\sigma^3} \right]
\end{align*}

\end{frame}

\begin{frame}
\begin{enumerate}
	\item $L(\mu,\sigma)=\prod_{i=1}^{n}f_{X_i}(\mu,\sigma)=\prod_{i=1}^{n} \frac{1}{\sqrt{2\pi\sigma^2}}e^{-\frac{(X_i-\mu)^2}{2\sigma^2}}$
	\item $\ln(L(\mu,\sigma))=\sum_{i=1}^{n}\left[-\ln \sqrt{2\pi} - \ln \sigma -\frac{(X_i-\mu)^2}{2\sigma^2}\right]$
	\item Bestimmung der Ableitung
	\item Ableitung gleich Null setzen und nach dem gesuchten Parameter auflösen
\end{enumerate}

\begin{align*}
&\frac{\partial \ln(L(\mu,\sigma)) }{\partial \mu} = \sum_{i=1}^{n} \frac{(X_i-\mu)}{\sigma^2} = 0\\
&\frac{\partial \ln(L(\mu,\sigma)) }{\partial \sigma} = \sum_{i=1}^{n} \left[ -\frac{1}{\sigma} +\frac{(X_i-\mu)^2}{\sigma^3} \right] = 0
\end{align*}

\end{frame}

\begin{frame}
	\begin{align*}
	&\frac{\partial \ln(L(\mu,\sigma)) }{\partial \mu} = \sum_{i=1}^{n} \frac{(X_i-\mu)}{\sigma^2} = 0\\
	&\frac{\partial \ln(L(\mu,\sigma)) }{\partial \sigma} = \sum_{i=1}^{n} \left[ -\frac{1}{\sigma} +\frac{(X_i-\mu)^2}{\sigma^3} \right] = 0
	\end{align*}

\end{frame}

\begin{frame}
\begin{align*}
&\frac{\partial \ln(L(\mu,\sigma)) }{\partial \mu} = \sum_{i=1}^{n} \frac{(X_i-\mu)}{\sigma^2} = 0\\
&\frac{\partial \ln(L(\mu,\sigma)) }{\partial \sigma} = \sum_{i=1}^{n} \left[ -\frac{1}{\sigma} +\frac{(X_i-\mu)^2}{\sigma^3} \right] = 0
\end{align*}

\begin{align*}
(1) &\Rightarrow \sum_{i=1}^{n}(X_i-\mu)=0 \Rightarrow \mu= \frac{1}{n}\sum_{i=1}^{n} X_i =\bar{X}\\
\end{align*}
\end{frame}

\begin{frame}
\begin{align*}
&\frac{\partial \ln(L(\mu,\sigma)) }{\partial \mu} = \sum_{i=1}^{n} \frac{(X_i-\mu)}{\sigma^2} = 0\\
&\frac{\partial \ln(L(\mu,\sigma)) }{\partial \sigma} = \sum_{i=1}^{n} \left[ -\frac{1}{\sigma} +\frac{(X_i-\mu)^2}{\sigma^3} \right] = 0
\end{align*}

\begin{align*}
(1) &\Rightarrow \sum_{i=1}^{n}(X_i-\mu)=0 \Rightarrow \mu= \frac{1}{n}\sum_{i=1}^{n} X_i =\bar{X}\\
(2) &\Rightarrow -\frac{n}{\sigma} + \sum_{i=1}^{n} \frac{(X_i-\mu)^2}{\sigma^3} =0 \Rightarrow \frac{1}{\sigma^3} \sum_{i=1}^{n}(X_i-\mu)^2 = \frac{n}{\sigma}\\
\end{align*}
\end{frame}

\begin{frame}
\begin{align*}
&\frac{\partial \ln(L(\mu,\sigma)) }{\partial \mu} = \sum_{i=1}^{n} \frac{(X_i-\mu)}{\sigma^2} = 0\\
&\frac{\partial \ln(L(\mu,\sigma)) }{\partial \sigma} = \sum_{i=1}^{n} \left[ -\frac{1}{\sigma} +\frac{(X_i-\mu)^2}{\sigma^3} \right] = 0
\end{align*}

\begin{align*}
(1) &\Rightarrow \sum_{i=1}^{n}(X_i-\mu)=0 \Rightarrow \mu= \frac{1}{n}\sum_{i=1}^{n} X_i =\bar{X}\\
(2) &\Rightarrow -\frac{n}{\sigma} + \sum_{i=1}^{n} \frac{(X_i-\mu)^2}{\sigma^3} =0 \Rightarrow \frac{1}{\sigma^3} \sum_{i=1}^{n}(X_i-\mu)^2 = \frac{n}{\sigma}\\
&\Rightarrow \sigma = \sqrt{\frac{1}{n}  \sum_{i=1}^{n}(X_i-\mu)^2 } \overset{\mu = \bar{X}}{=} \sqrt{\frac{1}{n}  \sum_{i=1}^{n}(X_i-\bar{X})^2 }
\end{align*}
\end{frame}


\begin{frame}
Maximum Likelihood-Schätzer für $\mu$ und $\sigma^2$
\begin{align*}
	& \hat{\mu}_{ML} = \bar{X} \\
	& \hat{\sigma}_{ML} = \sqrt{\frac{1}{n}  \sum_{i=1}^{n}(X_i-\bar{X})^2 }
\end{align*}
\end{frame}

\begin{frame}
Maximum Likelihood-Schätzer für $\mu$ und $\sigma^2$
\begin{align*}
& \hat{\mu}_{ML} = \bar{X} \\
& \hat{\sigma}_{ML} = \sqrt{\frac{1}{n}  \sum_{i=1}^{n}(X_i-\bar{X})^2 }
\end{align*}
\begin{myTheorem}
	Es liegen nun die folgenden Realisationen der Zufallsvariablen vor:
	$$x_1=4,x_2=5,x_3=1,x_4=2,x_5=7,x_6=11,x_7=2,x_8=8$$
	Bestimme die Maximum Likelihood Schätzwerte $\hat{\mu}_{ML}$ und $ \hat{\sigma}_{ML} $ für $\mu$ und $\sigma$.
\end{myTheorem}
\end{frame}

\begin{frame}
\begin{myTheorem}
	Es liegen nun die folgenden Realisationen der Zufallsvariablen vor:
	$$x_1=4,x_2=5,x_3=1,x_4=2,x_5=7,x_6=11,x_7=2,x_8=8$$
	Bestimme die Maximum Likelihood Schätzwerte $\hat{\mu}_{ML}$ und $ \hat{\sigma}_{ML} $ für $\mu$ und $\sigma$.
\end{myTheorem}
\begin{align*}
& \hat{\mu}_{ML} = \frac{1}{8}\sum_{i=1}^{8}x_i = 5 \\
& \hat{\sigma}_{ML} = \sqrt{\frac{1}{8}  \sum_{i=1}^{8}(x_i-5)^2 } = 3.24037
\end{align*}
\end{frame}



\end{document}