\documentclass[t,11pt]{beamer}
\usepackage{tikz}
\usepackage{pgfplots}
\usetikzlibrary{calc}
\usepackage[utf8]{inputenc}
\usepackage[ngerman]{babel}
\usepackage{amsmath,amsfonts,amssymb}
\usepackage{framed}
\usecolortheme{orchid}
\usepackage{etoolbox}
\usepackage{tikz}
\usetikzlibrary{arrows,automata}
\useinnertheme[shadow=true]{rounded}

%%% PROGRESSBAR
\definecolor{pbblue}{HTML}{D8D8D8}% filling color for the progress bar
\definecolor{pbgray}{HTML}{F2F2F2}% background color for the progress bar
\useoutertheme{infolines}
\setbeamerfont{footline}{size=\normalsize}
\setbeamersize{text margin left=30pt,text margin right=30pt}
\makeatletter
\setbeamertemplate{footline}
{
	\leavevmode%
	\hbox{%
		\begin{beamercolorbox}[wd=.333333\paperwidth,ht=2.5ex,dp=1ex,center]{title in head/foot}%
			\usebeamerfont{title in head/foot}\insertshorttitle
		\end{beamercolorbox}%
		\begin{beamercolorbox}[wd=.333333\paperwidth,ht=2.5ex,dp=1ex,center]{date in head/foot}%
			%\usebeamerfont{date in head/foot}\insertshortdate{}\hspace*{2em}
			%\insertframenumber\hspace*{2ex} 
		\end{beamercolorbox}
		\begin{beamercolorbox}[wd=.333333\paperwidth,ht=3ex,dp=1ex,center]{author in head/foot}%
			\usebeamerfont{author in head/foot}\insertshortauthor~~%\beamer@ifempty{\insertshortinstitute}{}{(\insertshortinstitute)}
		\end{beamercolorbox}%
	}%
	\vskip0pt%
}
\makeatother
\makeatletter
\def\progressbar@progressbar{} % the progress bar
\newcount\progressbar@tmpcounta% auxiliary counter
\newcount\progressbar@tmpcountb% auxiliary counter
\newdimen\progressbar@pbht %progressbar height
\newdimen\progressbar@pbwd %progressbar width
\newdimen\progressbar@tmpdim % auxiliary dimension
\progressbar@pbwd=\linewidth
\progressbar@pbht=1.5ex
\def\progressbar@progressbar{%
    \progressbar@tmpcounta=\insertframenumber
    \progressbar@tmpcountb=\inserttotalframenumber
    \progressbar@tmpdim=\progressbar@pbwd
    \multiply\progressbar@tmpdim by \progressbar@tmpcounta
    \divide\progressbar@tmpdim by \progressbar@tmpcountb
  \begin{tikzpicture}[rounded corners=3pt,very thin]
    \shade[top color=pbgray!20,bottom color=pbgray!20,middle color=pbgray!50]
      (0pt, 0pt) rectangle ++ (\progressbar@pbwd, \progressbar@pbht);
      \shade[draw=pbblue,top color=pbblue!50,bottom color=pbblue!50,middle color=pbblue] %
        (0pt, 0pt) rectangle ++ (\progressbar@tmpdim, \progressbar@pbht);
    \draw[color=normal text.fg!50]  
      (0pt, 0pt) rectangle (\progressbar@pbwd, \progressbar@pbht) 
        node[pos=0.5,color=normal text.fg] {\textnormal{%
             \pgfmathparse{\insertframenumber*100/\inserttotalframenumber}%
             \pgfmathprintnumber[fixed,precision=0]{\pgfmathresult}\,\%%
        }%
    };
  \end{tikzpicture}%
}
\addtobeamertemplate{headline}{}
{%
  \begin{beamercolorbox}[wd=\paperwidth,ht=4ex,center,dp=1ex]{white}%
    \progressbar@progressbar%
  \end{beamercolorbox}%
}
\makeatother


%%% BLOCKS
% block = Aufgabe
\setbeamercolor{block title}{fg=black,bg=blue!30!white} 
\setbeamercolor{block body}{fg=black, bg=blue!3!white}

% alertblock = Definition
\setbeamercolor{block title alerted}{fg=black,bg=red!50!white} 
\setbeamercolor{block body alerted}{fg=black, bg=red!3!white}

% exampleblock = Wiederholung
\setbeamercolor{block title example}{fg=black,bg=green!30!white} 
\setbeamercolor{block body example}{fg=black, bg=green!3!white}



\begin{document}
	%\author{www.oilbat.de}
	%\title{Stochastik}
	\subtitle{}
	\logo{}
	\institute{}
	\date{}
	\subject{}
	\setbeamercovered{transparent}
	\setbeamertemplate{navigation symbols}{}

\addtocounter{framenumber}{-1}
\setbeamercovered{invisible}

\begin{frame}
	\begin{block}{Aufgabe}
		Betrachte die Markov-Kette mit $P=\begin{bmatrix}
		1-p & \frac{p}{2} & \frac{p}{2} \\
		1 & 0 & 0 \\
		1 & 0 & 0 \\
		\end{bmatrix}$, $p\in[0,1]$.
		\begin{enumerate}
			\item Bestimme alle stationären Verteilungen der Markov-Kette.
			\item Für welche $p$ ist die Markov-Kette irreduzibel? 
			\item Für welche $p$ ist sie aperiodisch?
		\end{enumerate}
	\end{block}
\end{frame}

\begin{frame}
\begin{block}{Aufgabe}
	Betrachte die Markov-Kette mit $P=\begin{bmatrix}
	1-p & \frac{p}{2} & \frac{p}{2} \\
	1 & 0 & 0 \\
	1 & 0 & 0 \\
	\end{bmatrix}$, $p\in[0,1]$.
	\begin{enumerate}
		\item Bestimme alle stationären Verteilungen der Markov-Kette.
		\item Für welche $p$ ist die Markov-Kette irreduzibel? 
		\item Für welche $p$ ist sie aperiodisch?
	\end{enumerate}
\end{block}
\begin{center}
	\begin{tikzpicture}[->,>=stealth',shorten >=1pt,auto,node distance=2.8cm,
	semithick]
	\tikzstyle{every state}=[fill=blue!20,draw=black,text=black]
	\node[state] (A)                    {$1$};
	\node[state] (B) [below left of=A] {$2$};
	\node[state] (C) [below right of=A] {$3$};
	
	\path (A) edge [loop above] node [left,pos=0.3] {$1-p$} (A);
	\path (A) edge [bend left] node [pos=0.7] {$p/2$} (B);
	\path (A) edge [bend left] node {$p/2$} (C);
	\path (B) edge [bend left] node {$1$} (A);
	\path (C) edge [bend left] node [pos=0.7] {$1$} (A);
	\end{tikzpicture}
\end{center}
\end{frame}

\begin{frame}
	\begin{alertblock}{Definition: Stationäre Verteilung}
		Die Verteilung $\pi$ ist stationär bezüglich der Markov-Kette mit Übergangsmatrix $P$, falls $$\pi P = \pi.$$
	\end{alertblock}
\end{frame}

\begin{frame}
\begin{alertblock}{Definition: Stationäre Verteilung}
	Die Verteilung $\pi$ ist stationär bezüglich der Markov-Kette mit Übergangsmatrix $P$, falls $$\pi P = \pi.$$
\end{alertblock}
\begin{align*}
&\pi P = \pi \\
\end{align*}
\end{frame}

\begin{frame}
\begin{alertblock}{Definition: Stationäre Verteilung}
	Die Verteilung $\pi$ ist stationär bezüglich der Markov-Kette mit Übergangsmatrix $P$, falls $$\pi P = \pi.$$
\end{alertblock}
\begin{align*}
&\pi P = \pi \\
\Longleftrightarrow & \begin{bmatrix}
\pi_1 & \pi_2 & \pi_3 
\end{bmatrix} \begin{bmatrix}
1-p & \frac{p}{2} & \frac{p}{2} \\
1 & 0 & 0 \\
1 & 0 & 0 \\
\end{bmatrix} = \begin{bmatrix}
\pi_1 & \pi_2 & \pi_3 
\end{bmatrix} 
\end{align*}
\end{frame}
\begin{frame}
\begin{alertblock}{Definition: Stationäre Verteilung}
	Die Verteilung $\pi$ ist stationär bezüglich der Markov-Kette mit Übergangsmatrix $P$, falls $$\pi P = \pi.$$
\end{alertblock}
\begin{align*}
&\pi P = \pi \\
\Longleftrightarrow & \begin{bmatrix}
\pi_1 & \pi_2 & \pi_3 
\end{bmatrix} \begin{bmatrix}
1-p & \frac{p}{2} & \frac{p}{2} \\
1 & 0 & 0 \\
1 & 0 & 0 \\
\end{bmatrix} = \begin{bmatrix}
\pi_1 & \pi_2 & \pi_3 
\end{bmatrix} \\
\Longleftrightarrow
&\begin{matrix}
(1-p) \pi_1 + \pi_2 + \pi_3 &= \pi_1 \\
\frac{p}{2}\pi_1 &= \pi_2 \\
\frac{p}{2}\pi_1 &= \pi_3 \\
\end{matrix}
\end{align*}
\end{frame}
\begin{frame}
\begin{alertblock}{Definition: Stationäre Verteilung}
	Die Verteilung $\pi$ ist stationär bezüglich der Markov-Kette mit Übergangsmatrix $P$, falls $$\pi P = \pi.$$
\end{alertblock}
\begin{align*}
&\pi P = \pi \\
\Longleftrightarrow & \begin{bmatrix}
\pi_1 & \pi_2 & \pi_3 
\end{bmatrix} \begin{bmatrix}
1-p & \frac{p}{2} & \frac{p}{2} \\
1 & 0 & 0 \\
1 & 0 & 0 \\
\end{bmatrix} = \begin{bmatrix}
\pi_1 & \pi_2 & \pi_3 
\end{bmatrix} \\
\Longleftrightarrow
&\begin{matrix}
(1-p) \pi_1 + \pi_2 + \pi_3 &= \pi_1 \\
\frac{p}{2}\pi_1 &= \pi_2 \\
\frac{p}{2}\pi_1 &= \pi_3 \\
\end{matrix}
\end{align*}
Und als vierte Gleichung haben wir $\pi_1+\pi_2+\pi_3 = 1.$
\end{frame}

\begin{frame}
	Unser Gleichungssystem:
	\begin{align*}
		\begin{matrix}
		(1) & (1-p) \pi_1 + \pi_2 + \pi_3 &= \pi_1 \\
		(2) & \frac{p}{2}\pi_1 &= \pi_2 \\
		(3) & \frac{p}{2}\pi_1 &= \pi_3 \\
		(4) & \pi_1+\pi_2+\pi_3 &= 1 \\
		\end{matrix}
	\end{align*}
\end{frame}

\begin{frame}
Unser Gleichungssystem:
\begin{align*}
\begin{matrix}
(1) & (1-p) \pi_1 + \pi_2 + \pi_3 &= \pi_1 \\
(2) & \frac{p}{2}\pi_1 &= \pi_2 \\
(3) & \frac{p}{2}\pi_1 &= \pi_3 \\
(4) & \pi_1+\pi_2+\pi_3 &= 1 \\
\end{matrix}
\end{align*}
\hrule
\begin{align*}
\begin{matrix}
(1) & \Longrightarrow & \pi_1-p\pi_1+\pi_2+\pi_3 &=& \pi_1 \\
& \Longrightarrow & \pi_1+\pi_2+\pi_3 &=& \pi_1 + p \pi_1 \\
(4) & \Longrightarrow & 1 &=& (1+p)\pi_1 \\
& \Longrightarrow & \pi_1 &=& \frac{1}{1+p} \\
(2) & \Longrightarrow & \pi_2 &=& \frac{p}{2+2p} \\
(3) & \Longrightarrow & \pi_3 &=& \frac{p}{2+2p} \\
\end{matrix}
\end{align*}
\end{frame}

\begin{frame}
Unser Gleichungssystem:
\begin{align*}
\begin{matrix}
(1) & (1-p) \pi_1 + \pi_2 + \pi_3 &= \pi_1 \\
(2) & \frac{p}{2}\pi_1 &= \pi_2 \\
(3) & \frac{p}{2}\pi_1 &= \pi_3 \\
(4) & \pi_1+\pi_2+\pi_3 &= 1 \\
\end{matrix}
\end{align*}
\hrule
\begin{align*}
\begin{matrix}
(1) & \Longrightarrow & \pi_1-p\pi_1+\pi_2+\pi_3 &=& \pi_1 \\
& \Longrightarrow & \pi_1+\pi_2+\pi_3 &=& \pi_1 + p \pi_1 \\
(4) & \Longrightarrow & 1 &=& (1+p)\pi_1 \\
& \Longrightarrow & \pi_1 &=& \frac{1}{1+p} \\
(2) & \Longrightarrow & \pi_2 &=& \frac{p}{2+2p} \\
(3) & \Longrightarrow & \pi_3 &=& \frac{p}{2+2p} \\
\end{matrix}
\end{align*}
\hrule
\vspace{0.2cm}

Stationäre Verteilungen: $\pi=\begin{bmatrix}
\pi_1 & \pi_2 & \pi_3 
\end{bmatrix} = \begin{bmatrix}
\frac{1}{1+p} & \frac{p}{2+2p} & \frac{p}{2+2p} 
\end{bmatrix}$, $p\in[0,1]$.
\end{frame}

\begin{frame}
	\begin{block}{Aufgabe}
		Betrachte die Markov-Kette mit $P=\begin{bmatrix}
		1-p & \frac{p}{2} & \frac{p}{2} \\
		1 & 0 & 0 \\
		1 & 0 & 0 \\
		\end{bmatrix}$, $p\in[0,1]$.
		\begin{enumerate}
			\item Bestimme alle stationären Verteilungen der Markov-Kette.
			\item Für welche $p$ ist die Markov-Kette irreduzibel? 
			\item Für welche $p$ ist sie aperiodisch?
		\end{enumerate}
	\end{block}
\end{frame}

\begin{frame}
\begin{block}{Aufgabe}
	Betrachte die Markov-Kette mit $P=\begin{bmatrix}
	1-p & \frac{p}{2} & \frac{p}{2} \\
	1 & 0 & 0 \\
	1 & 0 & 0 \\
	\end{bmatrix}$, $p\in[0,1]$.
	\begin{enumerate}
		\item Bestimme alle stationären Verteilungen der Markov-Kette.
		\item Für welche $p$ ist die Markov-Kette irreduzibel? 
		\item Für welche $p$ ist sie aperiodisch?
	\end{enumerate}
\end{block}
\begin{alertblock}{Definition: Irreduzible und aperiodische Markov-Kette}
	Eine Markov-Kette mit Übergangsmatrix $P$ heißt irreduzibel, falls es für alle $i,j$ ein $n\in\mathbb{N}$ gibt, sodass $P_{i,j}^n>0.$
	
	Eine Markov-Kette mit Übergangsmatrix $P$ heißt aperiodisch, falls $\text{ggT}\{n\in\mathbb{N}:P^n_{i,i}>0\}=1$ für alle $i$.
\end{alertblock}
\end{frame}

\begin{frame}
	\begin{center}
		\begin{tikzpicture}[->,>=stealth',shorten >=1pt,auto,node distance=2.8cm,
		semithick]
		\tikzstyle{every state}=[fill=blue!20,draw=black,text=black]
		\node[state] (A)                    {$1$};
		\node[state] (B) [below left of=A] {$2$};
		\node[state] (C) [below right of=A] {$3$};
		
		\path (A) edge [loop above] node [left,pos=0.3] {$1-p$} (A);
		\path (A) edge [bend left] node [pos=0.7] {$p/2$} (B);
		\path (A) edge [bend left] node {$p/2$} (C);
		\path (B) edge [bend left] node {$1$} (A);
		\path (C) edge [bend left] node [pos=0.7] {$1$} (A);
		\end{tikzpicture}
	\end{center}
\end{frame}

\begin{frame}
\begin{center}
	\begin{tikzpicture}[->,>=stealth',shorten >=1pt,auto,node distance=2.8cm,
	semithick]
	\tikzstyle{every state}=[fill=blue!20,draw=black,text=black]
	\node[state] (A)                    {$1$};
	\node[state] (B) [below left of=A] {$2$};
	\node[state] (C) [below right of=A] {$3$};
	
	\path (A) edge [loop above] node [left,pos=0.3] {$1-p$} (A);
	\path (A) edge [bend left] node [pos=0.7] {$p/2$} (B);
	\path (A) edge [bend left] node {$p/2$} (C);
	\path (B) edge [bend left] node {$1$} (A);
	\path (C) edge [bend left] node [pos=0.7] {$1$} (A);
	\end{tikzpicture}
\end{center}
\begin{itemize}
	\item Für $p\in (0,1]$ ist die Markov-Kette irreduzible.
\end{itemize}
\end{frame}

\begin{frame}
\begin{center}
	\begin{tikzpicture}[->,>=stealth',shorten >=1pt,auto,node distance=2.8cm,
	semithick]
	\tikzstyle{every state}=[fill=blue!20,draw=black,text=black]
	\node[state] (A)                    {$1$};
	\node[state] (B) [below left of=A] {$2$};
	\node[state] (C) [below right of=A] {$3$};
	
	\path (A) edge [loop above] node [left,pos=0.3] {$1-p$} (A);
	\path (A) edge [bend left] node [pos=0.7] {$p/2$} (B);
	\path (A) edge [bend left] node {$p/2$} (C);
	\path (B) edge [bend left] node {$1$} (A);
	\path (C) edge [bend left] node [pos=0.7] {$1$} (A);
	\end{tikzpicture}
\end{center}
\begin{itemize}
	\item Für $p\in (0,1]$ ist die Markov-Kette irreduzible.
	\item Für $p\in [0,1)$ ist die Markov-Kette aperiodisch.
\end{itemize}
\end{frame}



\end{document}