\documentclass[t,11pt]{beamer}
\usepackage{tikz}
\usepackage{pgfplots}
\usetikzlibrary{calc}
\usepackage[utf8]{inputenc}
\usepackage[ngerman]{babel}
\usepackage{amsmath,amsfonts,amssymb}
\usepackage{framed}
\usecolortheme{orchid}
\usepackage{etoolbox}
\useinnertheme[shadow=true]{rounded}



%%% PROGRESSBAR
\definecolor{pbblue}{HTML}{D8D8D8}% filling color for the progress bar
\definecolor{pbgray}{HTML}{F2F2F2}% background color for the progress bar
\useoutertheme{infolines}
\setbeamerfont{footline}{size=\normalsize}
\setbeamersize{text margin left=30pt,text margin right=30pt}
\makeatletter
\setbeamertemplate{footline}
{
	\leavevmode%
	\hbox{%
		\begin{beamercolorbox}[wd=.333333\paperwidth,ht=2.5ex,dp=1ex,center]{title in head/foot}%
			\usebeamerfont{title in head/foot}\insertshorttitle
		\end{beamercolorbox}%
		\begin{beamercolorbox}[wd=.333333\paperwidth,ht=2.5ex,dp=1ex,center]{date in head/foot}%
			%\usebeamerfont{date in head/foot}\insertshortdate{}\hspace*{2em}
			%\insertframenumber\hspace*{2ex} 
		\end{beamercolorbox}
		\begin{beamercolorbox}[wd=.333333\paperwidth,ht=3ex,dp=1ex,center]{author in head/foot}%
			\usebeamerfont{author in head/foot}\insertshortauthor~~%\beamer@ifempty{\insertshortinstitute}{}{(\insertshortinstitute)}
		\end{beamercolorbox}%
	}%
	\vskip0pt%
}
\makeatother
\makeatletter
\def\progressbar@progressbar{} % the progress bar
\newcount\progressbar@tmpcounta% auxiliary counter
\newcount\progressbar@tmpcountb% auxiliary counter
\newdimen\progressbar@pbht %progressbar height
\newdimen\progressbar@pbwd %progressbar width
\newdimen\progressbar@tmpdim % auxiliary dimension
\progressbar@pbwd=\linewidth
\progressbar@pbht=1.5ex
\def\progressbar@progressbar{%
    \progressbar@tmpcounta=\insertframenumber
    \progressbar@tmpcountb=\inserttotalframenumber
    \progressbar@tmpdim=\progressbar@pbwd
    \multiply\progressbar@tmpdim by \progressbar@tmpcounta
    \divide\progressbar@tmpdim by \progressbar@tmpcountb
  \begin{tikzpicture}[rounded corners=3pt,very thin]
    \shade[top color=pbgray!20,bottom color=pbgray!20,middle color=pbgray!50]
      (0pt, 0pt) rectangle ++ (\progressbar@pbwd, \progressbar@pbht);
      \shade[draw=pbblue,top color=pbblue!50,bottom color=pbblue!50,middle color=pbblue] %
        (0pt, 0pt) rectangle ++ (\progressbar@tmpdim, \progressbar@pbht);
    \draw[color=normal text.fg!50]  
      (0pt, 0pt) rectangle (\progressbar@pbwd, \progressbar@pbht) 
        node[pos=0.5,color=normal text.fg] {\textnormal{%
             \pgfmathparse{\insertframenumber*100/\inserttotalframenumber}%
             \pgfmathprintnumber[fixed,precision=0]{\pgfmathresult}\,\%%
        }%
    };
  \end{tikzpicture}%
}
\addtobeamertemplate{headline}{}
{%
  \begin{beamercolorbox}[wd=\paperwidth,ht=4ex,center,dp=1ex]{white}%
    \progressbar@progressbar%
  \end{beamercolorbox}%
}
\makeatother


%%% BLOCKS
% block = Aufgabe
\setbeamercolor{block title}{fg=black,bg=blue!30!white} 
\setbeamercolor{block body}{fg=black, bg=blue!3!white}

% alertblock = Definition
\setbeamercolor{block title alerted}{fg=black,bg=red!50!white} 
\setbeamercolor{block body alerted}{fg=black, bg=red!3!white}

% exampleblock = Wiederholung
\setbeamercolor{block title example}{fg=black,bg=green!30!white} 
\setbeamercolor{block body example}{fg=black, bg=green!3!white}






\begin{document}
	\author{www.oilbat.de}
	%\title{Stochastik}
	\subtitle{}
	\logo{}
	\institute{}
	\date{}
	\subject{}
	\setbeamercovered{transparent}
	\setbeamertemplate{navigation symbols}{}

\addtocounter{framenumber}{-1}
\setbeamercovered{invisible}

\begin{frame}
	\begin{block}{Aufgabe}
		Bestimme den Maximum-Likelihood-Schätzer für den Erfolgsparameter der Geometrischen Verteilung.
	\end{block}
\end{frame}

\begin{frame}
\begin{block}{Aufgabe}
	Bestimme den Maximum-Likelihood-Schätzer für den Erfolgsparameter der Geometrischen Verteilung.
\end{block}
\begin{alertblock}{Definition: Geometrische Verteilung}
	Eine Zufallsvariable $X$ ist gemäß der Geometrischen Verteilung zu dem Erfolgsparameter $p\in (0,1)$ verteilt, falls
	\begin{equation*}
	f(x)=P(X=x)=(1-p)^xp, \ \ \ x=0,1,2,\dots
	\end{equation*}
\end{alertblock}
\end{frame}

\begin{frame}
\begin{block}{Aufgabe}
	Bestimme den Maximum-Likelihood-Schätzer für den Erfolgsparameter der Geometrischen Verteilung.
\end{block}
\begin{alertblock}{Definition: Geometrische Verteilung}
	Eine Zufallsvariable $X$ ist gemäß der Geometrischen Verteilung zu dem Erfolgsparameter $p\in (0,1)$ verteilt, falls
	\begin{equation*}
	f(x)=P(X=x)=(1-p)^xp, \ \ \ x=0,1,2,\dots
	\end{equation*}
\end{alertblock}
\vfill
Stichprobe:
\begin{align*}
	2 ~~ 5 ~~ 16 ~~ 6 ~~ 10 ~~ 5 ~~ 2 ~~ 3 ~~ 5 ~~ 0 ~~ 10 ~~ 3 ~~ 3 ~~ 1 ~~ 1 ~~ 6 ~~ 15 ~~ 6 ~~ 18 ~~ 3
\end{align*}
\end{frame}

\begin{frame}
	\begin{exampleblock}{Wiederholung: Allgemeines Verfahren}
		\begin{enumerate}
			\item Aufstellen der ML-Funktion $$L(\theta)=\prod_{i=1}^{n}f_{X_i}(\theta)$$
			\item Logarithmierung $$\log L(\theta)=\sum_{i=1}^{n}\log f_{X_i}(\theta)$$
			\item Bestimmung der Ableitung $\frac{d}{d\theta}\log L(\theta)$
			\item Ableitung gleich Null setzen und nach dem gesuchten Parameter auflösen
		\end{enumerate}
	\end{exampleblock}
\end{frame}

\begin{frame}
\begin{tikzpicture}
\begin{axis}[
domain=0:10,
xmin=0, xmax=10,
ymin=0, ymax=5,
samples=100,
axis y line=center,
axis x line=middle,
]
\addplot+[mark=none,color=green] {ln(x)};
\addlegendentry{$\log(x)$}
\end{axis}
\end{tikzpicture}
\end{frame}

\begin{frame}
	\begin{tikzpicture}
	\begin{axis}[
		domain=0:10,
		xmin=0, xmax=10,
		ymin=0, ymax=5,
		samples=100,
		axis y line=center,
		axis x line=middle,
		]
		\addplot+[mark=none] {-(x-4)^2+5};
		\addlegendentry{$f(x)$}
	\end{axis}
	\end{tikzpicture}
\end{frame}

\begin{frame}
\begin{tikzpicture}
\begin{axis}[
domain=0:10,
xmin=0, xmax=10,
ymin=0, ymax=5,
samples=100,
axis y line=center,
axis x line=middle,
]
\addplot+[mark=none] {-(x-4)^2+5};
\addplot+[mark=none] {ln(-(x-4)^2+5)};
\addlegendentry{$f(x)$}
\addlegendentry{$\log f(x)$}
\end{axis}
\end{tikzpicture}
\end{frame}

\begin{frame}
	\begin{exampleblock}{Wiederholung: Allgemeines Verfahren}
		\begin{enumerate}
			\item Aufstellen der ML-Funktion $$L(\theta)=\prod_{i=1}^{n}f_{X_i}(\theta)$$
		\end{enumerate}
	\end{exampleblock}
\end{frame}

\begin{frame}
\begin{exampleblock}{Wiederholung: Allgemeines Verfahren}
	\begin{enumerate}
		\item Aufstellen der ML-Funktion $$L(\theta)=\prod_{i=1}^{n}f_{X_i}(\theta)$$
	\end{enumerate}
\end{exampleblock}
\begin{alertblock}{Definition: Geometrische Verteilung}
	Eine Zufallsvariable $X$ ist gemäß der Geometrischen Verteilung zu dem Erfolgsparameter $p\in (0,1)$ verteilt, falls
	\begin{equation*}
	f(x)=P(X=x)=(1-p)^xp, \ \ \ x=0,1,2,\dots
	\end{equation*}
\end{alertblock}
\end{frame}

\begin{frame}
\begin{exampleblock}{Wiederholung: Allgemeines Verfahren}
	\begin{enumerate}
		\item Aufstellen der ML-Funktion $$L(\theta)=\prod_{i=1}^{n}f_{X_i}(\theta)$$
	\end{enumerate}
\end{exampleblock}
\begin{alertblock}{Definition: Geometrische Verteilung}
	Eine Zufallsvariable $X$ ist gemäß der Geometrischen Verteilung zu dem Erfolgsparameter $p\in (0,1)$ verteilt, falls
	\begin{equation*}
	f(x)=P(X=x)=(1-p)^xp, \ \ \ x=0,1,2,\dots
	\end{equation*}
\end{alertblock}
$$L(p)=\prod_{i=1}^{n}f_{X_i}(p)=\prod_{i=1}^{n}(1-p)^{X_i}p$$
\end{frame}

\begin{frame}
	$$L(p)=\prod_{i=1}^{n}f_{X_i}(p)=\prod_{i=1}^{n}(1-p)^{X_i}p$$
	\begin{exampleblock}{Wiederholung: Allgemeines Verfahren}
		\begin{enumerate}
			\item Aufstellen der ML-Funktion $\checkmark$
			\item Logarithmierung: $\log L(\theta)=\sum_{i=1}^{n}\log f_{X_i}(\theta)$
		\end{enumerate}
	\end{exampleblock}
\end{frame}

\begin{frame}
$$L(p)=\prod_{i=1}^{n}f_{X_i}(p)=\prod_{i=1}^{n}(1-p)^{X_i}p$$
\begin{exampleblock}{Wiederholung: Allgemeines Verfahren}
	\begin{enumerate}
		\item Aufstellen der ML-Funktion $\checkmark$
		\item Logarithmierung: $\log L(\theta)=\sum_{i=1}^{n}\log f_{X_i}(\theta)$
	\end{enumerate}
\end{exampleblock}
\begin{align*}
\log L(p) &= \sum_{i=1}^{n}\log f_{X_i}(p) = \sum_{i=1}^{n}\log \left[ (1-p)^{X_i}p \right] \\
&= \sum_{i=1}^{n} \left[ \log (1-p)^{X_i} + \log p \right] = \sum_{i=1}^{n} X_i \cdot  \log (1-p) + \sum_{i=1}^{n} \log p \\
&= \log (1-p) \sum_{i=1}^{n} X_i + n\cdot \log p
\end{align*}
\end{frame}

\begin{frame}
	$$\log L(p) = \log (1-p) \sum_{i=1}^{n} X_i + n\cdot \log p$$
	\begin{exampleblock}{Wiederholung: Allgemeines Verfahren}
		\begin{enumerate}
			\item Aufstellen der ML-Funktion $\checkmark$ 
			\item Logarithmierung $\checkmark$
			\item Bestimmung der Ableitung $\frac{d}{d\theta}\log L(\theta)$
		\end{enumerate}
	\end{exampleblock}
\end{frame}

\begin{frame}
$$\log L(p) = \log (1-p) \sum_{i=1}^{n} X_i + n\cdot \log p$$
\begin{exampleblock}{Wiederholung: Allgemeines Verfahren}
	\begin{enumerate}
		\item Aufstellen der ML-Funktion $\checkmark$ 
		\item Logarithmierung $\checkmark$
		\item Bestimmung der Ableitung $\frac{d}{d\theta}\log L(\theta)$
	\end{enumerate}
\end{exampleblock}
\begin{align*}
\frac{d}{d\theta}\log L(p) &=  (-1) \cdot \frac{1}{1-p} \sum_{i=1}^{n} X_i + n\cdot \frac{1}{p} \\
&= -\frac{1}{1-p}\sum_{i=1}^{n} X_i +  \frac{n}{p}
\end{align*}
\end{frame}

\begin{frame}
	$$  \frac{d}{d\theta}\log L(p) = -\frac{1}{1-p}\sum_{i=1}^{n} X_i +  \frac{n}{p}$$
	\begin{exampleblock}{Wiederholung: Allgemeines Verfahren}
		\begin{enumerate}
			\item Aufstellen der ML-Funktion $\checkmark$
			\item Logarithmierung $\checkmark$
			\item Bestimmung der Ableitung $\checkmark$
			\item Ableitung gleich Null setzen und nach dem gesuchten Parameter auflösen
		\end{enumerate}
	\end{exampleblock}
\end{frame}

\begin{frame}
$$  \frac{d}{d\theta}\log L(p) = -\frac{1}{1-p}\sum_{i=1}^{n} X_i +  \frac{n}{p}$$
\begin{exampleblock}{Wiederholung: Allgemeines Verfahren}
	\begin{enumerate}
		\item Aufstellen der ML-Funktion $\checkmark$
		\item Logarithmierung $\checkmark$
		\item Bestimmung der Ableitung $\checkmark$
		\item Ableitung gleich Null setzen und nach dem gesuchten Parameter auflösen
	\end{enumerate}
\end{exampleblock}
$$   -\frac{1}{1-p}\sum_{i=1}^{n} X_i +  \frac{n}{p} \overset{!}{=} 0$$
\end{frame}

\begin{frame}
	\begin{align*}
		&-\frac{1}{1-p}\sum_{i=1}^{n} X_i +  \frac{n}{p} \overset{!}{=} 0 \\
		\Longleftrightarrow&  -\frac{n\bar{X}}{1-p} +  \frac{n}{p} = 0 \\
		\Longleftrightarrow& n\left(-\frac{\bar{X}}{1-p} +  \frac{1}{p} \right) = 0 \\
		\Longleftrightarrow& \frac{-\bar{X}p+(1-p)}{(1-p)p} = 0 \\
		\Longleftrightarrow& -\bar{X}p+(1-p) = 0 \\
		\Longleftrightarrow& 1 = \bar{X}p+p \\
		\Longleftrightarrow& 1 = p\left(\bar{X}+1\right) \\
		\Longleftrightarrow& p = \frac{1}{\bar{X}+1} \\
	\end{align*}
\end{frame}

\begin{frame}
	\begin{block}{Aufgabe}
		Bestimme den Maximum-Likelihood-Schätzer für den Erfolgsparameter der Geometrischen Verteilung.
	\end{block}
	$$\hat{p}_{ML}=\frac{1}{\bar{X}+1}$$
\end{frame}

\begin{frame}
\begin{block}{Aufgabe}
	Bestimme den Maximum-Likelihood-Schätzer für den Erfolgsparameter der Geometrischen Verteilung.
\end{block}
$$\hat{p}_{ML}=\frac{1}{\bar{X}+1}$$
\vfill
Stichprobe:
\begin{align*}
2 ~~ 5 ~~ 16 ~~ 6 ~~ 10 ~~ 5 ~~ 2 ~~ 3 ~~ 5 ~~ 0 ~~ 10 ~~ 3 ~~ 3 ~~ 1 ~~ 1 ~~ 6 ~~ 15 ~~ 6 ~~ 18 ~~ 3
\end{align*}
\vfill
Maximum-Likelihood-Schätzer:
\begin{align*}
\hat{p}_{ML} = \frac{1}{\bar{X}+1} = \frac{1}{6+1} =  \frac{1}{7}
\end{align*}
\end{frame}

\end{document}