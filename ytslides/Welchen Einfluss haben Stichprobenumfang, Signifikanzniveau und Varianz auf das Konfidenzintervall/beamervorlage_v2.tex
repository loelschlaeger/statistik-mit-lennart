\documentclass[t,11pt,aspectratio=169]{beamer}
\usepackage{tikz}
\usepackage{pgfplots}
\usetikzlibrary{calc}
\usepackage[utf8]{inputenc}
\usepackage[ngerman]{babel}
\usepackage{amsmath,amsfonts,amssymb}
\usepackage{framed}
\usecolortheme{orchid}
\usepackage{etoolbox}
\useinnertheme[shadow=true]{rounded}

%%% PROGRESSBAR
\definecolor{pbblue}{HTML}{D8D8D8}% filling color for the progress bar
\definecolor{pbgray}{HTML}{F2F2F2}% background color for the progress bar
\useoutertheme{infolines}
\setbeamerfont{footline}{size=\normalsize}
\setbeamersize{text margin left=30pt,text margin right=30pt}
\makeatletter
\setbeamertemplate{footline}
{
	\leavevmode%
	\hbox{%
		\begin{beamercolorbox}[wd=.333333\paperwidth,ht=2.5ex,dp=1ex,center]{title in head/foot}%
			\usebeamerfont{title in head/foot}\insertshorttitle
		\end{beamercolorbox}%
		\begin{beamercolorbox}[wd=.333333\paperwidth,ht=2.5ex,dp=1ex,center]{date in head/foot}%
			%\usebeamerfont{date in head/foot}\insertshortdate{}\hspace*{2em}
			%\insertframenumber\hspace*{2ex} 
		\end{beamercolorbox}
		\begin{beamercolorbox}[wd=.333333\paperwidth,ht=3ex,dp=1ex,center]{author in head/foot}%
			\usebeamerfont{author in head/foot}\insertshortauthor~~%\beamer@ifempty{\insertshortinstitute}{}{(\insertshortinstitute)}
		\end{beamercolorbox}%
	}%
	\vskip0pt%
}
\makeatother
\makeatletter
\def\progressbar@progressbar{} % the progress bar
\newcount\progressbar@tmpcounta% auxiliary counter
\newcount\progressbar@tmpcountb% auxiliary counter
\newdimen\progressbar@pbht %progressbar height
\newdimen\progressbar@pbwd %progressbar width
\newdimen\progressbar@tmpdim % auxiliary dimension
\progressbar@pbwd=\linewidth
\progressbar@pbht=1.5ex
\def\progressbar@progressbar{%
    \progressbar@tmpcounta=\insertpagenumber
    \progressbar@tmpcountb=\insertdocumentendpage
    \progressbar@tmpdim=\progressbar@pbwd
    \multiply\progressbar@tmpdim by \progressbar@tmpcounta
    \divide\progressbar@tmpdim by \progressbar@tmpcountb
  \begin{tikzpicture}[rounded corners=3pt,very thin]
    \shade[top color=pbgray!20,bottom color=pbgray!20,middle color=pbgray!50]
      (0pt, 0pt) rectangle ++ (\progressbar@pbwd, \progressbar@pbht);
      \shade[draw=pbblue,top color=pbblue!50,bottom color=pbblue!50,middle color=pbblue] %
        (0pt, 0pt) rectangle ++ (\progressbar@tmpdim, \progressbar@pbht);
    \draw[color=normal text.fg!50]  
      (0pt, 0pt) rectangle (\progressbar@pbwd, \progressbar@pbht) 
        node[pos=0.5,color=normal text.fg] {\textnormal{%
             \pgfmathparse{\insertpagenumber*100/\insertdocumentendpage}%
             \pgfmathprintnumber[fixed,precision=0]{\pgfmathresult}\,\%%
        }%
    };
  \end{tikzpicture}%
}
\addtobeamertemplate{headline}{}
{%
  \begin{beamercolorbox}[wd=\paperwidth,ht=4ex,center,dp=1ex]{white}%
    \progressbar@progressbar%
  \end{beamercolorbox}%
}
\makeatother

\setbeamertemplate{frametitle}[default][center]

%%% BLOCKS
% block = Aufgabe
\setbeamercolor{block title}{fg=black,bg=blue!30!white} 
\setbeamercolor{block body}{fg=black, bg=blue!3!white}

% alertblock = Definition
\setbeamercolor{block title alerted}{fg=black,bg=red!50!white} 
\setbeamercolor{block body alerted}{fg=black, bg=red!3!white}

% exampleblock = Wiederholung
\setbeamercolor{block title example}{fg=black,bg=green!30!white} 
\setbeamercolor{block body example}{fg=black, bg=green!3!white}

\setbeamercovered{transparent}
\setbeamertemplate{navigation symbols}{}

\addtocounter{page}{-1}
\addtocounter{framenumber}{-1}
\setbeamercovered{invisible}



\begin{document}
	
\begin{frame}[fragile]
	Stichprobe aus einer Normalverteilung:
	\begin{verbatim}
	> x
	-0.3109498  1.3417310 -0.9245257 -0.1026436  1.0769149 
	-1.1520844 -0.9227701 -1.7029334 -0.3855018  0.6828499
	\end{verbatim}
	\pause
	Maximum-Likelihood-Schätzwert für $\mu$:
	\begin{align*}
		\hat{\mu}_{\text{ML}} = \bar{X} = -0.2399913
	\end{align*}
	\pause
	95\%-Konfidenzintervall für $\mu$:
	\begin{align*}
		\left[ -0.9569236;  0.476941 \right]
	\end{align*}
\end{frame}

\begin{frame}
\begin{alertblock}{$(1-\alpha)$ - Konfidenzintervall für den Erwartungswert einer Normalverteilung}
	$\sigma$ bekannt:
	\begin{align*}
	\left[ \bar{X} - z_{1-\alpha/2} \frac{\sigma}{\sqrt{n}};~\bar{X} + z_{1-\alpha/2} \frac{\sigma}{\sqrt{n}}  \right]
	\end{align*}
	$\sigma$ unbekannt:
	\begin{align*}
	\left[ \bar{X} - t_{1-\alpha/2; n-1} \frac{S}{\sqrt{n}};~\bar{X} + t_{1-\alpha/2; n-1} \frac{S}{\sqrt{n}} \right]
	\end{align*}
\end{alertblock}
\pause Wie verändert sich das Konfidenzintervall, wenn
\begin{itemize}
	\item $n$ steigt?
	\item $\alpha$ sinkt?
	\item $\sigma$ steigt?
	\item $\sigma$ durch die Stichprobenstandardabweichung $S$ geschätzt wird?
\end{itemize}
\end{frame}

\begin{frame}
\begin{alertblock}{$(1-\alpha)$ - Konfidenzintervall für den Erwartungswert einer Normalverteilung}
$\sigma$ bekannt:
\begin{align*}
\left[ \bar{X} - z_{1-\alpha/2} \frac{\sigma}{\sqrt{n}};~\bar{X} + z_{1-\alpha/2} \frac{\sigma}{\sqrt{n}}  \right]
\end{align*}
\end{alertblock}
Breite $b$ des KI:
\begin{align*}
b = \bar{X} + z_{1-\alpha/2} \frac{\sigma}{\sqrt{n}} -  \left(\bar{X} - z_{1-\alpha/2} \frac{\sigma}{\sqrt{n}} \right)= 2z_{1-\alpha/2}\frac{\sigma}{\sqrt{n}}
\end{align*}
\pause Und damit gilt:
\begin{itemize}
	\item Wenn $n$ steigt, dann wird das KI schmaler.
	\item Wenn $\alpha$ sinkt wird $z_{1-\alpha/2}$ größer und damit das KI breiter.
	\item Wenn $\sigma$ steigt wird das KI breiter.
\end{itemize}
\end{frame}

\begin{frame}
Folgende konkrete Maßnahmen können wir ableiten, um das KI schmaler zu machen:
\begin{itemize}
	\item Erhöhe den Stichprobenumfang. Das führt evtl. zu höheren Kosten.
	\item Wähle ein größeres $\alpha$. Damit riskieren wir aber, den wahren Parameter nicht einzufangen.
\end{itemize}
\end{frame}

\begin{frame}
Will man zu fester Irrtumswahrscheinlichkeit $\alpha$ ein Konfidenzintervall mit bestimmter Maximalbreite haben, kann man den notwendigen Stichprobenumfang berechnen:
\begin{align*}
&b = 2z_{1-\alpha/2}\frac{\sigma}{\sqrt{n}} \\
\Leftrightarrow ~~ & n = \left(\frac{2z_{1-\alpha/2}\sigma}{b} \right )^2
\end{align*}
\end{frame}

\begin{frame}
Was ist mit dem letzten Fall: $\sigma$ wird durch die Stichprobenstandardabweichung $S$ geschätzt?
\begin{align*}
	S = \sqrt{\frac{1}{n-1}\sum_{i=1}^{n}(X_i-\bar{X})}
\end{align*}
\pause
\begin{center}
\begin{tabular}{cccc}
$\alpha$ & $z_{1-\alpha/2}$ & $t_{1-\alpha/2; 10}$ & $t_{1-\alpha/2; 30}$\\
\hline
0,01 & 2,58 & 3,17 & 2,75\\
0,05 & 1,96 & 2,23 & 2,04 \\
0,10 & 1,64 & 1,81 & 1,70
\end{tabular}
\end{center}
\pause
Und damit gilt:
\begin{itemize}
	\item Wenn $\sigma$ durch $S$ geschätzt wird, wird das KI breiter.
	\item Wenn $n>30$, ist $z_{1-\alpha/2}\approx t_{1-\alpha/2; n-1}$ und der Effekt ist vernachlässigbar.
\end{itemize}
\end{frame}

\end{document}