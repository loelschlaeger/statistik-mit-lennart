\documentclass[t,11pt,aspectratio=169]{beamer}
\usepackage{tikz}
\usepackage{pgfplots}
\usetikzlibrary{calc}
\usepackage[utf8]{inputenc}
\usepackage[ngerman]{babel}
\usepackage{amsmath,amsfonts,amssymb}
\usepackage{framed}
\usecolortheme{orchid}
\usepackage{etoolbox}
\useinnertheme[shadow=true]{rounded}

%%% PROGRESSBAR
\definecolor{pbblue}{HTML}{D8D8D8}% filling color for the progress bar
\definecolor{pbgray}{HTML}{F2F2F2}% background color for the progress bar
\useoutertheme{infolines}
\setbeamerfont{footline}{size=\normalsize}
\setbeamersize{text margin left=30pt,text margin right=30pt}
\makeatletter
\setbeamertemplate{footline}
{
	\leavevmode%
	\hbox{%
		\begin{beamercolorbox}[wd=.333333\paperwidth,ht=2.5ex,dp=1ex,center]{title in head/foot}%
			\usebeamerfont{title in head/foot}\insertshorttitle
		\end{beamercolorbox}%
		\begin{beamercolorbox}[wd=.333333\paperwidth,ht=2.5ex,dp=1ex,center]{date in head/foot}%
			%\usebeamerfont{date in head/foot}\insertshortdate{}\hspace*{2em}
			%\insertframenumber\hspace*{2ex} 
		\end{beamercolorbox}
		\begin{beamercolorbox}[wd=.333333\paperwidth,ht=3ex,dp=1ex,center]{author in head/foot}%
			\usebeamerfont{author in head/foot}\insertshortauthor~~%\beamer@ifempty{\insertshortinstitute}{}{(\insertshortinstitute)}
		\end{beamercolorbox}%
	}%
	\vskip0pt%
}
\makeatother
\makeatletter
\def\progressbar@progressbar{} % the progress bar
\newcount\progressbar@tmpcounta% auxiliary counter
\newcount\progressbar@tmpcountb% auxiliary counter
\newdimen\progressbar@pbht %progressbar height
\newdimen\progressbar@pbwd %progressbar width
\newdimen\progressbar@tmpdim % auxiliary dimension
\progressbar@pbwd=\linewidth
\progressbar@pbht=1.5ex
\def\progressbar@progressbar{%
    \progressbar@tmpcounta=\insertpagenumber
    \progressbar@tmpcountb=\insertdocumentendpage
    \progressbar@tmpdim=\progressbar@pbwd
    \multiply\progressbar@tmpdim by \progressbar@tmpcounta
    \divide\progressbar@tmpdim by \progressbar@tmpcountb
  \begin{tikzpicture}[rounded corners=3pt,very thin]
    \shade[top color=pbgray!20,bottom color=pbgray!20,middle color=pbgray!50]
      (0pt, 0pt) rectangle ++ (\progressbar@pbwd, \progressbar@pbht);
      \shade[draw=pbblue,top color=pbblue!50,bottom color=pbblue!50,middle color=pbblue] %
        (0pt, 0pt) rectangle ++ (\progressbar@tmpdim, \progressbar@pbht);
    \draw[color=normal text.fg!50]  
      (0pt, 0pt) rectangle (\progressbar@pbwd, \progressbar@pbht) 
        node[pos=0.5,color=normal text.fg] {\textnormal{%
             \pgfmathparse{\insertpagenumber*100/\insertdocumentendpage}%
             \pgfmathprintnumber[fixed,precision=0]{\pgfmathresult}\,\%%
        }%
    };
  \end{tikzpicture}%
}
\addtobeamertemplate{headline}{}
{%
  \begin{beamercolorbox}[wd=\paperwidth,ht=4ex,center,dp=1ex]{white}%
    \progressbar@progressbar%
  \end{beamercolorbox}%
}
\makeatother

\setbeamertemplate{frametitle}[default][center]

%%% BLOCKS
% block = Aufgabe
\setbeamercolor{block title}{fg=black,bg=blue!30!white} 
\setbeamercolor{block body}{fg=black, bg=blue!3!white}

% alertblock = Definition
\setbeamercolor{block title alerted}{fg=black,bg=red!50!white} 
\setbeamercolor{block body alerted}{fg=black, bg=red!3!white}

% exampleblock = Wiederholung
\setbeamercolor{block title example}{fg=black,bg=green!30!white} 
\setbeamercolor{block body example}{fg=black, bg=green!3!white}

\setbeamercovered{transparent}
\setbeamertemplate{navigation symbols}{}

\addtocounter{page}{-1}
\addtocounter{framenumber}{-1}
\setbeamercovered{invisible}



\begin{document}
	
\begin{frame}{Ein Beispiel}
\begin{itemize}
	\item Ein Tinder-Match alle 40 Stunden
	\pause\item Zeit bis zum nächsten Match ist $X\sim\text{Exp}(\lambda=1/40)$
	\pause\item Gerade hatte ich ein Match, die Wahrscheinlichkeit mehr als $t$ Stunden auf das nächste Match zu warten ist $P(X\geq t)$
	\pause\item An einem Morgen sehe ich, dass ich am Abend zuvor (vor 10 Stunden) ein Match hatte. Die Verteilung der Wartezeit bis zum nächsten Match ist $P(X\geq 10+t \mid X\geq 10)$
	\pause\item Entscheidend ist jetzt: $P(X\geq t)=P(X\geq 10+t \mid X\geq 10)$, das heißt die Verteilungen sind gleich!
	\pause\item Aber warum?
\end{itemize}
\end{frame}

\begin{frame}{Die Exponentialverteilung}
\begin{itemize}
	\item Träger: nichtnegative, reelle Zahlen
	\item Parameter: $\lambda>0$
	\item Erwartungswert: $1/\lambda$
	\item Varianz: $1/\lambda^2$
	\item Einsatz: Modellierung von Lebensdauern oder Wartezeiten
	\pause \item Dichtefunktion: 
	\begin{align*}
		f(x)= \begin{cases}
			0 & x\leq 0 \\
			\lambda e^{-\lambda x} & x>0 
		\end{cases}
	\end{align*}
	\pause \item Verteilungsfunktion:
	\begin{align*}
	F(x)=\begin{cases}
		0 & x\leq 0 \\
		1-e^{-\lambda x} & x>0
	\end{cases}
	\end{align*}
\end{itemize}
\end{frame}

\begin{frame}
\begin{center}
	\includegraphics[height=\textheight]{exp.pdf}
\end{center}
\end{frame}

\begin{frame}
	\begin{center}
		\includegraphics[height=\textheight]{ExpDF.pdf}
	\end{center}
\end{frame}

\begin{frame}{Die Gedächtnislosigkeit der Exponentialverteilung}
	Wenn $X\sim \text{Exp}(\lambda)$, dann ist
	\begin{align*}
		P(X\geq s+t\mid X\geq s) = P(X\geq t)~~~\forall s,t > 0
	\end{align*}

	\pause Die Exponentialverteilung ist die \textbf{einzige stetige} Verteilung, die diese Eigenschaft erfüllt. Im diskreten Fall erfüllt die geometrische Verteilung diese Eigenschaft.
\end{frame}

\begin{frame}{Ein Beispiel}
	\begin{itemize}
		\item Ein Tinder-Match alle 40 Stunden
		\item Zeit bis zum nächsten Match ist $X\sim\text{Exp}(\lambda=1/40)$
		\item Gerade hatte ich ein Match, die Wahrscheinlichkeit mehr als $t$ Stunden auf das nächste Match zu warten ist $P(X\geq t)$
		\item An einem Morgen sehe ich, dass ich am Abend zuvor (vor 10 Stunden) ein Match hatte. Die Verteilung der Wartezeit bis zum nächsten Match ist $P(X\geq 10+t \mid X\geq 10)$
		\item Entscheidend ist jetzt: $P(X\geq t)=P(X\geq 10+t \mid X\geq 10)$, das heißt die Verteilungen sind gleich!
		\item Aber warum?
	\end{itemize}
\end{frame}

\begin{frame}{Der Beweis}
	Warum ist $P(X\geq t)=P(X\geq 10+t \mid X\geq 10)$?
	\pause \begin{align*}
		P(X\geq 10+t \mid X\geq 10) &= \frac{P(X\geq 10+t, X\geq 10)}{P(X\geq 10)} \\
		&= \frac{P(X\geq 10+t)}{P(X\geq 10)} \\
		&= \frac{1-P(X\leq 10+t)}{1-P(X\leq 10)} \\
		&= \frac{1-(1-e^{-\lambda(10+t)})}{1-(1-e^{-\lambda 10})} \\
		&= \frac{e^{-\lambda(10+t)}}{e^{-\lambda 10}} \\
		&= e^{-\lambda(10+t)-(-\lambda10)} = e^{-\lambda t} = P(X\geq t)
	\end{align*}
\end{frame}

\end{document}