\documentclass[t,11pt]{beamer}
\usepackage{tikz}
\usepackage{pgfplots}
\usetikzlibrary{calc}
\usepackage[utf8]{inputenc}
\usepackage[ngerman]{babel}
\usepackage{amsmath,amsfonts,amssymb}
\usepackage{framed}
\usecolortheme{orchid}
\usepackage{etoolbox}
\useinnertheme[shadow=true]{rounded}

%%% PROGRESSBAR
\definecolor{pbblue}{HTML}{D8D8D8}% filling color for the progress bar
\definecolor{pbgray}{HTML}{F2F2F2}% background color for the progress bar
\useoutertheme{infolines}
\setbeamerfont{footline}{size=\normalsize}
\setbeamersize{text margin left=30pt,text margin right=30pt}
\makeatletter
\setbeamertemplate{footline}
{
	\leavevmode%
	\hbox{%
		\begin{beamercolorbox}[wd=.333333\paperwidth,ht=2.5ex,dp=1ex,center]{title in head/foot}%
			\usebeamerfont{title in head/foot}\insertshorttitle
		\end{beamercolorbox}%
		\begin{beamercolorbox}[wd=.333333\paperwidth,ht=2.5ex,dp=1ex,center]{date in head/foot}%
			%\usebeamerfont{date in head/foot}\insertshortdate{}\hspace*{2em}
			%\insertframenumber\hspace*{2ex} 
		\end{beamercolorbox}
		\begin{beamercolorbox}[wd=.333333\paperwidth,ht=3ex,dp=1ex,center]{author in head/foot}%
			\usebeamerfont{author in head/foot}\insertshortauthor~~%\beamer@ifempty{\insertshortinstitute}{}{(\insertshortinstitute)}
		\end{beamercolorbox}%
	}%
	\vskip0pt%
}
\makeatother
\makeatletter
\def\progressbar@progressbar{} % the progress bar
\newcount\progressbar@tmpcounta% auxiliary counter
\newcount\progressbar@tmpcountb% auxiliary counter
\newdimen\progressbar@pbht %progressbar height
\newdimen\progressbar@pbwd %progressbar width
\newdimen\progressbar@tmpdim % auxiliary dimension
\progressbar@pbwd=\linewidth
\progressbar@pbht=1.5ex
\def\progressbar@progressbar{%
    \progressbar@tmpcounta=\insertframenumber
    \progressbar@tmpcountb=\inserttotalframenumber
    \progressbar@tmpdim=\progressbar@pbwd
    \multiply\progressbar@tmpdim by \progressbar@tmpcounta
    \divide\progressbar@tmpdim by \progressbar@tmpcountb
  \begin{tikzpicture}[rounded corners=3pt,very thin]
    \shade[top color=pbgray!20,bottom color=pbgray!20,middle color=pbgray!50]
      (0pt, 0pt) rectangle ++ (\progressbar@pbwd, \progressbar@pbht);
      \shade[draw=pbblue,top color=pbblue!50,bottom color=pbblue!50,middle color=pbblue] %
        (0pt, 0pt) rectangle ++ (\progressbar@tmpdim, \progressbar@pbht);
    \draw[color=normal text.fg!50]  
      (0pt, 0pt) rectangle (\progressbar@pbwd, \progressbar@pbht) 
        node[pos=0.5,color=normal text.fg] {\textnormal{%
             \pgfmathparse{\insertframenumber*100/\inserttotalframenumber}%
             \pgfmathprintnumber[fixed,precision=0]{\pgfmathresult}\,\%%
        }%
    };
  \end{tikzpicture}%
}
\addtobeamertemplate{headline}{}
{%
  \begin{beamercolorbox}[wd=\paperwidth,ht=4ex,center,dp=1ex]{white}%
    \progressbar@progressbar%
  \end{beamercolorbox}%
}
\makeatother


%%% BLOCKS
% block = Aufgabe
\setbeamercolor{block title}{fg=black,bg=blue!30!white} 
\setbeamercolor{block body}{fg=black, bg=blue!3!white}

% alertblock = Definition
\setbeamercolor{block title alerted}{fg=black,bg=red!50!white} 
\setbeamercolor{block body alerted}{fg=black, bg=red!3!white}

% exampleblock = Wiederholung
\setbeamercolor{block title example}{fg=black,bg=green!30!white} 
\setbeamercolor{block body example}{fg=black, bg=green!3!white}



\begin{document}
	%\author{www.oilbat.de}
	%\title{Stochastik}
	\subtitle{}
	\logo{}
	\institute{}
	\date{}
	\subject{}
	\setbeamercovered{transparent}
	\setbeamertemplate{navigation symbols}{}

\addtocounter{framenumber}{-1}
\setbeamercovered{invisible}

\begin{frame}
\begin{alertblock}{Verschiebungssatz}
	Für eine diskrete oder eine stetige Zufallsvariable $X$ gilt
	\begin{align*}
	\mathbb{V}(X) =  \mathbb{E}(X^2)-\left( \mathbb{E}(X) \right)^2.
	\end{align*}
\end{alertblock}
\end{frame}

\begin{frame}
\begin{exampleblock}{Varianz und Erwartungswert einer diskreten Zufallsvariablen}
	Sei $X$ eine diskrete Zufallsvariable mit der W'keitsfunktion $f(x)$.
	\begin{align*}
	\mathbb{E}(X) &= \sum_i x_i f(x_i) \\
	\mathbb{V}(X) &= \sum_i \left(x_i-\mathbb{E}(X)\right)^2 f(x_i)
	\end{align*}
\end{exampleblock}
\end{frame}

\begin{frame}
	\begin{exampleblock}{Varianz und Erwartungswert einer diskreten Zufallsvariablen}
		Sei $X$ eine diskrete Zufallsvariable mit der W'keitsfunktion $f(x)$.
		\begin{align*}
		\mathbb{E}(X) &= \sum_i x_i f(x_i) \\
		\mathbb{V}(X) &= \sum_i \left(x_i-\mathbb{E}(X)\right)^2 f(x_i)
		\end{align*}
	\end{exampleblock}
	\begin{exampleblock}{Varianz und Erwartungswert einer stetigen Zufallsvariablen}
		Sei $X$ eine stetige Zufallsvariable mit der Dichte $f(x)$.
		\begin{align*}
		\mathbb{E}(X) &= \int_{-\infty}^{\infty} xf(x)dx \\
		\mathbb{V}(X) &= \int_{-\infty}^{\infty} \left( x-\mathbb{E}(X) \right)^2 f(x)dx
		\end{align*}
	\end{exampleblock}
\end{frame}

\begin{frame}
\begin{exampleblock}{Varianz und Erwartungswert einer diskreten Zufallsvariablen}
	Für $X$ diskret:
	\begin{align*}
	\mathbb{E}(X) = \sum_i x_i f(x_i), \ \
	\mathbb{V}(X) = \sum_i \left(x_i-\mathbb{E}(X)\right)^2 f(x_i)
	\end{align*}
	Für $X$ stetig:
	\begin{align*}
	\mathbb{E}(X) = \int_{-\infty}^{\infty} xf(x)dx, \ \
	\mathbb{V}(X) = \int_{-\infty}^{\infty} \left( x-\mathbb{E}(X) \right)^2 f(x)dx
	\end{align*}
\end{exampleblock}
\end{frame}

\begin{frame}
\begin{exampleblock}{Varianz und Erwartungswert einer diskreten Zufallsvariablen}
Für $X$ diskret:
	\begin{align*}
	\mathbb{E}(X) = \sum_i x_i f(x_i), \ \
	\mathbb{V}(X) = \sum_i \left(x_i-\mathbb{E}(X)\right)^2 f(x_i)
	\end{align*}
Für $X$ stetig:
	\begin{align*}
	\mathbb{E}(X) = \int_{-\infty}^{\infty} xf(x)dx, \ \
	\mathbb{V}(X) = \int_{-\infty}^{\infty} \left( x-\mathbb{E}(X) \right)^2 f(x)dx
	\end{align*}
\end{exampleblock}
Betrachte die Zufallsvariable $(X-\mathbb{E}(X))^2$.
\end{frame}

\begin{frame}
\begin{exampleblock}{Varianz und Erwartungswert einer diskreten Zufallsvariablen}
	Für $X$ diskret:
	\begin{align*}
	\mathbb{E}(X) = \sum_i x_i f(x_i), \ \
	\mathbb{V}(X) = \sum_i \left(x_i-\mathbb{E}(X)\right)^2 f(x_i)
	\end{align*}
	Für $X$ stetig:
	\begin{align*}
	\mathbb{E}(X) = \int_{-\infty}^{\infty} xf(x)dx, \ \
	\mathbb{V}(X) = \int_{-\infty}^{\infty} \left( x-\mathbb{E}(X) \right)^2 f(x)dx
	\end{align*}
\end{exampleblock}
Betrachte die Zufallsvariable $(X-\mathbb{E}(X))^2$. Dann gilt im diskreten und im stetigen Fall
\begin{align*}
\mathbb{V}(X) = \mathbb{E}\left[(X-\mathbb{E}(X))^2\right].
\end{align*}
\end{frame}

\begin{frame}
Im diskreten und im stetigen Fall gilt
\begin{align*}
\mathbb{V}(X) = \mathbb{E}\left[(X-\mathbb{E}(X))^2\right].
\end{align*}
\end{frame}

\begin{frame}
Im diskreten und im stetigen Fall gilt
\begin{align*}
\mathbb{V}(X) = \mathbb{E}\left[(X-\mathbb{E}(X))^2\right].
\end{align*}
\begin{alertblock}{Verschiebungssatz}
	Für eine diskrete oder eine stetige Zufallsvariable $X$ gilt
	\begin{align*}
	\mathbb{V}(X) =  \mathbb{E}(X^2)-\left( \mathbb{E}(X) \right)^2.
	\end{align*}
\end{alertblock}
\end{frame}

\begin{frame}
Im diskreten und im stetigen Fall gilt
\begin{align*}
\mathbb{V}(X) = \mathbb{E}\left[(X-\mathbb{E}(X))^2\right].
\end{align*}
\begin{alertblock}{Verschiebungssatz}
	Für eine diskrete oder eine stetige Zufallsvariable $X$ gilt
	\begin{align*}
	\mathbb{V}(X) =  \mathbb{E}(X^2)-\left( \mathbb{E}(X) \right)^2.
	\end{align*}
\end{alertblock}
\begin{align*}
\onslide<1->{\mathbb{V}(X) &= \mathbb{E}\left[(X-\mathbb{E}(X))^2\right] \\}
\onslide<2->{&= \mathbb{E} \left[ X^2 - 2X\mathbb{E}(X) + \left( \mathbb{E}(X) \right)^2   \right] \\}
\onslide<3->{&= \mathbb{E} (X^2) - \mathbb{E}(2X\mathbb{E}(X)) + \mathbb{E}\left[\left( \mathbb{E}(X) \right)^2\right] \\}
\onslide<4->{&= \mathbb{E} (X^2) - 2\mathbb{E}(X)\mathbb{E}(X) + \left( \mathbb{E}(X) \right)^2 \\}
\onslide<5>{&= \mathbb{E} (X^2) - \left( \mathbb{E}(X) \right)^2}
\end{align*}
\end{frame}



\end{document}