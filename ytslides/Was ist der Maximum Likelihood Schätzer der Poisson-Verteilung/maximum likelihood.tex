\documentclass[t,11pt]{beamer}
\usepackage{tikz}
\usetikzlibrary{calc}
\usepackage[utf8]{inputenc}
\usepackage[ngerman]{babel}
\usepackage{amsmath}
\usepackage{amsfonts}
\usepackage{amssymb}
\usepackage{framed}
\colorlet{LightBlue}{blue}

% Theorem box
\pgfdeclarelayer{background}
\pgfsetlayers{background,main}
\tikzstyle{thmbox} = [inner sep=1em]
\tikzstyle{thmborder} = [draw=blue, fill=none,line width =1.pt, rounded corners=5pt]

\def\parchmentframe#1{
	\tikz{
		\node[thmbox] (A) {#1};
		\begin{pgfonlayer}{background}
			\fill[thmborder] 
			(A.south east) -- (A.south west) -- 
			(A.north west) -- (A.north east) -- cycle;
\end{pgfonlayer}}}

\def\parchmentframetop#1{
	\tikz{
		\node[thmbox] (A) {#1};
		\begin{pgfonlayer}{background}    
			\fill[thmborder]
			(A.south west) -- (A.north west) -- 
			(A.north east) -- (A.south east);
\end{pgfonlayer}}}

\def\parchmentframebottom#1{
	\tikz{
		\node[thmbox] (A) {#1};
		\begin{pgfonlayer}{background}    
			\fill[thmborder]
			(A.north west) -- (A.south west) -- 
			(A.south east) -- (A.north east);
\end{pgfonlayer}}}

\def\parchmentframemiddle#1{
	\tikz{
		\node[thmbox] (A) {#1};
		\begin{pgfonlayer}{background}    
			\fill[thmborder]
			(A.north west) -- (A.south west);
			\fill[thmborder]
			(A.south east) -- (A.north east);
\end{pgfonlayer}}}

\newenvironment<>{myTheorem}[1][]{%
	\def\FrameCommand{\parchmentframe}%
	\def\FirstFrameCommand{\parchmentframetop}%
	\def\LastFrameCommand{\parchmentframebottom}%
	\def\MidFrameCommand{\parchmentframemiddle}%
	\vskip\baselineskip
	\MakeFramed{\FrameRestore}
	\noindent\tikz\node[inner sep=1.2ex, draw=blue, fill=blue!10,
	anchor=west, overlay, line width = 1.pt, rounded corners=4pt] at (0em, 1em) 
	{\color{LightBlue}{Aufgabe\if\relax\detokenize{#1}\relax\else\space (#1)\fi}};\par\nobreak}%
{\endMakeFramed}
%%end theorem box

\definecolor{pbblue}{HTML}{D8D8D8}% filling color for the progress bar
\definecolor{pbgray}{HTML}{F2F2F2}% background color for the progress bar

\useoutertheme{infolines}
\setbeamerfont{footline}{size=\normalsize}
\setbeamersize{text margin left=30pt,text margin right=30pt}
\makeatletter
\setbeamertemplate{footline}
{
	\leavevmode%
	\hbox{%
		\begin{beamercolorbox}[wd=.333333\paperwidth,ht=2.5ex,dp=1ex,center]{title in head/foot}%
			\usebeamerfont{title in head/foot}\insertshorttitle
		\end{beamercolorbox}%
		\begin{beamercolorbox}[wd=.333333\paperwidth,ht=2.5ex,dp=1ex,center]{date in head/foot}%
			%\usebeamerfont{date in head/foot}\insertshortdate{}\hspace*{2em}
			%\insertframenumber\hspace*{2ex} 
		\end{beamercolorbox}
		\begin{beamercolorbox}[wd=.333333\paperwidth,ht=3ex,dp=1ex,center]{author in head/foot}%
			\usebeamerfont{author in head/foot}\insertshortauthor~~%\beamer@ifempty{\insertshortinstitute}{}{(\insertshortinstitute)}
		\end{beamercolorbox}%
	}%
	\vskip0pt%
}
\makeatother


\makeatletter
\def\progressbar@progressbar{} % the progress bar
\newcount\progressbar@tmpcounta% auxiliary counter
\newcount\progressbar@tmpcountb% auxiliary counter
\newdimen\progressbar@pbht %progressbar height
\newdimen\progressbar@pbwd %progressbar width
\newdimen\progressbar@tmpdim % auxiliary dimension

\progressbar@pbwd=\linewidth
\progressbar@pbht=1.5ex

% the progress bar
\def\progressbar@progressbar{%

    \progressbar@tmpcounta=\insertframenumber
    \progressbar@tmpcountb=\inserttotalframenumber
    \progressbar@tmpdim=\progressbar@pbwd
    \multiply\progressbar@tmpdim by \progressbar@tmpcounta
    \divide\progressbar@tmpdim by \progressbar@tmpcountb

  \begin{tikzpicture}[rounded corners=3pt,very thin]

    \shade[top color=pbgray!20,bottom color=pbgray!20,middle color=pbgray!50]
      (0pt, 0pt) rectangle ++ (\progressbar@pbwd, \progressbar@pbht);

      \shade[draw=pbblue,top color=pbblue!50,bottom color=pbblue!50,middle color=pbblue] %
        (0pt, 0pt) rectangle ++ (\progressbar@tmpdim, \progressbar@pbht);

    \draw[color=normal text.fg!50]  
      (0pt, 0pt) rectangle (\progressbar@pbwd, \progressbar@pbht) 
        node[pos=0.5,color=normal text.fg] {die Aufgabe ist zu \textnormal{%
             \pgfmathparse{\insertframenumber*100/\inserttotalframenumber}%
             \pgfmathprintnumber[fixed,precision=0]{\pgfmathresult}\,\% gelöst%
        }%
    };
  \end{tikzpicture}%
}

\addtobeamertemplate{headline}{}
{%
  \begin{beamercolorbox}[wd=\paperwidth,ht=4ex,center,dp=1ex]{white}%
    \progressbar@progressbar%
  \end{beamercolorbox}%
}
\makeatother

\begin{document}
	\author{www.oilbat.de}
	%\title{Stochastik}
	\subtitle{}
	\logo{}
	\institute{}
	\date{}
	\subject{}
	\setbeamercovered{transparent}
	\setbeamertemplate{navigation symbols}{}

\addtocounter{framenumber}{-1}
\setbeamercovered{invisible}



\begin{frame}
\begin{myTheorem}
	Wie wird eine Maximum Likelihood Schätzfunktion hergeleitet? Formuliere die allgemeine Vorgehensweise.
\end{myTheorem}
\end{frame}

\begin{frame}
\begin{myTheorem}
	Wie wird eine Maximum Likelihood Schätzfunktion hergeleitet? Formuliere die allgemeine Vorgehensweise.
\end{myTheorem}
\begin{enumerate}
	\item Aufstellen der ML-Funktion: $L(\theta)=\prod_{i=1}^{n}f_{X_i}(\theta)$
	
\end{enumerate}
\end{frame}

\begin{frame}
\begin{myTheorem}
	Wie wird eine Maximum Likelihood Schätzfunktion hergeleitet? Formuliere die allgemeine Vorgehensweise.
\end{myTheorem}
\begin{enumerate}
	\item Aufstellen der ML-Funktion: $L(\theta)=\prod_{i=1}^{n}f_{X_i}(\theta)$
	\item Logarithmierung: $\ln(L(\theta))=\sum_{i=1}^{n}\ln(f_{X_i}(\theta))$
	
\end{enumerate}
\end{frame}

\begin{frame}
\begin{myTheorem}
	Wie wird eine Maximum Likelihood Schätzfunktion hergeleitet? Formuliere die allgemeine Vorgehensweise.
\end{myTheorem}
\begin{enumerate}
	\item Aufstellen der ML-Funktion: $L(\theta)=\prod_{i=1}^{n}f_{X_i}(\theta)$
	\item Logarithmierung: $\ln(L(\theta))=\sum_{i=1}^{n}\ln(f_{X_i}(\theta))$
	\item Bestimmung der Ableitung: $\frac{d}{d\theta}\ln(L(\theta))$
\end{enumerate}
\end{frame}

\begin{frame}
\begin{myTheorem}
	Wie wird eine Maximum Likelihood Schätzfunktion hergeleitet? Formuliere die allgemeine Vorgehensweise.
\end{myTheorem}
\begin{enumerate}
	\item Aufstellen der ML-Funktion: $L(\theta)=\prod_{i=1}^{n}f_{X_i}(\theta)$
	\item Logarithmierung: $\ln(L(\theta))=\sum_{i=1}^{n}\ln(f_{X_i}(\theta))$
	\item Bestimmung der Ableitung: $\frac{d}{d\theta}\ln(L(\theta))$
	\item Ableitung gleich Null setzen und nach dem gesuchten Parameter auflösen
\end{enumerate}
\end{frame}

\begin{frame}
\begin{myTheorem}
	Sei $X$ poissonverteilt zu dem Parameter $\lambda>0$. Wie lautet die Wahrscheinlichkeitsfunktion der Zufallsvariable $X$?
\end{myTheorem}

\end{frame}

\begin{frame}
\begin{myTheorem}
	Sei $X$ poissonverteilt zu dem Parameter $\lambda>0$. Wie lautet die Wahrscheinlichkeitsfunktion der Zufallsvariable $X$?
\end{myTheorem}
$P(X=k)=e^{-\lambda}\frac{\lambda^k}{k!},~k\in \mathbb{N}_0$
\end{frame}

\begin{frame}
\begin{myTheorem}
	Es seien $X_1,...,X_n$ poissonverteilte Zufallsvariablen zu dem Parameter $\lambda>0$. Führe die allgemeinen Schritte für diese Zufallsvariablen durch, um den Maximum Likelihood Schätzer der Poissonverteilung zu bestimmen.
\end{myTheorem}
\end{frame}

\begin{frame}
\begin{myTheorem}
	Es seien $X_1,...,X_n$ poissonverteilte Zufallsvariablen zu dem Parameter $\lambda>0$. Führe die allgemeinen Schritte für diese Zufallsvariablen durch, um den Maximum Likelihood Schätzer der Poissonverteilung zu bestimmen.
\end{myTheorem}
\begin{enumerate}
	\item $L(\lambda)=\prod_{i=1}^{n}f_{X_i}(\lambda)=\prod_{i=1}^{n}e^{-\lambda}\frac{\lambda^{X_i}}{X_i!}$
\end{enumerate}
\end{frame}

\begin{frame}
\begin{myTheorem}
	Es seien $X_1,...,X_n$ poissonverteilte Zufallsvariablen zu dem Parameter $\lambda>0$. Führe die allgemeinen Schritte für diese Zufallsvariablen durch, um den Maximum Likelihood Schätzer der Poissonverteilung zu bestimmen.
\end{myTheorem}
\begin{enumerate}
	\item $L(\lambda)=\prod_{i=1}^{n}f_{X_i}(\lambda)=\prod_{i=1}^{n}e^{-\lambda}\frac{\lambda^{X_i}}{X_i!}$
	\item $\ln(L(\lambda))=\sum_{i=1}^{n}\ln(e^{-\lambda}\frac{\lambda^{X_i}}{X_i!})=\sum_{i=1}^{n}(-\lambda+X_i\ln(\lambda)-\ln(X_i!))=-n\lambda + \ln (\lambda) \sum_{i=1}^{n} X_i - \sum_{i=1}^{n} \ln (X_i!)$
\end{enumerate}
\end{frame}

\begin{frame}
\begin{myTheorem}
	Es seien $X_1,...,X_n$ poissonverteilte Zufallsvariablen zu dem Parameter $\lambda>0$. Führe die allgemeinen Schritte für diese Zufallsvariablen durch, um den Maximum Likelihood Schätzer der Poissonverteilung zu bestimmen.
\end{myTheorem}
\begin{enumerate}
	\item $L(\lambda)=\prod_{i=1}^{n}f_{X_i}(\lambda)=\prod_{i=1}^{n}e^{-\lambda}\frac{\lambda^{X_i}}{X_i!}$
	\item $\ln(L(\lambda))=\sum_{i=1}^{n}\ln(e^{-\lambda}\frac{\lambda^{X_i}}{X_i!})=\sum_{i=1}^{n}(-\lambda+X_i\ln(\lambda)-\ln(X_i!))=-n\lambda + \ln (\lambda) \sum_{i=1}^{n} X_i - \sum_{i=1}^{n} \ln (X_i!)$
	\item $\frac{d}{d\lambda}\ln(L(\lambda))=-n+ \frac{1}{\lambda} \sum_{i=1}^{n} X_i$
\end{enumerate}
\end{frame}

\begin{frame}
\begin{myTheorem}
	Es seien $X_1,...,X_n$ poissonverteilte Zufallsvariablen zu dem Parameter $\lambda>0$. Führe die allgemeinen Schritte für diese Zufallsvariablen durch, um den Maximum Likelihood Schätzer der Poissonverteilung zu bestimmen.
\end{myTheorem}
\begin{enumerate}
	\item $L(\lambda)=\prod_{i=1}^{n}f_{X_i}(\lambda)=\prod_{i=1}^{n}e^{-\lambda}\frac{\lambda^{X_i}}{X_i!}$
	\item $\ln(L(\lambda))=\sum_{i=1}^{n}\ln(e^{-\lambda}\frac{\lambda^{X_i}}{X_i!})=\sum_{i=1}^{n}(-\lambda+X_i\ln(\lambda)-\ln(X_i!))=-n\lambda + \ln (\lambda) \sum_{i=1}^{n} X_i - \sum_{i=1}^{n} \ln (X_i!)$
	\item $\frac{d}{d\lambda}\ln(L(\lambda))=-n+ \frac{1}{\lambda} \sum_{i=1}^{n} X_i$
	\item $-n+ \frac{1}{\lambda} \sum_{i=1}^{n} X_i\overset{!}{=}0 \Rightarrow \hat{\lambda}_{ML}=\frac{1}{n} \sum_{i=1}^{n} X_i=\bar{X}$
\end{enumerate}
\end{frame}

\begin{frame}
\begin{myTheorem}
	Es liegen nun die folgenden Realisationen der Zufallsvariablen vor:
	$$x_1=4,x_2=5,x_3=1,x_4=2,x_5=7,x_6=11,x_7=2,x_8=8$$
	Bestimme den Maximum Likelihood Schätzwert $\hat{\lambda}_{ML}$ für $\lambda$.
\end{myTheorem}
\end{frame}

\begin{frame}
\begin{myTheorem}
	Es liegen nun die folgenden Realisationen der Zufallsvariablen vor:
	$$x_1=4,x_2=5,x_3=1,x_4=2,x_5=7,x_6=11,x_7=2,x_8=8$$
	Bestimme den Maximum Likelihood Schätzwert $\hat{\lambda}_{ML}$ für $\lambda$.
\end{myTheorem}
$\hat{\lambda}_{ML}=\frac{1}{8} \sum_{i=1}^{8} x_i = \frac{1}{8}\cdot 40 =5$
\end{frame}

\begin{frame}
\begin{myTheorem}
	Mit Hilfe der Maximum Likelihood Methode wird aus einer Menge an möglichen Parametern ein Parameter bestimmt, der als Schätzwert für die zugrunde liegende Verteilung dient. Allgemein formuliert, welcher Parameter wird bei der Maximum Likelihood Methode ausgewählt?
\end{myTheorem}
\end{frame}

\begin{frame}
\begin{myTheorem}
	Mit Hilfe der Maximum Likelihood Methode wird aus einer Menge an möglichen Parametern ein Parameter bestimmt, der als Schätzwert für die zugrunde liegende Verteilung dient. Allgemein formuliert, welcher Parameter wird bei der Maximum Likelihood Methode ausgewählt?
\end{myTheorem}
Es wird derjenige Parameter ausgewählt, der die Stichprobe am plausibelsten erscheinen lässt.
\end{frame}


\end{document}