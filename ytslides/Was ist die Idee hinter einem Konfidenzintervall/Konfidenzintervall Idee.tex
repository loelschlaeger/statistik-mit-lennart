\documentclass[t,11pt]{beamer}
\usepackage{tikz}
\usepackage{pgfplots}
\usetikzlibrary{calc}
\usepackage[utf8]{inputenc}
\usepackage[ngerman]{babel}
\usepackage{amsmath,amsfonts,amssymb}
\usepackage{framed}
\usecolortheme{orchid}
\usepackage{etoolbox}
\useinnertheme[shadow=true]{rounded}

%%% PROGRESSBAR
\definecolor{pbblue}{HTML}{D8D8D8}% filling color for the progress bar
\definecolor{pbgray}{HTML}{F2F2F2}% background color for the progress bar
\useoutertheme{infolines}
\setbeamerfont{footline}{size=\normalsize}
\setbeamersize{text margin left=30pt,text margin right=30pt}
\makeatletter
\setbeamertemplate{footline}
{
	\leavevmode%
	\hbox{%
		\begin{beamercolorbox}[wd=.333333\paperwidth,ht=2.5ex,dp=1ex,center]{title in head/foot}%
			\usebeamerfont{title in head/foot}\insertshorttitle
		\end{beamercolorbox}%
		\begin{beamercolorbox}[wd=.333333\paperwidth,ht=2.5ex,dp=1ex,center]{date in head/foot}%
			%\usebeamerfont{date in head/foot}\insertshortdate{}\hspace*{2em}
			%\insertframenumber\hspace*{2ex} 
		\end{beamercolorbox}
		\begin{beamercolorbox}[wd=.333333\paperwidth,ht=3ex,dp=1ex,center]{author in head/foot}%
			\usebeamerfont{author in head/foot}\insertshortauthor~~%\beamer@ifempty{\insertshortinstitute}{}{(\insertshortinstitute)}
		\end{beamercolorbox}%
	}%
	\vskip0pt%
}
\makeatother
\makeatletter
\def\progressbar@progressbar{} % the progress bar
\newcount\progressbar@tmpcounta% auxiliary counter
\newcount\progressbar@tmpcountb% auxiliary counter
\newdimen\progressbar@pbht %progressbar height
\newdimen\progressbar@pbwd %progressbar width
\newdimen\progressbar@tmpdim % auxiliary dimension
\progressbar@pbwd=\linewidth
\progressbar@pbht=1.5ex
\def\progressbar@progressbar{%
    \progressbar@tmpcounta=\insertframenumber
    \progressbar@tmpcountb=\inserttotalframenumber
    \progressbar@tmpdim=\progressbar@pbwd
    \multiply\progressbar@tmpdim by \progressbar@tmpcounta
    \divide\progressbar@tmpdim by \progressbar@tmpcountb
  \begin{tikzpicture}[rounded corners=3pt,very thin]
    \shade[top color=pbgray!20,bottom color=pbgray!20,middle color=pbgray!50]
      (0pt, 0pt) rectangle ++ (\progressbar@pbwd, \progressbar@pbht);
      \shade[draw=pbblue,top color=pbblue!50,bottom color=pbblue!50,middle color=pbblue] %
        (0pt, 0pt) rectangle ++ (\progressbar@tmpdim, \progressbar@pbht);
    \draw[color=normal text.fg!50]  
      (0pt, 0pt) rectangle (\progressbar@pbwd, \progressbar@pbht) 
        node[pos=0.5,color=normal text.fg] {\textnormal{%
             \pgfmathparse{\insertframenumber*100/\inserttotalframenumber}%
             \pgfmathprintnumber[fixed,precision=0]{\pgfmathresult}\,\%%
        }%
    };
  \end{tikzpicture}%
}
\addtobeamertemplate{headline}{}
{%
  \begin{beamercolorbox}[wd=\paperwidth,ht=4ex,center,dp=1ex]{white}%
    \progressbar@progressbar%
  \end{beamercolorbox}%
}
\makeatother


%%% BLOCKS
% block = Aufgabe
\setbeamercolor{block title}{fg=black,bg=blue!30!white} 
\setbeamercolor{block body}{fg=black, bg=blue!3!white}

% alertblock = Definition
\setbeamercolor{block title alerted}{fg=black,bg=red!50!white} 
\setbeamercolor{block body alerted}{fg=black, bg=red!3!white}

% exampleblock = Wiederholung
\setbeamercolor{block title example}{fg=black,bg=green!30!white} 
\setbeamercolor{block body example}{fg=black, bg=green!3!white}



\begin{document}
	%\author{www.oilbat.de}
	%\title{Stochastik}
	\subtitle{}
	\logo{}
	\institute{}
	\date{}
	\subject{}
	\setbeamercovered{transparent}
	\setbeamertemplate{navigation symbols}{}

\addtocounter{framenumber}{-1}
\setbeamercovered{invisible}

\begin{frame}
\begin{exampleblock}{Ein Beispiel}
	Gegeben ist diese Stichprobe: 
	\begin{align*}
	2.4,~2.0,~3.0,~2.4,~2.9,~2.4,~2.6,~2.6,~1.6,~2.2
	\end{align*}
	Sie stammt aus einer Normalverteilung. Wie groß ist $\mu$?
	\begin{align*}
	\hat{\mu}_{ML} = \hat{\mu}_{MM} = \bar{x} = 2.4?
	\end{align*}
	Oder lieber: $\mu$ wird in $[2.0,2.8]$ liegen?
\end{exampleblock}
\end{frame}

\begin{frame}
	\begin{block}{Warum Konfidenzintervalle?}
		\begin{itemize}
			\item Punktschätzung (ML- oder Momenten-Methode) liefert uns einen Schätzwert $\hat{\theta}$
			\item der ist in der Regel nicht identisch mit dem wahren Parameter $\theta$
			\item auch ist unklar, wie nahe $\hat{\theta}$ an $\theta$ dran liegt
			\item besser: Intervall $[a,b]$, das $\theta$ einfängt
			\item gesucht ist ein Verfahren, das mit hoher Wahrscheinlichkeit ein Intervall liefert, das $\theta$ enthält
			\item die sogenannte Vertrauenswahrscheinlichkeit wollen wir festlegen können
		\end{itemize}
	\end{block}
\end{frame}

\begin{frame}
	\begin{alertblock}{Achtung: Die richtige Formulierung}
			\textbf{Falsch ist:} 
			
			$\mu$ liegt mit 95\% Wahrscheinlichkeit in $[2.0,2.8]$
			
			\vspace{0.2cm}
			\textbf{Richtig ist:} 
			
			Mit 95\% Wahrscheinlichkeit hat die beobachtete Stichprobe ein Intervall erzeugt, dass $\mu$ enthält.

	\end{alertblock}
\end{frame}

\begin{frame}
	\begin{block}{Definition: $(1-\alpha)$-Konfidenzintervall}
		Gegeben sei eine Irrtumswahrscheinlichkeit $\alpha$ und eine \textit{iid} Stichprobe $X_1,\dots,X_n$. Dann bilden die Stichprobenfunktionen
		\begin{itemize}
			\item $G_u = G_u(X_1,\dots,X_n)$
			\item und $G_o = G_o(X_1,\dots,X_n)$
		\end{itemize}
		durch $[G_u,G_o]$ ein $(1-\alpha)$-Konfidenzintervall für $\theta$, falls
		\begin{align*}
			P(G_u\leq \theta \leq G_o) = 1-\alpha.
		\end{align*}
	\end{block}
\end{frame}



\end{document}