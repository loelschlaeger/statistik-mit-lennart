\documentclass[t,11pt]{beamer}
\usepackage{tikz}
\usepackage{pgfplots}
\usetikzlibrary{calc}
\usepackage[utf8]{inputenc}
\usepackage[ngerman]{babel}
\usepackage{amsmath,amsfonts,amssymb}
\usepackage{framed}
\usecolortheme{orchid}
\usepackage{etoolbox}
\useinnertheme[shadow=true]{rounded}

%%% PROGRESSBAR
\definecolor{pbblue}{HTML}{D8D8D8}% filling color for the progress bar
\definecolor{pbgray}{HTML}{F2F2F2}% background color for the progress bar
\useoutertheme{infolines}
\setbeamerfont{footline}{size=\normalsize}
\setbeamersize{text margin left=30pt,text margin right=30pt}
\makeatletter
\setbeamertemplate{footline}
{
	\leavevmode%
	\hbox{%
		\begin{beamercolorbox}[wd=.333333\paperwidth,ht=2.5ex,dp=1ex,center]{title in head/foot}%
			\usebeamerfont{title in head/foot}\insertshorttitle
		\end{beamercolorbox}%
		\begin{beamercolorbox}[wd=.333333\paperwidth,ht=2.5ex,dp=1ex,center]{date in head/foot}%
			%\usebeamerfont{date in head/foot}\insertshortdate{}\hspace*{2em}
			%\insertframenumber\hspace*{2ex} 
		\end{beamercolorbox}
		\begin{beamercolorbox}[wd=.333333\paperwidth,ht=3ex,dp=1ex,center]{author in head/foot}%
			\usebeamerfont{author in head/foot}\insertshortauthor~~%\beamer@ifempty{\insertshortinstitute}{}{(\insertshortinstitute)}
		\end{beamercolorbox}%
	}%
	\vskip0pt%
}
\makeatother
\makeatletter
\def\progressbar@progressbar{} % the progress bar
\newcount\progressbar@tmpcounta% auxiliary counter
\newcount\progressbar@tmpcountb% auxiliary counter
\newdimen\progressbar@pbht %progressbar height
\newdimen\progressbar@pbwd %progressbar width
\newdimen\progressbar@tmpdim % auxiliary dimension
\progressbar@pbwd=\linewidth
\progressbar@pbht=1.5ex
\def\progressbar@progressbar{%
    \progressbar@tmpcounta=\insertframenumber
    \progressbar@tmpcountb=\inserttotalframenumber
    \progressbar@tmpdim=\progressbar@pbwd
    \multiply\progressbar@tmpdim by \progressbar@tmpcounta
    \divide\progressbar@tmpdim by \progressbar@tmpcountb
  \begin{tikzpicture}[rounded corners=3pt,very thin]
    \shade[top color=pbgray!20,bottom color=pbgray!20,middle color=pbgray!50]
      (0pt, 0pt) rectangle ++ (\progressbar@pbwd, \progressbar@pbht);
      \shade[draw=pbblue,top color=pbblue!50,bottom color=pbblue!50,middle color=pbblue] %
        (0pt, 0pt) rectangle ++ (\progressbar@tmpdim, \progressbar@pbht);
    \draw[color=normal text.fg!50]  
      (0pt, 0pt) rectangle (\progressbar@pbwd, \progressbar@pbht) 
        node[pos=0.5,color=normal text.fg] {\textnormal{%
             \pgfmathparse{\insertframenumber*100/\inserttotalframenumber}%
             \pgfmathprintnumber[fixed,precision=0]{\pgfmathresult}\,\%%
        }%
    };
  \end{tikzpicture}%
}
\addtobeamertemplate{headline}{}
{%
  \begin{beamercolorbox}[wd=\paperwidth,ht=4ex,center,dp=1ex]{white}%
    \progressbar@progressbar%
  \end{beamercolorbox}%
}
\makeatother


%%% BLOCKS
% block = Aufgabe
\setbeamercolor{block title}{fg=black,bg=blue!30!white} 
\setbeamercolor{block body}{fg=black, bg=blue!3!white}

% alertblock = Definition
\setbeamercolor{block title alerted}{fg=black,bg=red!50!white} 
\setbeamercolor{block body alerted}{fg=black, bg=red!3!white}

% exampleblock = Wiederholung
\setbeamercolor{block title example}{fg=black,bg=green!30!white} 
\setbeamercolor{block body example}{fg=black, bg=green!3!white}



\begin{document}
	%\author{www.oilbat.de}
	%\title{Stochastik}
	\subtitle{}
	\logo{}
	\institute{}
	\date{}
	\subject{}
	\setbeamercovered{transparent}
	\setbeamertemplate{navigation symbols}{}

\addtocounter{framenumber}{-1}
\setbeamercovered{invisible}

\begin{frame}
\begin{block}{Aufgabe}
	Wir wollen annehmen, dass sich die Akkuladedauer für Handys gut durch eine Normalverteilung beschreiben lässt. Wir beobachten folgenden Ladedauern (in Minuten):
	\begin{align*}
		106,~108,~113,~95,~102,~94,~101,~97,~98,~96 
	\end{align*}
	Ermitteln Sie ein 95\%-KI für den Erwartungswert~$\mu$.
\end{block}
\end{frame}

\begin{frame}
\begin{block}{Aufgabe}
	Wir wollen annehmen, dass sich die Akkuladedauer für Handys gut durch eine Normalverteilung beschreiben lässt. Wir beobachten folgenden Ladedauern (in Minuten):
	\begin{align*}
	106,~108,~113,~95,~102,~94,~101,~97,~98,~96 
	\end{align*}
	Ermitteln Sie ein 95\%-KI für den Erwartungswert~$\mu$.
\end{block}
\begin{alertblock}{Definition: $1-\alpha$-KI für $\mu$ falls $\sigma$ unbekannt ist}
	\begin{align*}
	\left[ \bar{X} - t_{1-\alpha/2; n-1} \frac{S}{\sqrt{n}};~\bar{X} + t_{1-\alpha/2; n-1} \frac{S}{\sqrt{n}} \right]
	\end{align*}
	\vspace{-0.5cm}
	\begin{itemize}
		\item $S=\sqrt{\frac{1}{n-1}\sum_{i=1}^{n}(X_i-\bar{X})^2}$
		\item $t_{1-\alpha/2; n-1}$ ist das $1-\alpha/2$-Quantil der t-Verteilung mit $n-1$ Freiheitsgeraden
	\end{itemize}
\end{alertblock}
\end{frame}

\begin{frame}
\begin{align*}
\frac{\bar{X}-\mu}{S/\sqrt{n}}\sim t_{n-1}
\end{align*}
\end{frame}

\begin{frame}
\begin{align*}
\frac{\bar{X}-\mu}{S/\sqrt{n}}\sim t_{n-1}
\end{align*}
\begin{align*}
1-\alpha &= P(-t_{1-\alpha/2; n-1} \leq \frac{\bar{X}-\mu}{S/\sqrt{n}} \leq t_{1-\alpha/2; n-1} ) \\
&= P(-t_{1-\alpha/2; n-1}  \frac{S}{\sqrt{n}}\leq \bar{X}-\mu \leq t_{1-\alpha/2; n-1} \frac{S}{\sqrt{n}}) \\
&= P(\bar{X} -t_{1-\alpha/2; n-1}  \frac{S}{\sqrt{n}}\leq \mu \leq \bar{X} + t_{1-\alpha/2; n-1}  \frac{S}{\sqrt{n}})
\end{align*}
\end{frame}

\begin{frame}
\begin{align*}
\frac{\bar{X}-\mu}{S/\sqrt{n}}\sim t_{n-1}
\end{align*}
\begin{align*}
1-\alpha &= P(-t_{1-\alpha/2; n-1} \leq \frac{\bar{X}-\mu}{S/\sqrt{n}} \leq t_{1-\alpha/2; n-1} ) \\
&= P(-t_{1-\alpha/2; n-1}  \frac{S}{\sqrt{n}}\leq \bar{X}-\mu \leq t_{1-\alpha/2; n-1} \frac{S}{\sqrt{n}}) \\
&= P(\bar{X} -t_{1-\alpha/2; n-1}  \frac{S}{\sqrt{n}}\leq \mu \leq \bar{X} + t_{1-\alpha/2; n-1}  \frac{S}{\sqrt{n}})
\end{align*}
\begin{align*}
\left[ \bar{X} - t_{1-\alpha/2; n-1} \frac{S}{\sqrt{n}};~\bar{X} + t_{1-\alpha/2; n-1} \frac{S}{\sqrt{n}} \right]
\end{align*}
\end{frame}

\begin{frame}
	\begin{block}{Aufgabe}
		Wir wollen annehmen, dass sich die Akkuladedauer für Handys gut durch eine Normalverteilung beschreiben lässt. Wir beobachten folgenden Ladedauern (in Minuten):
		\begin{align*}
		106,~108,~113,~95,~102,~94,~101,~97,~98,~96 
		\end{align*}
		Ermitteln Sie ein 95\%-KI für den Erwartungswert~$\mu$.
	\end{block}
	\begin{itemize}
		\item Arithmetisches Stichprobenmittel: $\bar{x} = 101$
		\item Irrtumsniveau: $\alpha=5\%$
		\item Stichprobenumfang ist $n=10$
		\item Quantil $t_{1-\alpha/2; n-1} = t_{0,975;9} = 2,26$
		\item Die \textbf{geschätzte} Standardabweichung ist $s=6,27$
	\end{itemize}
\end{frame}

\begin{frame}
\begin{itemize}
	\item Arithmetisches Stichprobenmittel: $\bar{x} = 101$
	\item Irrtumsniveau: $\alpha=5\%$
	\item Stichprobenumfang ist $n=10$
	\item Quantil $t_{1-\alpha/2; n-1} = t_{0,975;9} = 2,26$
	\item Die \textbf{geschätzte} Standardabweichung ist $s=6,27$
\end{itemize}
\begin{alertblock}{Definition: $1-\alpha$-KI für $\mu$ falls $\sigma$ unbekannt ist}
	\begin{align*}
	\left[ \bar{X} - t_{1-\alpha/2; n-1} \frac{S}{\sqrt{n}};~\bar{X} + t_{1-\alpha/2; n-1} \frac{S}{\sqrt{n}} \right]
	\end{align*}
\end{alertblock}
\end{frame}

\begin{frame}
\begin{itemize}
	\item Arithmetisches Stichprobenmittel: $\bar{x} = 101$
	\item Irrtumsniveau: $\alpha=5\%$
	\item Stichprobenumfang ist $n=10$
	\item Quantil $t_{1-\alpha/2; n-1} = t_{0,975;9} = 2,26$
	\item Die \textbf{geschätzte} Standardabweichung ist $s=6,27$
\end{itemize}
\begin{alertblock}{Definition: $1-\alpha$-KI für $\mu$ falls $\sigma$ unbekannt ist}
	\begin{align*}
	\left[ \bar{X} - t_{1-\alpha/2; n-1} \frac{S}{\sqrt{n}};~\bar{X} + t_{1-\alpha/2; n-1} \frac{S}{\sqrt{n}} \right]
	\end{align*}
\end{alertblock}
\begin{align*}
\left[ 101 - 2,26\cdot \frac{6,27}{\sqrt{10}};~ 101 + 2,26\cdot \frac{6,27}{\sqrt{10}} \right] = \left[ 96,5;~105,5 \right]
\end{align*}
\end{frame}

\end{document}